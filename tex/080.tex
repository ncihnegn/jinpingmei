
\chapter{潘金蓮售色赴東床 李嬌兒盜財歸麗院}

詩曰:
\begin{quote}
倚醉無端尋舊約,卻因惆悵轉難勝。
靜中樓閣深春雨,遠處簾櫳半夜燈。
抱柱立時風細細,繞廊行處思騰騰。
分明窗下聞裁剪,敲遍欄桿喚不應。
\end{quote}

話說西門慶死了,首七那日,卻是報國寺十六眾僧人做水陸。這應伯爵約會了謝希大、花子繇、祝實念、孫天化、常峙節、白賚光七人,坐在一處,伯爵先開口說:「大官人沒了,今一七光景。你我相交一場,當時也曾吃過他的,也曾用過他的,也曾使過他的,也曾借過他的。今日他死了,莫非推不知道?灑土也眯眯後人眼睛兒,他就到五閻王跟前,也不饒你我。如今這等計較,你我各出一錢銀子,七人共湊上七錢,辦一桌祭禮,買一幅軸子,再求水先生作一篇祭文,抬了去,大官人靈前祭奠祭奠,少不的還討了他七分銀子一條孝絹來,這個好不好?」眾人都道:「哥說的是。」當下每人湊出銀子來,交與伯爵,整備祭物停當,買了軸子,央水秀才做了祭文。這水秀才平昔知道應伯爵這起人,與西門慶乃小人之朋,於是暗含譏刺,作就一篇祭文。伯爵眾人把祭祀抬到靈前擺下,陳敬濟穿孝在旁還禮。伯爵為首,各人上了香,人人都粗俗,那裡曉得其中滋味。澆了奠酒,只顧把祝文宣念。其文略曰:
\begin{quote}
維重和元年,歲戊戌,二月戊子期,越初三日庚寅,侍教生應伯爵、謝希大、花子繇、祝實念、孫天化、常峙節、白賚光,謹以清酌庶饈之儀,致祭於故錦衣西門大官人之靈曰:維靈生前梗直,秉性堅剛;軟的不怕,硬的不降。常濟人以點水,恆助人以精光。囊篋頗厚,氣概軒昂。逢樂而舉,遇陰伏降。錦襠隊中居住,齊腰庫里收藏。有八角而不用撓摑,逢虱蟣而騷癢難當。受恩小子,常在胯下隨幫。也曾在章台而宿柳,也曾在謝館而猖狂。正宜撐頭活腦,久戰熬場,胡為罹一疾不起之殃?見今你便長伸著腳子去了,丟下小子輩,如班鳩跌腳,倚靠何方?難上他煙花之寨,難靠他八字紅牆。再不得同席而儇軟玉,再不得並馬而傍溫香。撇的人垂頭落腳,閃的人牢溫郎當。今特奠茲白濁,次獻寸觴。靈其不昧,來格來歆。尚享。
\end{quote}

眾人祭畢,陳敬濟下來還禮,請去捲棚內三湯五割,管待出門不題。

且說那日院中李家虔婆,聽見西門慶死了,鋪謀定計,備了一張祭桌,使了李桂卿、李桂姐坐轎子來上紙弔問。月娘不出來,都是李嬌兒、孟玉樓在上房管待。李家桂卿、桂姐悄悄對李嬌兒說:「俺媽說,人已是死了,你我院中人,守不的這樣貞節!自古千里長棚,沒個不散的筵席。教你手裡有東西,悄悄教李銘稍了家去防後。你還恁傻!常言道:『揚州雖好,不是久戀之家。』不拘多少時,也少不的離他家門。」那李嬌兒聽記在心。

不想那日韓道國妻王六兒,亦備了張祭桌,喬素打扮,坐轎子來與西門慶燒紙。在靈前擺下祭祀,只顧站著。站了半日,白沒個人兒出來陪待。原來西門慶死了,首七時分,就把王經打發家去不用了。小廝每見王六兒來,都不敢進去說。那來安兒不知就裡,到月娘房裡,向月娘說:「韓大嬸來與爹上紙,在前邊站了一日了,大舅使我來對娘說。」這吳月娘心中還氣忿不過,便喝罵道:「怪賊奴才,不與我走,還來甚麼韓大嬸、屄大嬸,賊狗攮的養漢淫婦,把人家弄的家敗人亡,父南子北,夫逃妻散的,還來上甚麼屄紙!」一頓罵的來安兒摸門不著,來到靈前。吳大舅問道:「對後邊說了不曾?」來安兒把嘴谷都著不言語。問了半日,才說:「娘稍出四馬兒來了。」這吳大舅連忙進去,對月娘說:「姐姐,你怎麼這等的?快休要舒口!自古人惡禮不惡。他男子漢領著咱偌多的本錢,你如何這等待人?好名兒難得,快休如此。你就不出去,教二姐姐、三姐姐好好待他出去,也是一般。做甚麼恁樣的,教人說你不是。」那月娘見他哥這樣說,才不言語了。良久,孟玉樓出來,還了禮,陪他在靈前坐的。只吃一鐘茶,婦人也有些省口,就坐不住,隨即告辭起身去了。正是:
\begin{quote}
誰人汲得西江水,難免今朝一面羞。
\end{quote}

那李桂卿、桂姐、吳銀兒都在上房坐著,見月娘罵韓道國老婆淫婦長、淫婦短,砍一株損百枝,兩個就有些坐不住,未到日落,就要家去。月娘再三留他姐兒兩個:「晚夕伙計每伴宿,你每看了提偶,明日去罷。」留了半日,桂姐、銀姐不去了,只打發他姐姐桂卿家去了。到了晚夕,僧人散了,果然有許多街坊、伙計、主管,喬大戶、吳大舅、吳二舅、沈姨父、花子繇、應伯爵、謝希大、常峙節,也有二十餘人,叫了一起偶戲,在大卷棚內,擺設酒席伴宿。提演的是「孫榮、孫華殺狗勸夫」戲文。堂客都在靈旁廳內,圍著幃屏,放下簾來,擺放桌席,朝外觀看。李銘、吳惠在這裡答應,晚夕也不家去了。不一時,眾人都到齊了。祭祀已畢,捲棚內點起燭來,安席坐下,打動鼓樂,戲文上來。直搬演到三更天氣,戲文方了。

原來陳敬濟自從西門慶死後,無一日不和潘金蓮兩個嘲戲,或在靈前溜眼,帳子後調笑。於是趕人散一亂,眾堂客都往後邊去了,小廝每都收家活,這金蓮趕眼錯,捏了敬濟一把,說道:「我兒,你娘今日成就了你罷。趁大姐在後邊,咱就往你屋裡去罷。」敬濟聽了,得不的一聲,先往屋裡開門去了。婦人黑影里,抽身鑽入他房內,更不答話,解開褲子,仰臥在炕上,雙鳧飛首,教陳敬濟好耍。正是:
\begin{quote}
色膽如天怕甚事,鴛幃雲雨百年情。
\end{quote}
真個是:
\begin{quote}
二載相逢,一朝配偶;數年姻眷,一旦和諧。一個柳腰款擺,一個玉莖忙舒。耳邊訴雨意雲情,枕上說山盟海誓。鶯恣蝶採,旖妮搏弄百千般;狂雨羞雲,嬌媚施逞千萬態。一個不住叫親親,一個摟抱呼達達。得多少柳色乍翻新樣綠,花容不減舊時紅。
\end{quote}

霎時雲雨了畢,婦人恐怕人來,連忙出房,往後邊去了。到次日,這小伙兒嘗著這個甜頭兒,早辰走到金蓮房來,金蓮還在被窩裡未起來。從窗眼裡張看,見婦人被擁紅雲,粉腮印玉,說道:「好管庫房的,這咱還不起來!今日喬親家爹來上祭,大娘分付把昨日擺的李三、黃四家那祭桌收進來罷。你快些起來,且拿鑰匙出來與我。」婦人連忙教春梅拿鑰匙與敬濟,敬濟先教春梅樓上開門去了。婦人便從窗眼裡遞出舌頭,兩個咂了一回。正是得多少脂香滿口涎空咽,甜唾顒心溢肺姦。有詞為證:
\begin{quote}
恨杜鵑聲透珠簾。心似針簽,情似膠粘。我則見笑臉腮窩愁粉黛,瘦損春纖寶髻亂,雲松翠鈿。睡顏酡,玉減紅添。檀口曾沾。到如今唇上猶香,想起來口內猶甜。
\end{quote}

良久,春梅樓上開了門,敬濟往前邊看搬祭祀去了。不一時,喬大戶家祭來擺下。喬大戶娘子並喬大戶許多親眷,靈前祭畢。吳大舅、吳二舅、甘伙計陪侍,請至捲棚內管待。李銘、吳惠彈唱。那日鄭愛月兒家也來上紙弔孝。月娘俱令玉樓打發了孝裙束腰,後邊與堂客一同坐的。鄭愛月兒看見李桂姐、吳銀姐都在這裡,便嗔他兩個不對他說:「我若知道爹沒了,有個不來的!你每好人兒,就不會我會兒去。」又見月娘生了孩兒,說道:「娘一喜一憂。惜乎爹只是去世太早了些兒,你老人家有了主兒,也不愁。」月娘俱打發了孝,留坐至晚方散。

到二月初三日,西門慶二七,玉皇廟吳道官十六眾道士,在家念經做法事。那日衙門中何千戶作創,約會了劉、薛二內相,周守備、荊都統、張團練、雲指揮等數員武官,合著上了壇祭。月娘這裡請了喬大戶、吳大舅、應伯爵來陪待,李銘、吳惠兩個小優兒彈唱,捲棚管待去了。俱不必細說。到晚夕念經送亡。月娘分付把李瓶兒靈床連影抬出去,一把火燒了。將箱籠都搬到上房內堆放。奶子如意兒並迎春收在後邊答應,把繡春與了李嬌兒房內使喚。將李瓶兒那邊房門,一把鎖鎖了。可憐正是:
\begin{quote}
畫棟雕梁猶未乾,堂前不見痴心客。
\end{quote}
有詩為證:
\begin{quote}
襄王臺下水悠悠,一種相思兩樣愁。
月色不如人事改,夜深還到粉牆頭。
\end{quote}

那時李銘日日假以孝堂助忙,暗暗教李嬌兒偷轉東西與他掖送到家,又來答應,常兩三夜不往家去,只瞞過月娘一人眼目。吳二舅又和李嬌兒舊有首尾,誰敢道個不字。初九日念了三七經,月娘出了暗房,四七就沒曾念經。十二日,陳敬濟破了土回來。二十日早發引,也有許多冥器紙札,送殯之人終不似李瓶兒那時稠密。臨棺材出門,也請了報恩寺朗僧官起棺,坐在轎上,捧的高高的,念了幾句偈文。念畢,陳敬濟摔破紙盆,棺材起身,合家大小孝眷放聲號哭。吳月娘坐魂轎,後面坐堂客上轎,都圍隨材走,徑出南門外五里原祖塋安厝。陳敬濟備了一匹尺頭,請雲指揮點了神主,陰陽徐先生下了葬。眾孝眷掩土畢。山頭祭桌,可憐通不上幾家,只是吳大舅、喬大戶、何千戶、沈姨夫、韓姨夫與眾伙計五六處而已。吳道官還留下十二眾道童回靈,安於上房明間正寢。陰陽灑掃已畢,打發眾親戚出門。吳月娘等不免伴夫靈守孝。一日暖了墓回來,答應班上排軍節級,各都告辭回衙門去了。西門慶五七,月娘請了薛姑子、王姑子、大師父、十二眾尼僧,在家誦經禮懺,超度夫主生天。吳大妗子並吳舜臣媳婦,都在家中相伴。

原來出殯之時,李桂卿同桂姐在山頭,悄悄對李嬌兒如此這般:「媽說,你摸量你手中沒甚細軟東西,不消只顧在他家了。你又沒兒女,守甚麼?教你一場嚷亂,登開了罷。昨日應二哥來說,如今大街坊張二官府,要破五百兩金銀,娶你做二房娘子,當家理紀。你那裡便圖出身,你在這裡守到老死,也不怎麼。你我院中人家,棄舊迎新為本,趨火附勢為強,不可錯過了時光。」這李嬌兒聽記在心,過了西門慶五七之後,因風吹火,用力不多。不想潘金蓮對孫雪娥說,出殯那日,在墳上看見李嬌兒與吳二舅在花園小房內,兩個說話來。春梅孝堂中又親眼看見李嬌兒帳子後遞了一包東西與李銘,塞在腰裡,轉了家去。嚷的月娘知道,把吳二舅罵了一頓,趕去鋪子里做買賣,再不許進後邊來。分付門上平安,不許李銘來往。這花娘惱羞變成怒,正尋不著這個由頭兒哩。一日因月娘在上房和大妗子吃茶,請孟玉樓,不請他,就惱了,與月娘兩個大鬧大嚷,拍著西門慶靈床子,啼啼哭哭,叫叫嚎嚎,到半夜三更,在房中要行上吊。丫頭來報與月娘。月娘慌了,與大妗子計議,請將李家虔婆來,要打發他歸院。虔婆生怕留下他衣服頭面,說了幾句言語:「我家人在你這裡做小伏低,頂缸受氣,好容易就開交了罷!須得幾十兩遮羞錢。」吳大舅居著官,又不敢張主,相講了半日,教月娘把他房中衣服、首飾、箱籠、床帳、家活盡與他,打發出門。只不與他元宵、繡春兩個丫頭去。李嬌兒生死要這兩個丫頭。月娘生死不與他,說道:「你倒好,買良為娼。」一句慌了鴇子,就不敢開言,變做笑吟吟臉兒,拜辭了月娘,李嬌兒坐轎子,抬的往家去了。

看官聽說,院中唱的,以賣俏為活計,將脂粉作生涯;早辰張風流,晚夕李浪子;前門進老子,後門接兒子;棄舊憐新,見錢眼開,自然之理。饒君千般貼戀,萬種牢籠,還鎖不住他心猿意馬。不是活時偷食抹嘴,就是死後嚷鬧離門。不拘幾時,還吃舊鍋粥去了。正是:
\begin{quote}
蛇入筒中曲性在,鳥出籠輕便飛騰。
\end{quote}
有詩為證:
\begin{quote}
堪笑煙花不久長,洞房夜夜換新郎。
兩隻玉腕千人枕,一點朱唇萬客嘗。
造就百般嬌艷態,生成一片假心腸。
饒君總有牢籠計,難保臨時思故鄉。
\end{quote}

月娘打發李嬌兒出門,大哭了一場。眾人都在旁解勸,潘金蓮道:「姐姐,罷,休煩惱了。常言道,娶淫婦,養海青,食水不到想海東。這個都是他當初乾的營生,今日教大姐姐這等惹氣。」

家中正亂著,忽有平安來報:「巡鹽蔡老爹來了,在廳上坐著哩,我說家老爹沒了。他問沒了幾時了,我回正月二十一日病故,到今過了五七。他問有靈沒靈,我回有靈,在後邊供養著哩。他要來靈前拜拜,我來對娘說。」月娘分付:「教你姐夫出去見他。」不一時,陳敬濟穿上孝衣出去,拜見了蔡御史。良久,後邊收拾停當,請蔡御史進來西門慶靈前參拜了。月娘穿著一身重孝,出來回禮,再不交一言,就讓月娘說:「夫人請回房。」又向敬濟說道:「我昔時曾在府相擾,今差滿回京去,敬來拜謝拜謝,不期作了故人。」便問:「甚麼病癥?」陳敬濟道:「是痰火之疾。」蔡御史道:「可傷,可傷。」即喚家人上來,取出兩匹杭州絹,一雙絨襪,四尾白鯗,四罐蜜餞,說道:「這些微禮,權作奠儀罷。」又拿出五十兩一封銀子來,「這個是我嚮日曾貸過老先生些厚惠,今積了些俸資奉償,以全終始之交。」分付平安道:「大官,交進房去。」敬濟道:「老爹忒多計較了。」月娘說:「請老爹前廳坐。」蔡御史道:「也不消坐了。拿茶來,吃了一鐘就是了。」左右須臾拿茶上來。蔡御史吃了,揚長起身上轎去了。月娘得了這五十兩銀子,心中又是那歡喜,又是那慘戚。想有他在時,似這樣官員來到,肯空放去了?又不知吃酒到多咱晚。今日他伸著腳子,空有家私,眼看著就無人陪待。正是:
\begin{quote}
人得交遊是風月,天開圖畫即江山。
\end{quote}

話說李嬌兒到家,應伯爵打聽得知,報與張二官知,就拿著五兩銀子來,請他歇了一夜。原來張二官小西門慶一歲,屬兔的,三十二歲了。李嬌兒三十四歲,虔婆瞞了六歲,只說二十八歲,教伯爵瞞著。使了三百兩銀子,娶到家中,做了二房娘子。祝實念、孫寡嘴依舊領著王三官兒,還來李家行走,與桂姐打熱,不在話下。

伯爵、李三、黃四借了徐內相五千兩銀子,張二官出了五千兩,做了東平府古器這批錢糧,逐日寶鞍大馬,在院內搖擺。張二官見西門慶死了,又打點了上千兩金銀,往東京尋了樞密院鄭皇親人情,對堂上朱太尉說,要討提刑所西門慶這個缺。家中收拾買花園,蓋房子。應伯爵無日不在他那邊趨奉,把西門慶家中大小之事,盡告訴與他,說:「他家中還有第五個娘子潘金蓮,排行六姐,生的上畫兒般標緻,詩詞歌賦,諸子百家,拆牌道字,雙陸象棋,無不通曉。又寫的一筆好字,彈的一手好琵琶。今年不上三十歲,比唱的還喬。」說的那張二官心中火動,巴不的就要了他,便問道:「莫非是當初賣炊餅的武大郎那老婆麼?」伯爵道:「就是他。占來家中,今也有五六年光景,不知他嫁人不嫁。」張二官道:「累你打聽著,待有嫁人的聲口,你來對我說,等我娶了罷。」伯爵道:「我身子里有個人,在他家做家人,名來爵兒。等我對他說,若有出嫁聲口,就來報你知道。難得你娶過他這個人來家,也強似娶個唱的。當時西門慶大官人在時,為娶他,不知費了許多心。大抵物各有主,也說不的,只好有福的匹配,你如有了這般勢耀,不得此女貌,同享榮華,枉自有許多富貴。我只叫來爵兒密密打聽,但有嫁人的風縫兒,憑我甜言美語,打動春心,你卻用幾百兩銀子,娶到家中,盡你受用便了。」看官聽說,但凡世上幫閑子弟,極是勢利小人。當初西門慶待應伯爵如膠似漆,賽過同胞弟兄,那一日不吃他的,穿他的,受用他的。身死未幾,骨肉尚熱,便做出許多不義之事。正是畫虎畫皮難畫骨,知人知面不知心。有詩為證:
\begin{quote}
昔年音氣似金蘭,百計趨奉不等閑。
自從西門身死後,紛紛謀妾伴人眠。
\end{quote}
