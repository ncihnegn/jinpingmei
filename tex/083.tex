
\chapter{秋菊含恨泄幽情 春梅寄柬諧佳會}

詩曰:
\begin{quote}
如此鐘情古所稀,吁嗟好事到頭非。
汪汪兩眼西風淚,猶向陽臺作雨飛。
月有陰晴與圓缺,人有悲歡與會別。
擁爐細語鬼神知,空把佳期為君說。
\end{quote}

話說潘金蓮見陳敬濟天明越牆過去了,心中又後悔。次日卻是七月十五日,吳月娘坐轎子往地藏庵薛姑子那裡,替西門慶燒盂蘭會箱庫去。金蓮眾人都送月娘到大門首。回來,孟玉樓、孫雪娥、大姐,都往後邊去了。獨金蓮落後,走到前廳儀門首,撞遇敬濟正在李瓶兒那邊樓上,尋瞭解當庫衣物抱出來。金蓮叫住,便向他說:「昨日我說了你幾句,你如何使性兒今早就跳出來了,莫不真個和我罷了?」敬濟道:「你老人家還說哩,一夜誰睡著來!險些兒一夜不曾把我麻煩死了,你看把我臉上肉也撾的去了!」婦人罵道:「賊短命,既不與他有首尾,賊人膽兒虛,你平白走怎的?」敬濟道:「天將明瞭,不走來,不教人看見了?誰與他有甚事來?」金蓮道:「既無此事,你今晚再來,我慢慢問你。」敬濟道:「吃你麻犯了人,一夜誰合眼兒來?等我白日里睡一覺兒去。」婦人道:「你不去,和你算帳。」說畢,婦人回房去了。

敬濟拿衣物往鋪子里來,做了一回買賣,歸到廂房,歪在床上睡了一覺。盼望天色晚了,要往金蓮那邊去。不想到黃昏時分,天色一陣黑陰來,窗外簌簌下起雨來。正是:
\begin{quote}
蕭蕭庭院黃昏雨,點點芭蕉不住聲。
\end{quote}

這敬濟見那雨下得緊,說道:「好個不做美的天!他甫能教我對證話去,今日不想又下起雨來,好悶倦人也。」於是長等短等,那雨不住,簌簌直下到初更時分,下的房檐上流水。這小郎君等不的雨住,披著一條茜紅毯子臥單在身上。那時吳月娘來家,大姐與元宵兒都在後邊沒出來。於是鎖了房門,從西角門大雨里走入花園,推了推角門。婦人知他今晚必來,早已分付春梅灌了秋菊幾鐘酒,同他在炕房裡先睡了,以此把角門虛掩。這敬濟推開角門,便挨身而入。進到婦人臥房,見紗房半啟,銀燭高燒,桌上酒果已陳,金尊滿泛。兩個並肩疊股而坐。婦人便問:「你既不曾與孟三兒勾搭,這簪子怎得到你手裡?」敬濟道:「本是我昨日在花園荼縻架下拾的,若哄你,便促死促灰。」婦人道:「既無此事,還把這簪子與你關頭,我不要你的。只要把我與你的簪子、香囊、帕兒物事收好著,少了我一件兒,錢與你答話。」兩個吃酒下棋,到一更方上床安寢。顛鸞倒鳳,整狂了半夜。婦人把昔日西門慶枕邊風月,一旦盡付與情郎身上。

卻說秋菊在那邊屋裡,忽聽見這邊屋裡恰似有男子聲音說話,更不知是那個。到天明雞叫時分,秋菊起來溺尿,忽聽那邊房內開的門響,朦朧月色,雨尚未止,打窗眼看見一人,披著紅臥單,從房中出去了。「恰似陳姐夫一般。原來夜夜和我娘睡。我娘自來會撇凈,乾凈暗裡養著女婿!」次日,徑走到後邊廚房裡,就如此這般對小玉說。不想小玉和春梅好,又告訴春梅說:「秋菊說你娘養著陳姐夫,昨日在房裡睡了一夜,今早出去了。大姑娘和元宵又沒在前邊睡。」這春梅歸房一五一十對婦人說:「娘不打與這奴才幾下,教他騙口張舌,葬送主子。」金蓮聽了大怒,就叫秋菊到面前跪著,罵道:「教你煎熬粥兒,就把鍋來打破了。你敢屁股大,吊了心也怎的?我這幾日沒曾打你這奴才,骨朵癢了!」於是拿棍子向他脊背上儘力狠抽了三十下,打得秋菊殺豬也似叫,身上都破了。春梅走將來說:「娘沒的打他這幾下兒,只好與他撾癢兒罷了。旋剝了,叫將小廝來,拿大板子儘力砍與他二三十板,看他怕不怕?湯他這幾下兒,打水不深的,只像鬥猴兒一般。他好小膽兒,你想他怕也怎的?做奴才,里言不出,外言不入,都似你這般,好養出家生哨兒來了。」秋菊道:「誰說甚麼來?」婦人道:「還說嘴哩!賊破家害主的奴才,還說甚麼!」幾聲喝的秋菊往廚下去了。正是:
\begin{quote}
蚊蟲遭扇打,只為嘴傷人。
\end{quote}

一日,八月中秋時分,金蓮夜間暗約敬濟賞月飲酒,和春梅同下鰲棋兒。晚夕貪睡失曉,至茶時前後還未起來,頗露圭角。不想被秋菊睃到眼裡,連忙走到後邊上房,對月娘說。不想月娘才梳頭,小玉正在上房門首站立。秋菊拉過他一邊,告他說:「俺姐夫如此這般,昨日又在我娘房裡歇了一夜,如今還未起來哩。前日為我告你說,打了我一頓。今日真實看見,我原不賴他,請奶奶快去瞧去。」小玉罵道:「張眼露睛奴才,又來葬送主子,俺奶奶梳頭哩,還不快走哩。」月娘便問:「他說甚麼?」小玉不能隱諱,只說:「五娘使秋菊來請奶奶說話。」更不說出別的事。

這月娘梳了頭,輕移蓮步,驀然來到前邊金蓮房門首。早被春梅看見,慌的先進來,報與金蓮。金蓮與敬濟兩個還在被窩內未起,聽見月娘到,兩個都吃了一驚,慌做手腳不迭,連忙藏敬濟在床身子里,用一床錦被遮蓋的沿沿的。教春梅放小桌兒在床上,拿過珠花來,且穿珠花。不一時,月娘到房中坐下,說:「六姐,你這咱還不見出門,只道你做甚,原來在屋裡穿珠花哩。」一面拿在手中觀看,誇道:「且是穿的好,正面芝麻花,兩邊槅子眼方勝兒,轅圍蜂趕菊,剛湊著同心結,且是好看。到明日,你也替我穿恁條箍兒戴。」婦人見月娘說好話兒,那心頭小鹿兒才不跳了,一面令春梅:、倒茶來與大娘吃。」少頃,月娘吃了茶,坐了回去了,說:「六姐快梳了頭,後邊坐。」金蓮道:「曉得。」打發月娘出來,連忙攛掇敬濟出港,往前邊去了。春梅與婦人整捏兩把汗,婦人說:「你大娘等閑無事再不來,今日大清早辰來做甚麼?」春梅道:「左右是咱家這奴才嚼舌來。」不一時,只見小玉走來,如此這般:「秋菊後邊說去,說姐夫在這屋裡明睡到夜,夜睡到明,被我罵喝了他兩聲,他還不動。俺奶奶問我,沒的說,只說五娘請奶奶說話,方纔來了。你老人家只放在心裡,大人不見小人之過,只堤防著這奴才就是了。」

看官聽說,雖是月娘不信秋菊說話,只恐金蓮少女嫩婦沒了漢子,日久一時心邪,著了道兒。恐傳出去,被外人唇舌。又以愛女之故,不教大姐遠出門,把李嬌兒廂房挪與大姐住,教他兩口兒搬進後邊儀門裡來。遇著傅伙計家去,方教敬濟輪番在鋪子里上宿。取衣物藥材,俱同玳安兒出入。各處門戶都上了鎖鑰,丫鬟婦女無事不許往外邊去。凡事都嚴緊,這潘金蓮與敬濟兩個熱突突恩情都間阻了。正是:
\begin{quote}
世間好事多間阻,就裡風光不久長。
\end{quote}
有詩為證:
\begin{quote}
幾向天台訪玉真,三山不見海沉沉。
侯門一日深如海,從此蕭郎是路人。
\end{quote}

潘金蓮自被秋菊泄露之後,與敬濟約一個多月不曾相會。金蓮每日難挨,怎禁繡幃孤冷,畫閣凄涼,未免害些木邊之目,田下之心。脂粉懶勻,茶飯頓減,帶圍寬褪,懨懨瘦損,每日只是思睡,扶頭不起。春梅道:「娘,你這等虛想也無用,昨日大娘留下兩個姑子,我聽見說今晚要宣捲,後邊關的儀門早。晚夕,我推往前邊馬房內取草裝枕頭,等我到鋪子里叫他去。我好歹叫了姐夫和娘會一面,娘心下如何?」婦人道:「我的好姐姐,你若肯可憐見,叫得他來,我恩有重報,決不有忘。」春梅道:「娘說的是那裡話!你和我是一個人,爹又沒了,你明日往前後進,我情願跟娘去。咱兩個還在一處。」婦人道:「你有此心,可知好哩。」

到於晚夕,婦人先在後邊月娘前,假託心中不自在,用了個金蟬脫殼,歸到前邊。月娘後邊儀門老早開了,丫鬟婦人都放出來,要聽尼僧宣捲。金蓮央及春梅,說道:「好姐姐,你快些請他去罷。」春梅道:「等我先把秋菊那奴才,與他幾鐘酒,灌醉了,倒扣他在廚房內。我方好去。」於是篩了兩大碗酒,打發秋菊吃了,扣他在廚房內,拿了個筐兒,走到前邊,先撮了一筐草,就悄悄到印子鋪門首,低聲叫門。正值傅伙計不在鋪中,往家去了。獨有敬濟在炕上才歪下,忽見有人叫門,聲音像是春梅,連忙開門,見是他,滿面笑道:「果然是小大姐,沒人,請裡面坐。」春梅走入房內,便問:「小廝們在那裡?」敬濟道:「玳安和平安,都在那邊生藥鋪中睡哩,獨我一個在此受孤凄,挨冷淡。」春梅道:「俺娘多上覆你,說你好人兒,這幾日就門邊兒也不往俺那屋裡走走去。說你另有了對門主顧兒了,不稀罕俺娘兒每了。」敬濟道:「說那裡話,自從那日著了唬,驚散了,又見大娘緊門緊戶,所以不敢走動。」春梅道:「俺娘為你這幾日心中好生不快,逐日無心無緒,茶飯懶吃,做事沒入腳處。今日大娘留他後邊聽宣捲,也沒去,就來了。一心只是牽掛想你,巴巴使我來,好歹教你快去哩。」敬濟道:「多感你娘稱們厚情,何以報答?你略先走一步兒,我收拾了,隨後就去。」一面開櫥門,取出一方白綾汗巾,一副銀三事挑牙兒與他。就和春梅兩個摟抱,按在炕上,且親嘴咂舌,不勝歡謔。正是:
\begin{quote}
無緣得會鶯鶯面,且把紅娘去解讒。
\end{quote}

兩個戲了一回,春梅先拿著草歸到房來,一五一十對婦人說:「姐夫我叫了,他便來也。見我去,好不喜歡,又與了我一方汗巾,一付銀挑牙兒。」婦人便叫春梅:「你在外邊看著,只怕他來。」

原來那日正值九月十二三,月色正明。陳敬濟旋到生藥鋪,叫過來安兒來這邊來。他只推月娘叫他聽宣捲,徑往後邊去了。因前邊花園門關了,打後邊角門走入金蓮那邊,搖木瑾花為號。春梅連忙接應,引入房中。婦人迎門接著,笑罵道:「賊短命,好人兒,就不進來走走兒。」敬濟道:「我巴不得要來哩,只怕弄出是非來,帶累你老人家,不好意思。」說著,二人攜手進房坐下。春梅關上角門,房中放桌兒,擺上酒餚。婦人和敬濟並肩疊股而坐,春梅打橫,把酒來斟,穿杯換盞,倚翠偎紅,吃了一回。吃的酒濃上來,婦人嬌眼乜斜,烏雲半軃,取出西門慶淫器包兒,裡面包著相思套、顫聲嬌、銀托子、勉鈴一弄兒淫器。教敬濟便在燈光影下,婦人便赤身露體,仰臥在一張醉翁椅兒上。敬濟亦脫的上下沒條絲,又拿出春意二十四解本兒,放在燈下,照著樣兒行事。婦人便叫春梅:「你在後邊推著你姐夫,只怕他身子乏了。」那春梅真個在後邊推送,敬濟那話插入婦人牝中,往來抽送,十分暢美,不可盡言。不想秋菊在後邊廚下,睡到半夜裡起來凈手,見房門倒扣著,推不開。於是伸手出來,撥開鳥弔兒,大月亮地里,躡足潛蹤,走到前房窗下。打窗眼裡望里張看,見房中掌著明晃晃燈燭,三個人吃得大醉,都光赤著身子,正做得好。兩個對面坐著,春梅便在身後推車,三人串作一處。但見:
\begin{quote}
一個不顧夫主名分,一個那管上下尊卑。
一個椅上逞雨意雲情,一個耳畔說山盟海誓。
一個寡婦房內翻為快活道場,一個丈母根前變作污淫世界。
一個把西門慶枕邊風月盡付與嬌婿,一個將韓壽偷香手段悉送與情娘。
\end{quote}
正是:
\begin{quote}
寫成今世不休書,結下來生歡喜帶。
\end{quote}

秋菊看到眼裡,口中不說,心內暗道:「他們還在人前撇清要打我,今日卻真實被我看見了。到明日對大娘說,莫非又說騙嘴張舌賴我不成!」於是瞧了個不亦樂乎,依舊還往廚房中睡去了。

三個整狂到三更時分才睡。春梅未曾天明先起來,走到廚房,見廚房門開了,便問秋菊。秋菊道:「你還說哩。我尿急了,往那裡溺?我拔開鳥弔,出來院子里溺尿來。」春梅道:「成精奴才,屋裡放著榪子,溺不是!」秋菊道:「我不知榪子在屋裡。」兩個後邊聒噪,敬濟天明起來,早往前邊去了。正是:
\begin{quote}
兩手劈開生死路,翻身跳出是非門。
\end{quote}

那婦人便問春梅:「後邊亂甚麼?」這春梅如此這般,告說秋菊夜裡開門一節。婦人發恨要打秋菊。這秋菊早辰又走來後邊,報與月娘知道,被月娘喝了一聲,罵道:「賊葬弄主子的奴才!前日平空走來,輕事重報,說他主子窩藏陳姐夫在房裡,明睡到夜,夜睡到明,叫了我去。他主子正在床上放炕桌兒穿珠花兒,那得陳姐夫來?落後陳姐夫打前邊來,恁一個弄主子的奴才!一個大人放在屋裡,端的是糖人兒,不拘那裡安放了?一個砂子那裡發落?莫不放在眼裡不成?傳出去,知道的是你這奴才葬送主子。不知道的,只說西門慶平日要的人強多了,人死了多少時兒,老婆們一個個都弄的七顛八倒。恰似我的這孩子,也有些甚根兒不正一般。」於是要打秋菊。唬得秋菊往前邊疾走如飛,再不敢來後邊說了。

婦人聽見月娘喝出秋菊,不信其事,心中越發放大膽了。西門大姐聽見此言,背地裡審問敬濟。敬濟道:「你信那汗邪了的奴才!我昨日見在鋪里上宿,幾時往花園那邊去來?花園門成日關著。」大姐罵道:「賊囚根子,你別要說嘴,你若有風吹草動,到我耳朵內,惹娘說我,你就信信脫脫去了,再也休想在這屋裡了。」敬濟道:「是非終日有,不聽自然無。大娘眼見不信他。」大姐道:「得你這般說就好了。」正是:
\begin{quote}
誰料郎心輕似絮,那知妾意亂如絲。
\end{quote}
