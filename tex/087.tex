
\chapter{王婆子貪財忘禍 武都頭殺嫂祭兄}

詩曰:
\begin{quote}
悠悠嗟我里,世亂各東西。
存者問消息,死者為塵泥。
賤子家既敗,壯士歸來時。
行久見空巷,日暮氣慘凄。
但逢狐與狸,豎毛怒裂眥。
我有鐲鏤劍,對此吐長霓。
\end{quote}

話說陳敬濟雇頭口起身,叫了張團練一個伴當跟隨,早上東京去不題。卻表吳月娘打發潘金蓮出門,次日使春鴻叫薛嫂兒來,要賣秋菊。這春鴻正走到大街,撞見應伯爵,叫住問:「春鴻,你往那裡去?」春鴻道:「大娘使小的叫媒人薛嫂兒去。」伯爵問:「叫媒人做甚麼?」春鴻道:「賣五娘房裡秋菊丫頭。」伯爵又問:「你五娘為甚麼打發出來嫁人?」這春鴻便如此這般,「因和俺姐夫有些說話,大娘知道了,先打發了春梅小大姐,然後打了俺姐夫一頓,趕出往家去了。昨日才打發出俺五娘來。」伯爵聽了,點了點頭兒,說道:「原來你五娘和你姐夫有楂兒,看不出人來。」又向春鴻說:「孩兒,你爹已是死了,你只顧還在他家做甚麼?終是沒出產。你心裡還要歸你南邊去?還是這裡尋個人家跟罷。」春鴻道:「便是這般說。老爹已是沒了,家中大娘好不嚴禁,各處買賣都收了,房子也賣了,琴童兒、畫童兒都走了,也攬不過這許多人口來。小的待回南邊去,又沒順便人帶去。這城內尋個人家跟,又沒個門路。」伯爵道:「傻孩兒,人無遠見,安身不牢。千山萬水,又往南邊去做甚?你肚裡會幾句唱,愁這城內尋不出主兒來答應。我如今舉保個門路與你。如今大街坊張二老爹家,有萬萬貫家財,見頂補了你爹在提刑院做掌刑千戶。如今你二娘又在他家做了二房,我把你送到他宅中答應,他見你會唱南曲,管情一箭就上垛,留下你做個親隨大官兒,又不比在你家裡。他性兒又好,年紀小小,又倜儻,又愛好,你就是個有造化的。」這春鴻扒倒地下就磕了個頭:「有累二爹。小的若見了張老爹,得一步之地,買禮與二爹磕頭。」伯爵一把手拉著春鴻說:「傻孩兒,你起來,我無有個不作成人的,肯要你謝?你那得錢兒來!」春鴻道:「小的去了,只怕家中大娘抓尋小的怎了?」伯爵道:「這個不打緊。我問你張二老爹討個貼兒,封一兩銀子與他家。他家銀子不敢受,不怕不把你不雙手兒送了去。」說畢,春鴻往薛嫂兒家,叫了薛嫂兒。見月娘,領秋菊出來,只賣了五兩銀子,交與月娘,不在話下。

卻說應伯爵領春鴻到張二官宅里見了。張二官見他生的清秀,又會唱南曲,就留下他答應。便拿拜貼兒,封了一兩銀子,送往西門慶家,討他箱子。那日吳月娘家中正陪雲離守娘子範氏吃酒。先是雲離守補在清河左衛做同知,見西門慶死了,吳月娘守寡,手裡有東西,就安心有垂涎圖謀之意。此日正買了八盤羹果禮物,來看月娘。見月娘生了孝哥,範氏房內亦有一女,方兩月兒,要與月娘結親。那日吃酒,遂兩家割衫襟,做了兒女親家,留下一雙金環為定禮。聽見玳安兒拿進張二官府貼兒,並一兩銀子,說春鴻投在他家答應去了,使人來討他箱子衣服。月娘見他見做提刑官,不好不與他,銀子也不曾收,只得把箱子與將出來。

初時,應伯爵對張二官說:「西門慶第五娘子潘金蓮生得標緻,會一手琵琶。百家詞曲,雙陸象棋,無不通曉,又會寫字。因為年小守不的,又和他大娘合氣,今打發出來,在王婆家嫁人。」這張二官一替兩替使家人拿銀子往王婆家相看,王婆只推他大娘子分付,不倒口要一百兩銀子。那人來回講了幾遍,還到八十兩上,王婆還不吐口兒。落後春鴻到他宅內,張二官聽見春鴻說,婦人在家養育女婿方打發出來。這張二官就不要了,對著伯爵說:「我家現放著十五歲未出幼兒子上學攻書,要這樣婦人來家做甚?」又聽見李嬌兒說,金蓮當初用毒藥擺佈死了漢子,被西門慶占將來家,又偷小廝,把第六個娘子娘兒兩個,生生吃他害殺了。以此張二官就不要了。

話分兩頭。卻說春梅賣到守備府中,守備見他生的標緻伶俐,舉止動人,心中大喜。與了他三間房住,手下使一個小丫鬟,就一連在他房中歇了三夜。三日,替他裁了兩套衣服。薛嫂兒去,賞了薛嫂五錢銀子。又買了個使女扶持他,立他做第二房。大娘子一目失明,吃長齋念佛,不管閑事。還有生姐兒孫二娘,在東廂居住。春梅在西廂房,各處鑰匙都教他掌管,甚是寵愛他。一日,聽薛嫂兒說,金蓮出來在王婆家聘嫁,這春梅晚夕啼啼哭哭對守備說:「俺娘兒兩個,在一處廝守這幾年,他大氣兒不著呵著我,把我當親女兒一般看承。只知拆散開了,不想今日他也出來了,你若肯娶將他來,俺娘兒每還在一處,過好日子。」又說他怎的好模樣兒,諸般詞曲都會,又會彈琵琶。聰明俊俏,百伶百俐。屬龍的,今才三十二歲兒。「他若來,奴情願做第三也罷。」於是把守備念轉了,使手下親隨張勝、李安封了二方手帕,二錢銀子,往王婆家相看,果然生的好個出色的婦人。王婆開口指稱他家大娘子要一百兩銀子。張勝、李安講了半日,還了八十兩,那王婆不肯,不轉口兒,要一百兩:「媒人錢不要便罷了,天也不使空人。」這張勝、李安只得又拿回銀子來稟守備。丟了兩日,怎禁這春梅晚夕啼啼哭哭:「好歹再添幾兩銀子,娶了來和奴做伴兒,死也甘心。」守備見春梅只是哭泣,只得又差了大管家周忠,同張勝《李安,氈包內拿著銀子,打開與婆子看,又添到九十兩上。婆子越發張致起來,說:「若九十兩,到不的如今,提刑張二老爹家抬的去了。」這周忠就惱了,分付李安把銀子包了,說道:「三隻腳蟾便沒處尋,兩腳老婆愁尋不出來!這老淫婦連人也不識。你說那張二官府怎的,俺府里老爹管不著你?不是新娶的小夫人再三在老爺跟前說念,要娶這婦人,平白出這些銀子,要他何用!」李安道:「勒掯俺兩番三次來回,賊老淫婦,越發鸚哥兒風了!」拉著周忠說:「管家,咱去來,到家回了老爺,好不好教牢子拿去,拶與他一頓好拶子。」這婆子終是貪著陳敬濟那口食,由他罵,只是不言語。二人到府中,回稟守備說:「已添到九十兩,還不肯。」守備說:「明日兌與他一百兩,拿轎子抬了來罷。」周忠說:「爺就與了一百兩,王婆還要五兩媒人錢。且丟他兩日,他若張致,拿到府中拶與他一頓拶子,他才怕。」看官聽說,大段金蓮生有地而死有處,不爭被周忠說這兩句話。有分交:這婦人從前作過事,今朝沒興一齊來。有詩為證:
\begin{quote}
人生雖未有前知,禍福因由更問誰。
善惡到頭終有報,只爭來早與來遲。
\end{quote}

按下一頭。單表武鬆自從墊發孟州牢城充軍之後,多虧小管營施恩看顧。次後,施恩與蔣門神爭奪快活林酒店,被蔣門神打傷,央武鬆出力,反打了蔣門神一頓。不想蔣門神妹子玉蘭,嫁與張都監為妾,賺武鬆去,假捏賊情,將武鬆拷打,轉又發安平寨充軍。這武鬆走到飛雲浦,又殺了兩個公人,復回身殺了張都監、蔣門神全家老小,逃躲在施恩家。施恩寫了一封書,皮箱內封了一百兩銀子,教武鬆到安平寨與知寨劉高,教看顧他。不想路上聽見太子立東宮,放郊天大赦,武鬆就遇赦回家,到清河縣下了文書,依舊在縣當差,還做都頭。來到家中,尋見上鄰姚一郎,交付迎兒。那時迎兒已長大十九歲了,收攬來家,一處居住。就有人告他說:「西門慶已死,你嫂子又出來了,如今還在王婆家,早晚嫁人。」這漢子扣了,舊仇在心。正是:
\begin{quote}
踏破鐵鞋無覓處,得來全不費工夫。
\end{quote}

次日,理幘穿衣,徑走過間壁王婆門首。金蓮正在簾下站著,見武鬆來,連忙閃入裡間去。武鬆掀開帘子便問:「王媽媽在家?」那婆子正在磨上掃面,連忙出來應道:「是誰叫老身?」見是武鬆,道了萬福。武鬆深深唱喏。婆子道:「武二哥,且喜,幾時回家來了?」武鬆道:「遇赦回家,昨日才到。一向多累媽媽看家,改日相謝。」婆子笑嘻嘻道:「武二哥比舊時保養,鬍子楂兒也有了,且是好身量,在外邊又學得這般知禮。」一面請他上坐,點茶吃了。武鬆道:「我有一樁事和媽媽說。」婆子道:「有甚事?武二哥只顧說。」武鬆道:「我聞的人說,西門慶已是死了,我嫂子出來,在你老人家這裡居住。敢煩媽媽對嫂子說,他若不嫁人便罷,若是嫁人,如是迎兒大了,娶得嫂子家去,看管迎兒,早晚招個女婿,一家一計過日子,庶不教人笑話。」婆子初時還不吐口兒,便道:「他在便在我這裡,倒不知嫁人不嫁人。」次後聽見說謝他,便道:「等我慢慢和他說。」

那婦人在簾內聽見武鬆言語,要娶他看管迎兒,又見武鬆在外出落得長大身材,胖了,比昔時又會說話兒,舊心不改,心下暗道:「我這段姻緣還落在他手裡。」就等不得王婆叫他,自己出來,向武鬆道了萬福,說道:「既是叔叔還要奴家去看管迎兒,招女婿成家,可知好哩。」王婆道:「我一件,只如今他家大娘子,要一百兩銀子才嫁人。」武鬆道:「如何要這許多?」王婆道:「西門大官人,當初為他使了許多,就打恁個銀人兒也勾了。」武鬆道:「不打緊,我既要請嫂嫂家去,就使一百兩也罷。另外破五兩銀子,與你老人家。」這婆子聽見,喜歡的屁滾尿流,沒口說道:「還是武二哥知禮,這幾年江湖上見的事多,真是好漢。」婦人聽了此言,走到屋裡,又濃濃點了一鐘瓜仁泡茶,雙手遞與武鬆吃了。婆子問道:「如今他家要發脫的緊,又有三四個官戶人家爭著娶,都回阻了,價錢不兌。你這銀子,作速些便好。常言先下米先吃飯,千里姻緣著線牽,休要落在別人手內。」婦人道:「既要娶奴家,叔叔上緊些。」武鬆便道:「明日就來兌銀子,晚夕請嫂嫂過去。」那王婆還不信武鬆有這些銀子,胡亂答應去了。

到次日,武鬆打開皮箱,拿出施恩與知寨劉高那一百兩銀子來,又另外包了五兩碎銀子,走到王婆家,拿天平兌起來。那婆子看見白晃晃擺了一桌銀子,口中不言,心內暗道:「雖是陳敬濟許下一百兩,上東京去取,不知幾時到來。仰著合著,我見鐘不打,去打鑄鐘?」又見五兩謝他,連忙收了。拜了又拜,說道:「還是武二哥知人甘苦。」武鬆道:「媽媽收了銀子,今日就請嫂嫂過門。」婆子道:「武二哥,且是好急性。門背後放花兒——你等不到晚了?也待我往他大娘那裡交了銀子,才打發他過去。」又道:「你今日帽兒光光,晚夕做個新郎。」那武鬆緊著心中不自在,那婆子不知好歹,又奚落他。打發武鬆出門,自己尋思:「他家大娘只叫我發脫,又沒和我斷定價錢,我今胡亂與他一二十兩銀子就是了,綁著鬼也落他一半多養家。」就把銀鑿下二十兩銀子,往月娘家裡交割明白。月娘問:「甚麼人家娶去了?」王婆道:「兔兒沿山跑,還來歸舊窩。嫁了他家小叔,還吃舊鍋里粥去了。」月娘聽了,暗中跌腳,常言「仇人見仇人,分外眼睛明」,與孟玉樓說:「往後死在他小叔子手裡罷了。那漢子殺人不斬眼,豈肯干休!」

不說月娘家中嘆息,卻表王婆交了銀子到家,下午時,教王潮先把婦人箱籠桌兒送過去。這武鬆在家中又早收拾停當,打下酒肉,安排下菜蔬。晚上婆子領婦人過門,換了孝,帶著新䯼髻,身穿紅衣服,搭著蓋頭。進門來,見明間內明亮亮點著燈燭,重立武大靈牌供養在上面,先有些疑忌,由不的發似人揪,肉如鉤搭。進入門來,到房中,武鬆分付迎兒把前門上了拴,後門也頂了。王婆見了,說道:「武二哥,我去罷,家裡沒人。」武鬆道:「媽媽請進房裡吃盞酒。」武鬆教迎兒拿菜蔬擺在桌上,須臾燙上酒來,請婦人和王婆吃酒。那武鬆也不讓,把酒斟上,一連吃了四五碗酒。婆子見他吃得惡,便道:「武二哥,老身酒勾了,放我去,你兩口兒自在吃罷。」武鬆道:「媽媽,且休得胡說!我武二有句話問你!」只聞颼的一聲響,向衣底掣出一把二尺長刃薄背厚的樸刀來,一隻手籠著刀靶,一隻手按住掩心,便睜圓怪眼,倒豎剛須,說道:「婆子休得吃驚!自古冤有頭,債有主,休推睡里夢裡。我哥哥性命都在你身上!」婆子道:「武二哥,夜晚了,酒醉拿刀弄杖,不是耍處。」武鬆道:「婆子休胡說,我武二就死也不怕!等我問了這淫婦,慢慢來問你這老豬狗!若動一動步兒,先吃我五七刀子。」一面回過臉來,看著婦人罵道:「你這淫婦聽著!我的哥哥怎生謀害了?從實說來,我便饒你。」那婦人道:「叔叔如何冷鍋中豆兒炮?好沒道理!你哥哥自害心疼病死了,乾我甚事?」說由未了,武鬆把刀子忔楂的插在桌子上,用左手揪住婦人雲髻,右手匹胸提住,把桌子一腳踢番,碟兒盞兒都打得粉碎。那婦人能有多大氣脈,被這漢子隔桌子輕輕提將起來,拖出外間靈桌子前。那婆子見勢頭不好,便去奔前門走,前門又上了栓。被武鬆大叉步趕上,揪番在地,用腰間纏帶解下來,四手四腳捆住,如猿猴獻果一般,便脫身不得,口中只叫:「都頭不消動意,大娘子自做出來,不乾我事。」武鬆道:「老豬狗,我都知道了,你賴那個?你教西門慶那廝墊發我充軍去,今日我怎生又回家了!西門慶那廝卻在那裡?你不說時,先剮了這個淫婦,後殺你這老豬狗!」提起刀來,便望那婦人臉上撇了兩撇。

婦人慌忙叫道:「叔叔且饒,放我起來,等我說便了。」武鬆一提,提起那婆娘,旋剝凈了,跪在靈桌子前。武鬆喝道:「淫婦快說!」那婦人唬得魂不附體,只得從實招說,將那時收帘子打了西門慶起,並做衣裳入馬通姦,後怎的踢傷武大心窩,王婆怎地教唆下毒,撥置燒化,又怎的娶到家去,一五一十,從頭至尾,說了一遍。王婆聽見,只是暗中叫苦,說:「傻才料,你實說了,卻教老身怎的支吾。」這武鬆一面就靈前一手揪著婦人,一手澆奠了酒,把紙錢點著,說道:「哥哥,你陰魂不遠,今日武鬆與你報仇雪恨。」那婦人見勢頭不好,才待大叫。被武鬆向爐內撾了一把香灰,塞在他口,就叫不出來了。然後劈腦揪番在地。那婦人掙扎,把䯼髻簪環都滾落了。武鬆恐怕他掙扎,先用油靴只顧踢他肋肢,後用兩隻手去攤開他胸脯,說時遲,那時快,把刀子去婦人白馥馥心窩內只一剜,剜了個血窟窿,那鮮血就冒出來。那婦人就星眸半閃,兩隻腳只顧登踏。武鬆口噙著刀子,雙手去斡開他胸脯,扎乞的一聲,把心肝五臟生扯下來,血瀝瀝供養在靈前。後方一刀割下頭來,血流滿地。迎兒小女在旁看見,唬的只掩了臉。武鬆這漢子端的好狠也。可憐這婦人,正是三寸氣在千般用,一日無常萬事休。亡年三十二歲。但見:
\begin{quote}
手到處青春喪命,刀落時紅粉亡身。七魄悠悠,已赴森羅殿上;三魂渺渺,應歸枉成城中。好似初春大雪壓折金錢柳,臘月狂風吹折玉梅花。這婦人嬌媚不知歸何處,芳魂今夜落誰家?
\end{quote}

古人有詩一首,單悼金蓮死的好苦也:
\begin{quote}
堪悼金蓮誠可憐,衣裳脫去跪靈前。
誰知武二持刀殺,只道西門綁腿頑。
往事看嗟一場夢,今身不值半文錢。
世間一命還一命,報應分明在眼前。
\end{quote}

武鬆殺了婦人,那婆子便叫:「殺人了!」武鬆聽見他叫,向前一刀,也割下頭來。拖過屍首。一邊將婦人心肝五臟,用刀插在後樓房檐下。

那時有初更時分,倒扣迎兒在屋裡。迎兒道:「叔叔,我害怕!」武鬆道:「孩兒,我顧不得你了。」武鬆跳過王婆家來,還要殺他兒子王潮。不想王潮合當不該死,聽見他娘這邊叫,就知武鬆行兇,推前門不開,叫後門也不應,慌的走去街上叫保甲。那兩鄰明知武鬆凶惡,誰敢向前。武鬆跳過牆來,到王婆房內,只見點著燈,房內一人也沒有。一面打開王婆箱籠,就把他衣服撇了一地。那一百兩銀子止交與吳月娘二十兩,還剩了八十五兩,並些釵環首飾,武鬆都包裹了。提了樸刀,越後牆,趕五更挨出城門,投十字坡張青夫婦那裡躲住,做了頭佗,上梁山為盜去了。正是:
\begin{quote}
平生不作縐眉事,世上應無切齒人。
\end{quote}
