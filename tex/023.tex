
\chapter{賭棋枰瓶兒輸鈔 覷藏春潘氏潛蹤}

詞曰:
\begin{quote}
心中難自泄,暗裡深深謝。未必娘行,恁地能賢哲。衷腸怎好和君說?
說不願丫頭,願做官人的侍妾。他堅牢望我情真切。豈想風波,果應了他心料者。
\end{quote}

話說一日臘盡春回,新正佳節,西門慶賀節不在家,吳月娘往吳大妗子家去了。午間孟玉樓、潘金蓮都在李瓶兒房裡下棋。玉樓道:「咱們今日賭甚麼好?」金蓮道:「咱們賭五錢銀子東道,三錢銀子買金華酒兒,那二錢買個豬頭來,教來旺媳婦子燒豬頭咱們吃。說他會燒的好豬頭,只用一根柴禾兒,燒的稀爛。」玉樓道:「大姐姐不在家,卻怎的計較?」存下一分兒,送在他屋裡,也是一般。」說畢,三人下棋。下了三盤,李瓶兒輸了五錢。金蓮使繡春兒叫將來興兒來,把銀子遞與他,教他買一壇金華酒,一個豬首,連四隻蹄子,吩咐:「送到後邊廚房裡,教來旺兒媳婦蕙蓮快燒了,拿到你三娘屋裡等著,我們就去。」玉樓道:「六姐,教他燒了拿盒子拿到這裡來吃罷。在後邊,李嬌兒、孫雪娥兩個看著,是請他不請他?」金蓮遂依玉樓之言。

不一時,來興兒買了酒和豬首,送到廚下。蕙蓮正在後邊和玉簫在石台基上坐著,撾瓜子耍子哩。來興兒便叫他:「蕙蓮嫂子,五娘、三娘都上覆你,使我買了酒、豬頭連蹄子,都在廚房裡,教你替他燒熟了,送到前邊六娘房裡去。」蕙蓮道:「我不得閑,與娘納鞋哩。隨問教那個燒燒兒罷,巴巴坐名兒教我燒?」來興兒道:「你燒不燒隨你,交與你,我有勾當去。」說著,出去了。玉簫道:「你且丟下,替他燒燒罷。你曉的五娘嘴頭子,又惹的聲聲氣氣的。」蕙蓮笑道:「五娘怎麼就知道我會燒豬頭,栽派與我!」於是起到大廚竈里,舀了一鍋水,把那豬首蹄子剃刷乾凈,只用的一根長柴禾安在竈內,用一大碗油醬,並茴香大料,拌的停當,上下錫古子扣定。那消一個時辰,把個豬頭燒的皮脫肉化,香噴噴五味俱全。將大冰盤盛了,連薑蒜碟兒,用方盒拿到前邊李瓶兒房裡,旋打開金華酒來。玉樓揀齊整的,留下一大盤子,並一壺金華酒,使丫頭送到上房裡,與月娘吃。其餘三人坐定,斟酒共酌。

正吃中間,只見蕙蓮笑嘻嘻走到跟前,說道:「娘們試嘗這豬頭,今日燒的好不好?」金蓮道:「三娘剛纔誇你倒好手段兒!燒的且是稀爛。」李瓶兒問道:「真個你只用一根柴禾兒?」蕙蓮道:「不瞞娘們說,還消不得一根柴禾兒哩!若是一根柴禾兒,就燒的脫了骨。」玉樓叫繡春:「你拿個大盞兒,篩一盞兒與你嫂子吃。」李瓶兒連忙叫繡春斟酒,他便取碟兒揀了一碟豬頭肉兒遞與蕙蓮,說道:「你自造的,你試嘗嘗。」蕙蓮道:「小的自知娘們吃不的咸,沒曾好生加醬,胡亂罷了。下次再燒時,小的知道了。」便磕了三個頭,方纔在桌頭旁邊立著,做一處吃酒。

到晚夕月娘來家,眾婦人見了月娘,小玉悉將送來豬頭,拿與月娘看。玉樓笑道:「今日俺們下棋耍子,贏的李大姐豬頭,留與姐姐吃。」月娘道:「這般有些不均了。各人賭勝,虧了一個就不是了。咱們這等計較:只當大節下,咱姊妹這幾人每人輪流治一席酒兒,叫將鬱大姐來,晚間耍耍,有何妨礙?強如賭勝負,難為一個人。我主張的好不好?」眾人都說:「姐姐主張的是!」月娘道:「明日初五日,就是我起先罷。」李嬌兒佔了初六,玉樓佔了初七,金蓮佔了初八。金蓮道:「只我便宜,那日又是我的壽酒,卻一舉而兩得。」問著孫雪娥,孫雪娥半日不言語。月娘道:「他罷,你們不要纏他了,教李大姐挨著罷。」玉樓道:「初九日又是六姐生日,只怕有潘姥姥和他妗子來。」月娘道:「初九日不得閑,教李大姐挪在初十罷了。」眾人計議已定。

話休絮煩。先是初五日,西門慶不在家,往鄰家赴席去了。月娘在上房擺酒,鬱大姐供唱,請眾姐妹歡飲了一日方散。到第二日,卻該李嬌兒,就挨著玉樓、金蓮,都不必細說。須臾,過了金蓮生日,潘姥姥、吳大妗子,都在這裡過節頑耍。看看到初十日,該李瓶兒擺酒,使繡春往後邊請雪娥去。一連請了兩替,答應著來,只顧不來。玉樓道:「我就說他不來,李大姐只顧強去請他。可是他對著人說的:『你每有錢的,都吃十輪酒兒,沒的俺們去赤腳絆驢蹄。』似他這等說,俺們罷了,把大姐姐都當驢蹄看承!」月娘道:「他是恁不成材的行貨子,都不消理他了,又請他怎的!」於是擺上酒來,眾人都來前邊李瓶兒房裡吃酒。鬱大姐在旁彈唱。當下,吳大妗子和西門大姐,共八個人飲酒。只因西門慶不在,月娘吩咐玉簫:「等你爹來家要吃酒,你打發他吃就是了。」玉簫應諾。

後晌時分,西門慶來家,玉簫替他脫了衣裳。西門慶便問:「娘往那去了?」玉簫回道:「都在六娘房裡和大妗子、潘姥姥吃酒哩。」西門慶問道:「吃的是甚麼酒?」玉簫道:「是金華酒。」西門慶道:「還有年下你應二爹送的那一壇茉莉花酒,打開吃。」一面教玉簫把茉莉花酒打開,西門慶嘗了嘗,說道:「正好你娘們吃。」教小玉、玉簫兩個提著,送到前邊李瓶兒房裡。蕙蓮正在月娘旁邊侍立斟酒,見玉簫送酒來,蕙蓮俐便,連忙走下來接酒。玉簫便遞了個眼色與他,向他手上捏了一把,這婆娘就知其意。月娘問玉簫:「誰使你送酒來?」玉簫道:「爹使我來。」月娘道:「你爹來家多大回了?」玉簫道:「爹剛纔來家。因問娘們吃酒,教我把這一壇茉莉花酒,拿來與娘們吃。」月娘問:「你爹若吃酒,房中放桌兒,有見成菜兒打發他吃。」玉簫應的,往後邊去了。

這蕙蓮在席上站了一回,推說道:「我後邊看茶來,與娘們吃。」月娘吩咐道:「對你姐說,上房揀妝里有六安茶,頓一壺來俺們吃。」這老婆一個獵古調走到後邊,玉簫站在堂屋門首,努了個嘴兒與他。老婆掀開帘子,進月娘房來,只見西門慶坐在椅子上吃酒。走向前,一屁股就坐在他懷裡,兩個就親嘴咂舌做一處。婆娘一面用手攥著他那話,一面在上噙酒哺與他吃。便道:「爹,你有香茶再與我些,前日與我的都沒了。我少薛嫂兒幾錢花兒錢,你有銀子與我些兒。」西門慶道:「我茄袋內還有一二兩,你拿去。」說著。西門慶要解他褲子。婦人道:「不好,只怕人來看見。」西門慶道:「你今日不出去,晚夕咱好生耍耍。」蕙蓮搖頭說道:「後邊惜薪司擋路兒——柴眾。咱不如還在五娘那裡,色絲子女。」於是玉簫在堂屋門首觀風,由他二人在屋裡做一處頑耍。

不防孫雪娥從後來,聽見房裡有人笑,只猜玉簫在房裡和西門慶說笑,不想玉簫又在穿廊下坐的,就立住了腳。玉簫恐怕他進屋裡去,便支他說:「前邊六娘請姑娘,怎的不去?」雪娥鼻子里冷笑道:「俺們是沒時運的人兒,騎著快馬也趕他不上,拿甚麼伴著他吃十輪酒兒?自己窮的伴當兒伴的沒褲兒!」正說著,被西門慶房中咳嗽了一聲,雪娥就往廚房裡去了。

這玉簫把帘子欣開,婆娘見無人,急伶俐兩三步就叉出來,往後邊看茶去。須臾,小玉從後邊走來叫:「蕙蓮嫂子,娘說你怎的取茶就不去了?」婦人道:「茶有了,著姐拿果仁兒來。」不一時,小玉拿著盞托,他提著茶,一直來到前邊。月娘問道:「怎的茶這咱才來?」蕙蓮道:「爹在房裡吃酒,小的不敢進去。等著姐屋裡取茶葉,剝果仁兒來。」眾人吃了茶,這蕙蓮在席上,斜靠桌兒站立,看著月娘眾人擲骰兒,故作揚聲說道:「娘,把長麼搭在純六,卻不是天地分?還贏了五娘。」又道:「你這六娘,骰子是錦屏風對兒。我看三娘這麼三配純五,只是十四點兒,輸了。」被玉簫惱了,說道:「你這媳婦子,俺們在這裡擲骰兒,插嘴插舌,有你甚麼說處?」把老婆羞的站又站不住,立又立不住,緋紅了麵皮,往下去了。正是:
\begin{quote}
誰人汲得西江水,難洗今朝一面羞。
\end{quote}

這裡眾婦人飲酒,至掌燈時分,只見西門慶掀帘子進來,笑道:「你們好吃!」吳大妗子跳起來,說道:「姐夫來了!」連忙讓座兒與他坐。月娘道:「你在後邊吃酒罷了,女婦男子漢,又走來做甚麼?」西門慶道:「既是恁說,我去罷。」於是走過金蓮這邊來,金蓮隨即跟了來。西門慶吃得半醉,拉著金蓮說道:「小油嘴,我有句話兒和你說。我要留蕙蓮在後邊一夜兒,後邊沒地方。看你怎的容他在你這邊歇一夜兒罷?」金蓮道:「我不好罵的,沒的那汗邪的胡亂!隨你和他那裡㒲搗去,好嬌態,教他在我這裡!我是沒處安放他。我就算依了你,春梅賊小肉兒他也不容。你不信,叫了春梅問他,他若肯了,我就容你。」西門慶道:「既是你娘兒們不肯,罷!我和他往山子洞兒那裡過一夜。你吩咐丫頭拿床鋪蓋,生些火兒。不然,這一冷怎麼當。」金蓮忍不住笑了:「我不好罵出你來的,賊奴才淫婦,他是養你的娘?你是王祥,寒冬臘月行孝順,在那石頭床上臥冰哩。」西門慶笑道:「怪小油嘴兒,休奚落我。罷麼,好歹叫丫頭生個火兒。」金蓮道:「你去,我知道。」當晚眾人席散,金蓮吩咐秋菊,果然抱鋪蓋、籠火,在山子底下藏春塢雪洞里。

蕙蓮送月娘、李嬌兒、玉樓進到後邊儀門首,故意說道:「娘,小的不送,往前邊去罷。」月娘道:「也罷,你前邊睡去罷。」這婆娘打發月娘進內,還在儀門首站立了一回,見無人,一溜煙往山子底下去了。正是:
\begin{quote}
莫教襄王勞望眼,巫山自送雨雲來。
\end{quote}

這宋蕙蓮走到花園門首,只說西門慶還未進來,就不曾扣門子,只虛掩著。來到藏春塢洞兒內,只見西門慶早在那裡秉燭而坐。婆娘進到裡面,但覺冷氣侵人,塵囂滿榻。於是袖中取出兩枝棒兒香,燈上點了,插在地下。雖故地下籠著一盆碳火兒,還冷的打兢。婆娘在床上先伸下鋪,上面還蓋著一件貂鼠禪衣。掩上雙扉,兩個上床就寢。西門慶脫去上衣白綾道袍,坐在床上,把婦人褪了褲,抱在懷裡,兩隻腳蹺在兩邊,那話突入牝中。兩個摟抱,正做得好。卻不防潘金蓮打聽他二人入港了,在房中摘去冠兒,輕移蓮步,悄悄走來竊聽。到角門首,推開門,遂潛身悄步而入。也不怕蒼苔冰透了凌波,花刺抓傷了裙褶,躡跡隱身,在藏春塢月窗下站聽。良久,只見裡面燈燭尚明,婆娘笑聲說:「冷鋪中舍冰,把你賊受罪不濟的老花子,就沒本事尋個地方兒,走在這寒冰地獄里來了!口裡銜著條繩子,凍死了往外拉。」又道:「冷合合的,睡了罷,怎的只顧端詳我的腳?你看過那小腳兒的來,象我沒雙鞋面兒,那個買與我雙鞋面兒也怎的?看著人家做鞋,不能彀做!」西門慶道:「我兒,不打緊,到明日替你買幾錢的各色鞋面。誰知你比你五娘腳兒還小!」婦人道:「拿甚麼比他!昨日我拿他的鞋略試了試,還套著我的鞋穿。倒也不在乎大小,只是鞋樣子周正才好。」金蓮在外聽了:「這個奴才淫婦!等我再聽一回,他還說甚麼。」又聽彀多時,只聽老婆問西門慶說:「你家第五的秋胡戲,你娶他來家多少時了?是女招的,是後婚兒來?」西門慶道:「也是回頭人兒。」婦人說:「嗔道恁久慣牢成!原來也是個意中人兒,露水夫妻。」這金蓮不聽便罷,聽了氣的在外兩隻胳膊都軟了,半日移腳不動,說道:「若教這奴才淫婦在裡面,把俺們都吃他撐下去了!」待要那時就聲張罵起來,又恐怕西門慶性子不好,逞了淫婦的臉。待要含忍了他,恐怕他明日不認。「罷罷!留下個記兒,使他知道,到明日我和他答話。」於是走到角門首,拔下頭上一根銀簪兒,把門倒銷了,懊恨歸房。晚景題過。

到次日清早晨,婆娘先起來,穿上衣裳,蓬著頭走出來。見角門沒插,吃了一驚,又搖門,搖了半日搖不開。走去見西門慶,西門慶隔壁叫迎春替他開了。因看見簪銷著門,知是金蓮的簪子,就知晚夕他聽了出去。這婦人懷著鬼胎,走到前邊,正開房門,只見平安從東凈里出來,看見他只是笑。蕙蓮道:「怪囚根子,誰和你呲那牙笑哩?」平安兒道:「嫂子,俺們笑笑兒也嗔?」蕙蓮道:「大清早晨,平白笑的是甚麼?」平安道:「我笑嫂子三日沒吃飯,眼前花。我猜你昨日一夜不來家!」婦人聽了此言,便把臉紅了,罵道:「賊提口拔舌見鬼的囚根子,我那一夜不在屋裡睡?怎的不來家?」平安道:「我剛纔還看見嫂子鎖著門,怎的賴得過?」蕙蓮道:「我早起身,就往五娘屋裡,只剛纔出來。你這囚在那裡來?」平安道:「我聽見五娘教你腌螃蟹,說你會劈的好腿兒。嗔道五娘使你門首看著賣簸箕的,說你會咂得好舌頭。」把婦人說的急了,拿起條門閂來,趕著平安兒繞院子罵道:「賊汗邪囚根子,看我到明日對他說不說。不與你個功德也不怕,狂的有些褶兒也怎的?」那平安道:「耶嚛,嫂子,將就著些兒罷。對誰說?我曉得你往高枝兒上去了。」那蕙蓮急起來,只趕著他打。不料玳安正在印子鋪走出來,一把手將閂奪住了,說道:「嫂子為甚麼打他?」蕙蓮道:「你問那呲牙囚根子,口裡白說六道的,把我的胳膊都氣軟了!」那平安得手往外跑了。玳安推著他說:「嫂子,你少生氣著惱,且往屋裡梳頭去罷。」婦人便向腰間荷包里,取出三四分銀子來,遞與玳安道:「累你替我拿大碗燙兩個合汁來我吃,把湯盛在銚子里罷。」玳安道:「不打緊,等我去。」一手接了。連忙洗了臉,替他燙了合汁來。婦人讓玳安吃了一碗,他也吃了一碗,方纔梳了頭,鎖上門,先到後邊月娘房裡打了卯兒,然後來金蓮房裡。

金蓮正臨鏡梳頭。蕙蓮小意兒,在旁拿抵鏡、掇洗手水,殷情侍奉。金蓮正眼也不瞧他。蕙蓮道:「娘的睡鞋裹腳,我捲平收了去?」金蓮道:「由他。你放著,叫丫頭進來收。」便叫秋菊:「賊奴才,往那去了?」蕙蓮道:「秋菊掃地哩。春梅姐在那裡梳頭哩。」金蓮道:「你別要管他,丟著罷,亦發等他們來收拾。歪蹄潑腳的,沒的沾污了嫂子的手。你去扶侍你爹,爹也得你恁個人兒扶侍他,才可他的心。俺們都是露水夫妻,再醮貨兒。只嫂子是正名正頂轎子娶將來的,是他的正頭老婆,秋胡戲。」這婦人聽了,正道著昨日晚夕他的真病,於是向前雙膝跪下,說道:「娘是小的一個主兒,娘不高抬貴手,小的一時兒存站不的。當初不因娘寬恩,小的也不肯依隨爹。就是後邊大娘,無過只是個大綱兒。小的還是娘抬舉多,莫不敢在娘面前欺心?隨娘查訪,小的但有一字欺心,到明日不逢好死,一個毛孔兒里生下一個疔瘡。」金蓮道:「不是這等說。我眼裡放不下砂子的人。漢子既要了你,俺們莫不與爭?不許你在漢子跟前弄鬼,輕言輕語的。你說你把俺們踩下去了,你要在中間踢跳,我的姐姐,對你說,把這樣心兒且吐了些兒罷!」蕙蓮道:「娘再訪,小的並不敢欺心,到只怕昨日晚夕娘錯聽了。」金蓮道:「傻嫂子,我閑的慌,聽你怎的?我對你說了罷,十個老婆買不住一個男子漢的心。你爹雖故家裡有這幾個老婆,或是外邊請人家的粉頭,來家通不瞞我一些兒,一五一十就告我說。你大娘當時和他一個鼻子眼兒里出氣,甚麼事兒來家不告訴我?你比他差些兒。」說得老婆閉口無言,在房中立了一回,走出來了。剛到儀門夾道內,撞見西門慶,說道:「你好人兒,原來昨日人對你說的話兒,你就告訴與人。今日教人下落了我恁一頓!我和你說的話兒,只放在你心裡,放爛了才好。為甚麼對人說?乾凈你這嘴頭子就是個走水的槽。有話到明日不告你說了。」西門慶道:「甚麼話?我並不知道。」那婦人瞅了一眼,往前邊去了。

這婦人嘴兒乖,常在門前站立,買東買西,趕著傅伙計叫傅大郎,陳敬濟叫姑夫,賁四叫老四。因和西門慶勾搭上了,越發在人前花哨起來,常和眾人打牙犯嘴,全無忌憚。或一時叫:「傅大郎,我拜你拜,替我門首看著賣粉的。」那傅伙計老成,便驚心兒替他門首看著,過來叫住,請他出來買。玳安故意戲他,說道:「嫂子,賣粉的早晨過去了,你早出來,拿秤稱他的好來!」婆娘罵道:「賊猴兒,裡邊五娘、六娘使我要買搽的粉,你如何說拿秤稱二斤胭脂三斤粉,教那淫婦搽了又搽?看我進裡邊對他說不說?」玳安道:「耶嚛,嫂子,行動只拿五娘嚇我!」一回又叫:「賁老四,我對你說,門首看著賣梅花菊花的,我要買兩對兒戴。」那賁四誤了買賣,好歹專心替他看著賣的叫住,請他出來買。婦人立在二層門裡,打門廂兒揀,要了他兩對髩花大翠,又是兩方紫綾閃色銷金汗巾兒,共該他七錢五分銀子。婦人向腰裡摸出半側銀子兒來,央及賁四替他鑿,稱七錢五分與他。那賁四正寫著帳,丟下走來替他錘。只見玳安來說道:「等我與嫂子鑿。」一面接過銀子在手,且不鑿,只顧瞧這銀子。婦人道:「賊猴兒,不鑿,只顧端詳甚麼?你半夜沒聽見狗咬?是偷來的銀子!」玳安道:「偷到不偷。這銀子到有些眼熟,倒象爹銀子包兒里的。前日爹在燈市裡,鑿與賣勾金蠻子的銀子,還剩了一半,就是這銀子。我記得千真萬真。」婦人道:「賊囚,一個天下,人還有一樣的,爹的銀子怎的到得我手裡?」玳安笑道:「我知道甚麼帳兒!」婦人便趕著打。玳安把銀子鑿下七錢五分,交與賣花翠的,把剩的銀子拿在手裡,不與他去了。婦人道:「賊囚根子!你敢拿了去,我算你好漢!」玳安道:「我不拿你的。你把剩下的,與我些兒買果子吃。」那婦人道:「賊猴兒,你遞過來,我與你。」哄和玳安遞到他手裡,只掠了四五分一塊與他,別的還塞在腰裡,一直進去了。

自此以後,常在門首成兩價拿銀錢買剪截花翠汗巾之類,甚至瓜子兒四五升里進去,分與各房丫鬟並眾人吃。頭上治的珠子箍兒,金燈籠墜子,黃烘烘的。衣服底下穿著紅𫄉綢褲兒,線捺護膝。又大袖子袖著香茶、香桶子三四個,帶在身邊。見一日也花消二三錢銀子,都是西門慶背地與他的,此事不必細說。這婦人自從金蓮識破他機關,每日只在金蓮房裡,把小意兒貼戀,與他頓茶頓水,做鞋腳針指,不拿強拿,不動強動。正經月娘後邊,每日只打個到面兒,就到金蓮這邊來。每日和金蓮、瓶兒兩個下棋、抹牌,行成夥兒。或一時撞見西門慶來,金蓮故意令他旁邊斟酒,教他一處坐了頑耍,只圖漢子喜歡。正是:
\begin{quote}
顛狂柳絮隨風舞,輕薄桃花逐水流。
\end{quote}
