
\chapter{潘金蓮不憤憶吹簫 西門慶新試白綾帶}

詞曰:
\begin{quote}
喚多情,憶多情,誰把多情喚我名?喚名人可憎。
為多情,轉多情,死向多情心不平。休教情重輕。
\end{quote}

話說應伯爵回家去了。西門慶就在藏春塢坐著,看泥水匠打地炕。牆外燒火,安放花草,庶不至煤煙熏觸。忽見平安拿進帖兒,稟說:「帥府周爺差人送分資來了。」盒內封著五封分資:周守備、荊都監、張團練、劉薛二內相,每人五星,粗帕二方,奉引賀敬。西門慶令左右收入後邊,拿回帖打發去了。

且說那日,楊姑娘與吳大妗子、潘姥姥坐轎子先來了,然後薛姑子、大師父、王姑子,並兩個小姑子妙趣、妙鳳,並鬱大姐,都買了盒兒來,與玉樓做生日。月娘在上房擺茶,眾姊妹都在一處陪侍。須臾吃了茶,各人取便坐了。

潘金蓮想著要與西門慶做白綾帶兒,即便走到房裡,拿過針線匣,揀一條白綾兒,將磁盒內顫聲嬌藥末兒裝在裡面,周圍用倒口針兒撩縫的甚是細法,預備晚夕要與西門慶雲雨之歡。不想薛姑子驀地進房來,送那安胎氣的衣胞符藥與他。這婦人連忙收過,一面陪他坐的。薛姑子見左右無人,便悄悄遞與他,說道:「你揀個壬子日空心服,到晚夕與官人在一處,管情一度就成胎氣。你看後邊大菩薩,也是貧僧替他安的胎,今已有了半肚子了。我還說個法兒與你:縫個錦香囊,我書道硃砂符兒安在裡面,帶在身邊,管情就是男胎,好不準驗。」這婦人聽了,滿心歡喜,一面接了符藥,藏放在箱內。拿過歷日來看,二十九日是壬子日。於是就稱了三錢銀子送與他,說:「這個不當什麼,拿到家買菜吃。等坐胎之時,我尋匹絹與你做衣穿。」薛姑子道:「菩薩快休計較,我不象王和尚那樣利心重。前者因過世那位菩薩念經,他說我攙了他的主顧,好不和我嚷鬧,到處拿言語喪我。我的爺,隨他墮業,我不與他爭執。我只替人家行好事,救人苦難。」婦人道:「薛爺,你只行你的事,各人心地不同。我這勾當,你也休和他說。」薛姑子道:「法不傳六耳,我肯和他說!去年為後邊大菩薩喜事,他還說我背地得多少錢,擗了一半與他才罷了。一個僧家,戒行也不知,利心又重,得了十方施主錢糧,不修功果,到明日死後,披毛戴角還不起。」說了回話,婦人教春梅:「看茶與薛爺吃。」那姑子吃了茶,又同他到李瓶兒那邊參了靈,方歸後邊來。

約後晌時分,月娘放桌兒炕屋裡,請眾堂客並三個姑子坐的。又在明間內放八仙桌兒,鋪著火盆擺下案酒,與孟玉樓上壽。不一時,瓊漿滿泛,玉斝高擎,孟玉樓打扮的粉妝玉琢,先與西門慶遞了酒,然後與眾姊妹敘禮,安席而坐。陳敬濟和大姐又與玉樓上壽,行畢禮,就在旁邊坐下。廚下壽麵點心添換,一齊拿上來。眾人才吃酒,只見來安拿進盒兒來說:「應保送人情來了。」西門慶叫月娘收了,就教來安:「送應二娘帖兒去,就請你應二爹和大舅來坐坐。我曉的他娘子兒,明日也是不來,請你二爹來坐坐罷,改日回人情與他就是了。」來安拿帖兒同應保去了。西門慶坐在上面,不覺想起去年玉樓上壽還有李大姐,今日妻妾五個,只少了他,由不得心中痛酸,眼中落淚。

不一時,李銘和兩個小優兒進來了。月娘吩咐:「你會唱『比翼成連理』不會?」韓佐道:「小的記得。」才待拿起樂器來彈唱,被西門慶叫近前,吩咐:「你唱一套『憶吹簫』我聽罷。」兩個小優連忙改調唱《集賢賓》「憶吹簫,玉人何處也。」唱了一回,唱到「他為我褪湘裙杜鵑花上血」,潘金蓮見唱此詞,就知西門慶念思李瓶兒之意。及唱到此句,在席上故意把手放在臉兒上,這點兒那點兒羞他,說道:「孩兒,那裡豬八戒走在冷鋪中坐著——你怎的醜的沒對兒!一個後婚老婆,又不是女兒,那裡討『杜鵑花上血』來?好個沒羞的行貨子!」西門慶道:「怪奴才,聽唱罷麼,我那裡曉得什麼。單管胡枝扯葉的。」只見兩個小優又唱到:「一個相府內懷春女,忽剌八拋去也。我怎肯恁隨邪,又去把牆花亂折!」那西門慶只顧低著頭留心細聽。須臾唱畢,這潘金蓮就不憤他,兩個在席上只顧拌嘴起來。月娘有些看不上,便道:「六姐,你也耐煩,兩個只顧強什麼?楊姑奶奶和他大妗子丟在屋裡,冷清清的,沒個人兒陪他,你每著兩個進去陪他坐坐兒,我就來。」當下金蓮和李嬌兒就往房裡去了。

不一時,只見來安來說:「應二娘帖兒送到了。二爹來了,大舅便來。」西門慶道:「你對過請溫師父來坐坐。」因對月娘說:「你吩咐廚下拿菜出來,我前邊陪他坐去。」又叫李銘:「你往前邊唱罷。」李銘即跟著西門慶出來,到西廂房內陪伯爵坐的。又謝他人情:「明日請令正好歹來走走。」伯爵道:「他怕不得來,家下沒人。」良久,溫秀才到,作揖坐下。伯爵舉手道:「早晨多有累老先生。」溫秀才道:「豈敢。」吳大舅也到了,相見讓位畢,一面琴童兒秉燭來,四人圍暖爐坐定。來安拿春盛案酒擺在桌上。伯爵燈下看見西門慶白綾襖子上,罩著青緞五彩飛魚蟒衣,張牙舞爪,頭角崢嶸,揚須鼓鬣,金碧掩映,蟠在身上,唬了一跳,問:「哥,這衣服是那裡的?」西門慶便立起身來,笑道:「你每瞧瞧,猜是那裡的?」伯爵道:「俺每如何猜得著。」西門慶道:「此是東京何太監送我的。我在他家吃酒,因害冷,他拿出這件衣服與我披。這是飛魚,因朝廷另賜了他蟒龍玉帶,他不穿這件,就送我了。此是一個大分上。」伯爵極口誇道:「這花衣服,少說也值幾個錢兒。此是哥的先兆,到明日高轉做到都督上,愁沒玉帶蟒衣?何況飛魚!只怕穿過界兒去哩!」說著,琴童安放鐘箸,拿酒上來。李銘在面前彈唱。伯爵道:「也該進去與三嫂遞杯酒兒才好,如何就吃酒?」西門慶道:「我兒,你既有孝順之心,往後邊與三嫂磕個頭兒就是了,說他怎的?」伯爵道:「磕頭到不打緊,只怕惹人議論我做大不尊,到不如你替我磕個兒罷。」被西門慶向他頭上打了一下,罵道:「你這狗才,單管恁沒大小!」伯爵道:「有大小到不教孩兒們打了。」兩個戲說了一回,琴童拿將壽麵來,西門慶讓他三人吃。自己因在後邊吃了,就遞與李銘吃。那李銘吃了,又上來彈唱。伯爵叫吳大舅:「吩咐曲兒叫他唱。」大舅道:「不要索落他,隨他揀熟的唱罷。」西門慶道:「大舅好聽《瓦盆兒》這一套。」一面令琴童斟上酒,李銘於是箏排雁柱,款定冰弦,唱了一套「叫人對景無言,終日減芳容」,下邊去了。只見來安上來稟說:「廚子家去,請問爹,明日叫幾名答應?」西門慶吩咐:「六名廚役、二名茶酒,酒筵共五桌,俱要齊備。」來安應諾去了。吳大舅便問:「姐夫明日請甚麼人?」西門慶悉把安郎中作東請蔡九知府說了。吳大舅道:「既明日大巡在姐夫這裡吃酒,又好了。」西門慶道:「怎的說?」吳大舅道:「還是我修倉的事,要在大巡手裡題本,望姐夫明日說說,教他青目青目,到年終考滿之時保舉一二,就是姐夫情分。」西門慶道:「這不打緊。大舅明日寫個履歷揭帖來,等我取便和他說。」大舅連忙下來打恭。伯爵道:「老舅,你老人家放心,你是個都根主子,不替你老人家說,再替誰說?管情消不得吹噓之力,一箭就上垛。」前邊吃酒到二更時分散了,西門慶打發李銘等出門,就吩咐:「明日俱早來伺候。」李銘等應諾去了。小廝收進傢伙,上房內擠著一屋裡人,聽見前邊散了,都往那房裡去了。

卻說金蓮,只說往他屋裡去,慌的往外走不迭。不想西門慶進儀門來了,他便藏在影壁邊黑影兒里,看著西門慶進入上房,悄悄走來窗下聽覷。只見玉簫站在堂屋門首,說道:「五娘怎的不進去?」又問:「姥姥怎的不見?」金蓮道:「老行貨子,他害身上疼,往房裡睡去了。」良久,只聽月娘問道:「你今日怎的叫恁兩個新小王八子?唱又不會唱,只一味『三弄梅花』。」玉樓道:「只你臨了教他唱『鴛鴦浦蓮開』,他才依了你唱。好兩個猾小王八子,不知叫什麼名字,一日在這裡只是頑。」西門慶道:「一個叫韓佐,一個叫邵謙。」月娘道:「誰曉的他叫什麼謙兒李兒!」不防金蓮躡足潛蹤進去,立在暖炕兒背後,忽說道:「你問他?正經姐姐吩咐的曲兒不叫他唱,平白鬍枝扯葉的教他唱什麼『憶吹簫』,支使的小王八子亂騰騰的,不知依那個的是。」玉樓「噦」了一聲,扭回頭看見是金蓮,便道:「這個六丫頭,你在那裡來?猛可說出話來,倒唬我一跳。單愛行鬼路兒。你從多咱走在我背後?」小玉道:「五娘在三娘背後,好少一回兒。」金蓮點著頭兒向西門慶道:「哥兒,你膿著些兒罷了。你那小見識兒,只說人不知道。他是甚『相府中懷春女』?他和我都是一般的後婚老婆。什麼他為你『褪湘裙杜鵑花上血』,三個官唱兩個喏,誰見來?孫小官兒問朱吉,別的都罷了,這個我不敢許。可是你對人說的,自從他死了,好應心的菜兒也沒一碟子兒。沒了王屠,連毛吃豬!你日逐只吃屎哩?俺們便不是上數的,可不著你那心罷了。一個大姐姐這般當家立紀,也扶持不過你來,可可兒只是他好。他死,你怎的不拉住他?當初沒他來時,你怎的過來?如今就是諸般兒稱不上你的心了。題起他來,就疼的你這心裡格地地的!拿別人當他,借汁兒下麵,也喜歡的你要不的。只他那屋裡水好吃麼?」月娘道:「好六姐,常言道:好人不長壽,禍害一千年。自古鏇的不圓砍的圓。你我本等是遲貨,應不上他的心,隨他說去罷了。」金蓮道:「不是咱不說他,他說出來的話灰人的心。只說人憤不過他。」那西門慶只是笑,罵道:「怪小淫婦兒,胡說了你,我在那裡說這個話來?」金蓮道:「還是請黃內官那日,你沒對著應二和溫蠻子說?怪不的你老婆都死絕了,就是當初有他在,也不怎麼的。到明日再扶一個起來,和他做對兒就是了。賊沒廉恥撒根基的貨!」說的西門慶急了,跳起來,趕著拿靴腳踢他,那婦人奪門一溜煙跑了。

這西門慶趕出去不見他,只見春梅站在上房門首,就一手搭伏春梅肩背往前邊來。月娘見他醉了,巴不的打發他前邊去睡,要聽三個姑子宣捲。於是教小玉打個燈籠,送他前邊去。金蓮和玉簫站在穿廊下黑影中,西門慶沒看見,逕走過去。玉簫向金蓮道:「我猜爹管情向娘屋裡去了。」金蓮道:「他醉了,快發訕,由他先睡,等我慢慢進去。」這玉簫便道:「娘,你等等,我取些果子兒捎與姥姥吃去。」於是走到床房內,拿些果子遞與婦人,婦人接的袖了,一直走到他前邊。只見小玉送了回來,說道:「五娘在那邊來?爹好不尋五娘。」

金蓮到房門首,不進去,悄悄向窗眼望里張覷,看見西門慶坐在床上,正摟著春梅做一處頑耍。恐怕攪擾他,連忙走到那邊屋裡,將果子交付秋菊。因問:「姥姥睡沒有?」秋菊道:「睡了一大回了。」金蓮囑咐他:「果子好生收在揀妝內。」又復往後邊來。只見月娘、李嬌兒、孟玉樓、西門大姐、大妗子、楊姑娘,並三個姑子帶兩個小姑子,坐了一屋裡人。薛姑子便盤膝坐在月娘炕上,當中放著一張炕桌兒,炷了香,眾人都圍著他,聽他說佛法。只見金蓮笑掀帘子進來,月娘道:「你惹下禍來,他往屋裡尋你去了。你不打發他睡,如何又來了?我還愁他到屋裡要打你。」金蓮笑道:「你問他敢打我不敢?」月娘道:「你頭裡話出來的忒緊了,他有酒的人,一時激得惱了,不打你打狗不成?俺每倒替你捏兩把汗,原來你到這等潑皮。」金蓮道:「他就惱,我也不怕他,看不上那三等兒九做的。正經姐姐吩咐的曲兒不教唱,且東溝犁西溝耙,唱他的心事。就是今日孟三姐的好日子,也不該唱這離別之詞。人也不知死到那裡去了,偏有那些佯慈悲假孝順,我是看不上。」大妗子道:「你姐妹每亂了這一回,我還不知因為什麼來。姑夫好好的進來坐著,怎的又出去了?」月娘道:「大妗子,你還不知道,那一個因想起李大姐來,說年時孟三姐生日還有他,今年就沒他,落了幾點眼淚,教小優兒唱了一套『憶吹簫,玉人兒何處也』。這一個就不憤他唱這詞,剛纔搶白了他爹幾句。搶白的那個急了,趕著踢打,這賊就走了。」楊姑娘道:「我的姐姐,你隨官人教他唱罷了,又搶白他怎的?想必每常見姐姐每都全全兒的,今日只不見了李家姐姐,漢子的心怎麼不慘切個兒。」孟玉樓道:「好奶奶,若是我每,誰嗔他唱!俺這六姐姐平昔曉的曲子里滋味,見那個誇死了的李大姐,比古人那個不如他,又怎的兩個相交情厚,又怎麼山盟海誓,你為我,我為你。這個牢成的又不服氣,只顧拿言語搶白他,整廝亂了這半日。」楊姑娘道:「我的姐姐,原來這等聰明!」月娘道:「他什麼曲兒不知道!但題起頭兒,就知尾兒。象我每叫唱老婆和小優兒來,只曉的唱出來就罷了。偏他又說那一段兒唱的不是了,那一句兒唱的差了,又那一節兒稍了。但是他爹說出個曲兒來,就和他白搽白亂,必須搽惱了才罷。」孟玉樓在旁邊戲道:「姑奶奶你不知,我三四胎兒只存了這個丫頭子,這般精靈古怪的。」金蓮笑向他打了一下,說道:「我到替你爭氣,你到沒規矩起來了。」楊姑娘道:「姐姐,你今後讓官人一句兒罷。常言:一夜夫妻百夜恩,相隨百步也有個徘徊之意。一個熱突突人兒,指頭兒似的少了一個,有個不想不疼不題念的?」金蓮道:「想怎不想,也有個常時兒。一般都是你的老婆,做什麼抬一個滅一個?只嗔俺們不替他戴孝,他又不是婆婆,胡亂戴過斷七罷了,只顧戴幾時?」楊姑娘道:「姐姐每見一半不見一半兒罷。」大妗子道:「好快!斷七過了,這一向又早百日來了。」楊姑娘問:「幾時是百日?」月娘道:「早哩,臘月二十六日。」王姑子道:「少不的念個經兒。」月娘道:「挨年近節,念什麼經!他爹只好過年念罷了。」說著,只見小玉拿上一道茶來,每人一盞。

須臾吃畢。月娘洗手,向爐中炷了香,聽薛姑子講說佛法。薛姑子就先宣念偈言,講了一段五戒禪師破戒戲紅蓮女子,轉世為東坡佛印的佛法。講說了良久方罷。只見玉樓房中蘭香,拿了兩方盒細巧素菜果碟、茶食點心來,收了香爐,擺在桌上。又是一壺茶,與眾人陪三個師父吃了。然後又拿葷下飯來,打開一壇麻姑酒,眾人圍爐吃酒。月娘便與大妗子擲色搶紅。金蓮便與李嬌兒猜枚,玉簫在旁邊斟酒,便替金蓮打桌底下轉子兒。須臾把李嬌兒贏了數杯。玉樓道:「等我和你猜,你只顧贏他罷。」卻要金蓮拿出手來,不許褪在袖子里,又不許玉簫近前。一連反贏了金蓮幾大鐘。

金蓮坐不住,去了。到前邊叫了半日,角門才開,只見秋菊揉眼。婦人罵道:「賊奴才,你睡來?」秋菊道:「我沒睡。」婦人道:「見睡起來,你哄我。你到自在,就不說往後來接我接兒去。」因問:「你爹睡了?」秋菊道:「爹睡了這一日了。」婦人走到炕房裡,摟起裙子來就在炕上烤火。婦人要茶吃,秋菊連忙傾了一盞茶來。婦人道:「賊奴才,好乾凈手兒,我不吃這陳茶,熬的怪泛湯氣。你叫春梅來,叫他另拿小銚兒頓些好甜水茶兒,多著些茶葉,頓的苦艷艷我吃。」秋菊道:「他在那邊床房裡睡哩,等我叫他來。」婦人道:「你休叫他,且教他睡罷。」這秋菊不依,走在那邊屋裡,見春梅歪在西門慶腳頭睡得正好。被他搖推醒了,道:「娘來了,要吃茶,你還不起來哩。」這春梅噦他一口,罵道:「見鬼的奴才,娘來了罷了,平白唬人剌剌的!」一面起來,慢條廝禮、撒腰拉褲走來見婦人,只顧倚著炕兒揉眼。婦人反罵秋菊:「恁奴才,你睡的甜甜兒的,把你叫醒了。」因叫他:「你頭上汗巾子跳上去了,還不往下扯扯哩。」又問:「你耳朵上墜子怎的只戴著一隻?」這春梅摸了摸,果然只有一隻。便點燈往那邊床上尋去,尋不見。良久,不想落在那腳踏板上,拾起來。婦人問:「在那裡來?」春梅道:「都是他失驚打怪叫我起來,吃帳鉤子抓下來了,才在踏板上拾起來。」婦人道:「我那等說著,他還只當叫起你來。」春梅道:「他說娘要茶吃來。」婦人道:「我要吃口茶兒,嫌他那手不乾凈。」這春梅連忙舀了一小銚子水,坐在火上,使他撾了些炭在火內,須臾就是茶湯。滌盞乾凈,濃濃的點上去,遞與婦人。婦人問春梅:「你爹睡下多大回了?」春梅道:「我打發睡了這一日了。問娘來,我說娘在後邊還未來哩。」

這婦人吃了茶,因問春梅:「我頭裡袖了幾個果子和蜜餞,是玉簫與你姥姥吃的,交付這奴才接進來,你收了?」春梅道:「我沒見,他知道放在那裡?」婦人叫秋菊,問他果子在那裡,秋菊道:「我放在揀妝內哩。」走去取來,婦人數了數兒,少了一個柑子,問他那裡去了。秋菊道:「我拿進來就放在揀妝內,那個害饞癆、爛了口吃他不成!」婦人道:「賊奴才,還漲漒嘴!你不偷,那去了?我親手數了交與你的,怎就少了一個?原來只孝順了你!」教春梅:「你與我把那奴才一邊臉上打與他十個嘴巴子。」春梅道:「那臢臉蛋子,倒沒的齷齪了我的手。」婦人道:「你與我拉過他來。」春梅用雙手推顙到婦人跟前。婦人用手擰著他腮頰,罵道:「賊奴才,這個柑子是你偷吃了不是?你實實說了,我就不打你。不然,取馬鞭子來,我這一旋剝就打個不數。我難道醉了?你偷吃了,一徑里鬼混我。」因問春梅:「我醉不醉?」那春梅道:「娘清省白醒,那討酒來?娘不信只掏他袖子,怕不的還有柑子皮兒在袖子里哩。」婦人於是扯過他袖子來,用手去掏,秋菊慌用手撇著不教掏。春梅一面拉起手來,果然掏出些柑子皮兒來。被婦人儘力臉上擰了兩把,打了兩下嘴巴,罵道:「賊奴才,你諸般兒不會,象這說舌偷嘴吃偏會。真贓實犯拿住,你還賴那個?我如今茶前酒後且不打你,到明日清省白醒,和你算帳。」春梅道:「娘到明日,休要與他行行忽忽的,好生旋剝了,叫個人把他實辣辣打與他幾十板子,叫他忍疼也懼怕些。甚麼逗猴兒似湯那幾棍兒,他才不放在心上!」那秋菊被婦人擰得臉脹腫的,谷都著嘴往廚下去了。婦人把那一個柑子平分兩半,又拿了個蘋婆石榴,遞與春梅,說道:「這個與你吃,把那個留與姥姥吃。」這春梅也不瞧,接過來似有如無,掠在抽屜內。婦人把蜜餞也要分開,春梅道:「娘不要分,我懶得吃這甜行貨子,留與姥姥吃罷。」以此婦人不分,都留下了。

婦人走到桶子上小解了,叫春梅掇進坐桶來,澡了牝,又問春梅:「這咱天有多時分了?」春梅道:「睡了這半日,也有三更了。」婦人摘了頭面,走來那邊床房裡,見桌上銀燈已殘,從新剔了剔,向床上看西門慶正打鼾睡。於是解松羅帶,卸褪湘裙,上床鑽入被窩裡,與西門慶並枕而臥。

睡下不多時,向他腰間摸他那話。弄了一回,白不起。原來西門慶與春梅才行房不久,那話綿軟,急切捏弄不起來。這婦人酒在腹中,欲情如火,蹲身在被底,把那話用口吮咂。挑弄蛙口,吞裹龜頭,只顧往來不絕。西門慶猛然醒了,便道:「怪小淫婦兒,如何這咱才來?」婦人道:「俺每在後邊吃酒,孟三兒又安排了兩大方盒酒菜,鬱大姐唱著,俺每猜枚擲骰兒,又頑了這一日,被我把李嬌兒贏醉了。落後孟三兒和我五子三猜,俺到輸了好幾鐘酒。你到是便宜,睡這一覺兒來好熬我,你看我依你不依?」西門慶道:「你整治那帶子有了?」婦人道:「在褥子底下不是?」一面探手取出來,與西門慶看了,替他扎在麈柄根下,系在腰間,拴的緊緊的。又問:「你吃了不曾?」西門慶道:「我吃了。」須臾,那話吃婦人一壁廂弄起來,只見奢棱跳腦,挺身直舒,比尋常更舒半寸有餘。婦人爬在身上,龜頭昂大,兩手扇著牝戶往裡放。須臾突入牝中,婦人兩手摟定西門慶脖項,令西門慶亦扳抱其腰,在上只顧揉搓,那話漸沒至根。婦人叫西門慶:「達達,你取我的柱腰子墊在你腰底下。」這西門慶便向床頭取過他大紅綾抹胸兒,四摺疊起墊著腰,婦人在他身上馬伏著,那消幾揉,那話盡入。婦人道:「達達,你把手摸摸,都全放進去了,撐的裡頭滿滿兒的。你自在不自在?」西門慶用手摸摸,見盡沒至根,間不容髮,止剩二卵在外,心中覺翕翕然暢美不可言。婦人道:「好急的慌,只是寒冷,咱不得拿燈兒照著乾,趕不上夏天好。」因問西門慶,說道:「這帶子比那銀托子好不好?又不格的陰門生痛的,又長出許多來。你不信,摸摸我小肚子,七八頂到奴心。」又道:「你摟著我,等我一發在你身上睡一覺。」西門慶道:「我的兒,你睡,達達摟著。」那婦人把舌頭放在他口裡含著,一面朦朧星眼,款抱香肩。睡不多時,怎禁那慾火燒身,芳心撩亂,於是兩手按著他肩膊,一舉一坐,抽徹至首,復送至根,叫:「親心肝,罷了,六兒的心了。」往來抽捲,又三百回。比及精泄,婦人口中只叫:「我的親達達,把腰扱緊了。」一面把奶頭教西門慶咂,不覺一陣昏迷,淫水溢下,婦人心頭小鹿突突的跳。登時四肢困軟,香雲撩亂。那話拽出來猶剛勁如故,婦人用帕搽之,說道:「我的達達,你不過卻怎麼的?」西門慶道:「等睡起一覺來再耍罷。」婦人道:「我的身子已軟癱熱化的。」當下雲收雨散,兩個並肩交股,相與枕籍於床上,不知東方之既白。正是:
\begin{quote}
等閑試把銀缸照,一對天生連理人。
\end{quote}
