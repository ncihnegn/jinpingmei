
\chapter{陳敬濟臨清逢舊識 韓愛姐翠館遇情郎}

詩曰:
\begin{quote}
教坊脂粉洗鉛華,一片閑心對落花。
舊曲聽來猶有恨,故園歸去已無家。
雲鬟半輓臨妝鏡,兩淚空流濕絳紗。
今日相逢白司馬,樽前重與訴琵琶。
\end{quote}

話說一日,周守備與濟南府知府張叔夜,領人馬剿梁山泊賊王宋江三十六人,萬餘草寇,都受了招安。地方平復,表奏朝廷,大喜。加升張叔夜為都御史、山東安撫大使、升備周秀為濟南兵馬制置,管理分巡河道,提察盜賊。部下從征有功人員,各升一級。軍門帶得敬濟名字,升為參謀之職,月給米二石,冠帶榮身。守備至十月中旬,領了敕書,率領人馬來家。先使人來報與春梅家中知道。春梅滿心歡喜,使陳敬濟與張勝、李安出城迎接。家中廳上排設酒筵,慶官賀喜。官員人等來拜賀送禮者不計其數。守備下馬,進入後堂,春梅、孫二娘接著。參賀已畢,陳敬濟就穿大紅員領,頭戴冠帽,腳穿皂靴,束著角帶,和新婦葛氏兩口兒拜見。守備見好個女子,賞了一套衣服、十兩銀子打頭面,不在話下。

晚夕,春梅和守備在房中飲酒,未免敘些家常事務。春梅道:「為娶我兄弟媳婦,又費許多東西。」守備道:「阿呀,你止這個兄弟,投奔你來,無個妻室,不成個前程道理。就是費了幾兩銀子,不曾為了別人。」春梅道:「你今又替他掙了這個前程,足以榮身勾了。」守備道:「朝廷旨意下來,不日我往濟南府到任。你在家看家,打點些本錢,教他搭個主管,做些大小買賣。三五日教他下去,查算帳目一遭,轉得些利錢來,也勾他攪計。」春梅道:「你說的也是。」兩個晚夕,夫妻同歡,不可細述。在家中住了十個日子,到十一月初旬時分,守備收拾起身。帶領張勝、李安,前去濟南到任,留周仁、周義看家。陳敬濟送到城南永福寺方回。

一日,春梅向敬濟商議:「守備教你如此這般,河下尋些買賣,搭個主管,覓得些利息,也勾家中費用。」這敬濟聽言,滿心歡喜。一日,正打街前走,尋覓主管伙計。也是合當有事,不料撞遇舊時朋友陸二哥陸秉義,作揖說:「哥怎的一向不見?」敬濟道:「我因亡妻為事,又被楊光彥那廝拐了我半船貨物,坑陷的我一貧如洗。我如今又好了,幸得我姐姐嫁在守備府中,又娶了親事,升做參謀,冠帶榮身。如今要尋個伙計作些買賣,一地裡沒尋處。」陸秉義道:「楊光彥那廝拐了你貨物,如今搭了個姓謝的做伙計,在臨清馬頭上開了一座大酒店,又放債與四方趁熟窠子娼門人使,好不獲大利息。他每日穿好衣,吃好肉,騎著一匹驢兒,三五日下去走一遭,算帳收錢,把舊朋友都不理。他兄弟在家開賭場,鬥雞養狗,人不敢惹他。」敬濟道:「我去年曾見他一遍,他反面無情,打我一頓,被一朋友救了。我恨他入於骨髓。」因拉陸二郎入路旁一酒店內吃酒。兩人計議:「如何處置他,出我這口氣?」陸秉義道:「常言說得好:恨小非君子,無毒不丈夫。咱如今將理和他說,不見棺材不下淚,他必然不肯。小弟有一計策,哥也不消做別的買賣,只寫一張狀子,把他告到那裡,追出你貨物銀子來。就奪了這座酒店,再添上些本錢,等我在馬頭上和謝三哥掌柜發賣。哥哥你三五日下去走一遭,查算帳目,管情見一月,你穩拍拍的有四十兩銀子利息,強如做別的生意。」看官聽說,當時只因這陸秉義說出這樁事,有分數,數個人死於非命。陳敬濟一種死,死之太苦;一種亡,亡之太屈。正是:
\begin{quote}
非乾前定數,半點不由人。
\end{quote}

敬濟聽了,道:「賢弟,你說的是。我到家就對我姐夫和姐姐說。這買賣成了,就安賢弟同謝三郎做主管。」當下兩個吃了回酒,各下樓來,還了酒錢。敬濟分付陸二哥:「兄弟,千萬謹言。」陸二郎道:「我知道。」各散回家。這敬濟就一五一十對春梅說:「爭奈他爺不在,如何理會?」有老家人周忠在旁,便道:「不要緊,等舅寫了一張狀子,該拐了多少銀子貨物,拿爺個拜貼兒,都封在裡面。等小的送與提刑所兩位官府案下,把這姓楊的拿去衙門中,一頓夾打追問,不怕那廝不拿出銀子來。」敬濟大喜,一面寫就一紙狀子,拿守備拜貼,彌封停當,就使老家人周忠送到提刑院。兩位官府正升廳問事,門上人稟道:「帥府周爺差人下書。」何千戶與張二官府喚周忠進見,問周爺上任之事,說了一遍。拆開封套觀看,見了拜貼、狀子。自恁要做分上,即便批行,差委緝捕番捉,往河下拿楊光彥去。回了個拜貼,付與周忠:「到家多上覆你爺、奶奶,待我這裡追出銀兩,伺候來領。」周忠拿回貼到府中,回覆了春梅說話:「即時準行拿人去了。待追出銀子,使人領去。」敬濟看見兩個折貼上面寫著:「侍生何永壽、張懋德頓首拜」。敬濟心中大喜。

遲不上兩日光景,提刑緝捕觀察番捉,往河下把楊光彥並兄弟楊二風都拿到衙門中。兩位官府,據著陳敬濟狀子審問。一頓夾打,監禁數日,追出三百五十兩銀子,一百桶生眼布。其餘酒店中家活,共算了五十兩,陳敬濟狀上告著九百兩,還差三百五十兩銀子。把房兒賣了五十兩,家產盡絕。這敬濟就把謝家大酒樓奪過來,和謝胖子合夥。春梅又打點出五百兩本錢,共湊了一千兩之數。委付陸秉義做主管,重新把酒樓裝修、油漆彩畫,闌干灼耀,棟宇光新,桌案鮮明,酒餚齊整。真個是:
\begin{quote}
啟瓮三家醉,開樽十里香。
神仙留玉佩,卿相解金貂。
\end{quote}

從正月半頭,陳敬濟在臨清馬頭上大酒樓開張,見一日也發賣三五十兩銀子。都是謝胖子和陸秉義眼同經手,在柜上掌柜。敬濟三五日騎頭口,伴當小薑兒跟隨,往河下算帳一遭。若來,陸秉義和謝胖子兩個伙計,在樓上收拾一間乾凈閣兒,鋪陳床帳,安放卓椅,糊的雪洞般齊整。擺設酒席,交四個好出色粉頭相陪。陳三兒那裡往來做量酒。

一日,三月佳節,春光明媚,景物芬芳,翠依依槐柳盈堤,紅馥馥杏桃燦錦。陳敬濟在樓上,搭伏定綠闌干,看那樓下景緻,好生熱鬧。有詩為證:
\begin{quote}
風拂煙籠錦繡妝,太平時節日初長。
能添壯士英雄膽,善解佳人愁悶腸。
三尺曉垂楊柳岸,一竿斜插杏花旁。
男兒未遂平生志,且樂高歌入醉鄉。
\end{quote}

一日,敬濟在樓窗後瞧看,正臨著河邊,泊著兩隻剝船。船上載著許多箱籠,卓凳家活,四五個人,盡搬入樓下空屋裡來。船上有兩個婦人,一個中年婦人,長挑身材,紫膛色;一個年小婦人,搽脂抹粉,生的白凈標緻,約有二十多歲。盡走入屋裡來。敬濟問謝主管:「是甚麼人?也不問一聲,擅自搬入我屋裡來。」謝主管道:「此兩個是東京來的婦人,投親不著,一時間無處尋房住,央此間鄰居範老來說,暫住兩三日便去。正欲報知官人,不想官人來問。」這敬濟正欲發怒,只見那年小婦人斂衽向前,望敬濟深深的道了個萬福,告說:「官人息怒,非干主管之事,是奴家大膽,一時出於無奈,不及先來宅上稟報,望乞恕罪。容略住得三五日,拜納房金,就便搬去。」這敬濟見小婦人會說話兒,只顧上上下下把眼看他。那婦人一雙星眼斜盼敬濟,兩情四目,不能定情。敬濟口中不言,心內暗想:「倒相那裡會過,這般眼熟。」那長挑身材中年婦人,也定睛看著敬濟,說道:「官人,你莫非是西門老爺家陳姑爺麼?」這敬濟吃了一驚,便道:「你怎的認得我?」那婦人道:「不瞞姑爺說,奴是舊伙計韓道國渾家,這個就是我女孩兒愛姐。」敬濟道:「你兩口兒在東京,如何來在這裡?你老公在那裡?」那婦人道:「在船上看家活。」敬濟急令量酒請來相見。

不一時,韓道國走來作揖,已是摻白須鬢,因說起:「韓中蔡太師、童太尉、李右相、朱太尉、高太尉、李太監六人,都被太學國子生陳東上本參劾,後被科道交章彈奏倒了。聖旨下來,拿送三法司問罪,發煙瘴地面,永遠充軍。太師兒子禮部尚書蔡攸處斬,家產抄沒入官。我等三口兒各自逃生,投到清河縣尋我兄弟第二的。不想第二的把房兒賣了,流落不知去向。三口兒雇船,從河道中來,不料撞遇姑夫在此,三生有幸。」因問:「姑夫今還在西門老爺家裡?」敬濟把頭項搖了一搖,說:「我也不在他家了。我在姐夫守備周爺府中,做了參謀官,冠帶榮身。近日合了兩個伙計,在此馬頭上開這個酒店,胡亂過日子。你每三口兒既遇著我,也不消搬去,便在此間住也不妨,請自穩便。」婦人與韓道國一齊下禮。說罷,就搬運船上家活箱籠上來。敬濟看得心癢,也使伴當小薑兒和陳三兒替他搬運了幾件家活。王六兒道:「不勞姑夫費心用力。」彼此俱各歡喜。敬濟道:「你我原是一家,何消計較?」敬濟見天色將晚,有申牌時分,要回家。分付主管:「咱蚤送些茶盒與他。」上馬,伴當跟隨來家,一夜心心念念,只是放韓愛姐不下。

過了一日,到第三日早起身,打扮衣服齊整,伴當小薑跟隨來河下大酒樓店中,看著做了回買賣。韓道國那邊使的八老來請吃茶。敬濟心下正要瞧去,恰好八老來請,便起身進去。只見韓愛姐見了,笑容可掬,接將出來,道了萬福:「官人請裡面坐。」敬濟到閣子內會下,王六兒和韓道國都來陪坐。少頃茶罷,彼此敘些舊時的閑話,敬濟不住把眼只睃那韓愛姐,愛姐一雙一雙涎澄澄秋波只看敬濟,彼此都有意了。有詩為證:
\begin{quote}
弓鞋窄窄剪春羅,香體酥胸玉一窩。
麗質不勝裊娜態,一腔幽恨蹙秋波。
\end{quote}

少頃,韓道國走出去了。愛姐因問:「官人青春多少?」敬濟道:「虛度二十六歲。」敬濟問:「姐姐青春幾何?」愛姐笑道:「奴與官人一緣一會,也是二十六歲。舊日又是大老爹府上相會過面,如何又幸遇在一處,正是有緣千里來相會。」那王六兒見他兩個說得入港,看見關目,推個故事,也走出去了。止有他兩人對坐。愛姐把些風月話兒來勾敬濟,敬濟自幼幹慣的道兒,怎不省得!便涎著臉兒,調戲答話。原來這韓愛姐從東京來,一路兒和他娘已做些道路。今見了敬濟,也是夙世有緣,三生一笑,不由的情投意合,見無人處,就走向前,挨在他身邊坐下,作嬌作痴,說道:「官人,你將頭上金簪子借我看一看。」敬濟正欲拔時,早被愛姐一手按住敬濟頭髻,一手拔下簪子來。便笑吟吟起身,說:「我和你去樓上說句話兒。」一頭說,一頭走。敬濟得不的這一聲,連忙跟上樓來。正是:
\begin{quote}
風來花自舞,春入鳥能言。
\end{quote}

敬濟跟他上樓,便道:「姐姐有甚話說?」愛姐道:「奴與你是宿世姻緣,今朝相遇,願偕枕席之歡,共效于飛之樂。」敬濟道:「難得姐姐見憐,只怕此間有人知覺。」韓愛姐做出許多妖嬈來,摟敬濟在懷,將尖尖玉手扯下他褲子來。兩個情興如火,按納不住,愛姐不免解衣仰臥,在床上交媾在一處。正是:
\begin{quote}
色膽如天怕甚事,鴛幃雲雨百年情。
\end{quote}

敬濟問:「你叫幾姐?」那韓愛姐道:「奴是端午所生,就叫五姐,又名愛姐。」霎時雲收雨散,偎倚共坐。韓愛姐將金簪子原插在他頭上,又告敬濟說:「自從三口兒東京來,投親不著,盤纏缺欠。你有銀子,見借與我父親五兩,奴按利納還,不可推阻。」敬濟應允,說:「不打緊,姐姐開口,就兌五兩來。」兩個又坐了半日,恐怕人談論,吃了一杯茶,愛姐留吃午飯,敬濟道:「我那邊有事,不吃飯了,少間就送盤纏來與你。」愛姐道:「午後奴略備一杯水酒,官人不要見卻,好歹來坐坐。」

敬濟在店內吃了午飯,又在街上閑散走了一回。撞見昔日晏公廟師兄金宗明作揖,把前事訴說了一遍。金宗明道:「不知賢弟在守備老爺府中認了親,在大樓開店,有失拜望。明日就使徒弟送茶來,閑中請去廟中坐一坐。」說罷,宗明歸去了。敬濟走到店中,陸主管道:「裡邊住的老韓請官人吃酒,沒處尋。」正說著,恰好八老又來請。就請二位主管相陪,再無他客。敬濟就同二主管,走到裡邊房內,蚤已安排酒席齊整。敬濟上坐,韓道國主位,陸秉義、謝胖子打橫,王六兒與愛姐旁邊僉坐,八老往來篩酒下菜。吃過數杯,兩個主管會意,說道:「官人慢坐,小人柜上看去。」起身去了。敬濟平昔酒量,不十分洪飲,又見主管去了,開懷與韓道國三口兒吃了數杯,便覺有些醉將上來。愛姐便問:「今日官人不回家去罷了?」敬濟道:「這咱晚了,回去不得,明日起身去罷。」王六兒、韓道國吃了一回,下樓去了。敬濟向袖中取出五兩銀子,遞與愛姐。愛姐到下邊交與王六兒,覆上來。兩個交杯換盞,倚翠偎紅,吃至天晚。愛姐卸下濃妝,留敬濟就在樓上閣兒里歇了。當下枕畔山盟,衾中海誓,鶯聲燕語,曲盡綢繆,不能悉記。愛姐在東京蔡太師府中,與翟管家做妾,曾扶持過老太太,也學會些彈唱,又能識字會寫,種種可人。敬濟歡喜不勝,就同六姐一般,正可在心上。以此與他盤桓一夜,停眠罷宿,免不的第二日起來得遲,約飯時才起來。王六兒安排些雞子肉圓子,做了個頭腦與他扶頭。兩個吃了幾杯暖酒。少頃主管來,請敬濟那邊擺飯。敬濟梳洗畢,吃了飯,又來辭愛姐,要回去。那愛姐不舍,只顧拋淚。敬濟道:「我到家三、五日,就來看你,你休煩惱。」說畢,伴當跟隨,騎馬往城中去了。一路上分付小薑兒:「到家休要說出韓家之事。」小薑兒道:「小的知道,不必分付。

敬濟到府中,只推店中買賣忙,算了帳目不覺天晚,歸來不得,歇了一夜。交割與春梅利息銀兩,見一遭兒也有三十兩銀子之數。回到家中,又被葛翠屏噪聒:「官人怎的外邊歇了一夜?想必在柳陌花街行踏,把我丟在家中,獨自空房,就不思想來家。」一連留住陳敬濟七八日,不放他往河下來。店中只使小薑兒,來問主管討算利息。主管一一封了銀子去。

韓道國免不得又交老婆王六兒又招惹別的熟人兒,或是商客來屋裡走動,吃茶吃酒。這韓道國先前嘗著這個甜頭,靠老婆衣飯肥家。況王六兒年紀雖老,風韻猶存,恰好又得他女兒來接代,也不斷絕這樣行業,如今索性大做了。當下見敬濟不來,量酒陳三兒替他勾了一個湖州販絲綿客人何官人來,請他女兒愛姐。那何官人年約五十餘歲,手中有千兩絲綿綢絹貨物,要請愛姐。愛姐一心想著敬濟,推心中不快,三回五次不肯下樓來,急的韓道國要不的。那何官人又見王六兒長挑身材,紫膛色,瓜子面皮,描的大大小鬢,涎鄧鄧一雙星眼,眼光如醉,抹的鮮紅嘴唇,料此婦人一定好風情,就留下一兩銀子,在屋裡吃酒,和王六兒歇了一夜。韓道國便躲避在外邊歇了,他女兒見做娘的留下客,只在樓上不下樓來,自此以後,那何官人被王六兒搬弄得快活,兩個打得一似火炭般熱,沒三兩日不來與他過夜。韓道國也禁過他許多錢使。

這韓愛姐見敬濟一去十數日不來,心中思想,挨一日似三秋,盼一夜如半夏,未免害木邊之目,田下之心。使八老往城中守備府中探聽。看見小薑兒,悄悄問他:「官人如何不去?」小薑兒說:「官人這兩日有些身子不快,不曾出門。」回來訴與愛姐。愛姐與王六兒商議,買了一副豬蹄,兩隻燒鴨,兩尾鮮魚,一盒酥餅,在樓上磨墨揮筆,寫封柬帖,使八老送到城中與敬濟去,叮嚀囑付:「你到城中,須索見陳官人親收,討回貼來。」八老懷內揣著柬帖,挑著禮物,一路無詞。來到城內守備府前,坐在沿街石台基上。只見伴當小薑兒出來,看見八老:「你又來做甚麼?」八老與他聲喏,拉在僻凈處說:「我特來見你官人,送禮來了。還有話說,我只在此等你。你可通報官人知道。」小薑隨即轉身進去。不多時,只見敬濟搖將出來。那時約五月,天氣暑熱。敬濟穿著紗衣服,頭戴著瓦楞帽,涼鞋凈襪。八老慌忙聲喏,說道:「官人貴體好些?韓愛姐使我稍一柬帖,送禮來了。」敬濟接了柬帖,說:「五姐好麼?」八老道:「五姐見官人一向不去,心中也不快在那裡。多上覆官人,幾時下去走走?」敬濟拆開柬帖觀看上面寫著甚言詞:
\begin{quote}
賤妾韓愛姐斂衽拜,謹啟情郎陳大官人臺下:自別尊顏,思慕之心未嘗少怠。向蒙期約,妾倚門凝望,不見降臨。昨遣八老探問起居,不遇而回。聞知貴恙欠安,令妾空懷悵望,坐臥悶懨,不能頓生兩翼而傍君之左右也。君在家,自有嬌妻美愛,又豈肯動念於妾,猶吐去之果核也。茲具腥味、茶盒數事,少伸問安誠意,幸希笑納。情照不宣。外具錦繡鴛鴦香囊一個,青絲一縷,少表寸心。仲夏念日賤妾愛姐再拜。
\end{quote}

敬濟看了柬帖並香囊。香囊裡面安放青絲一縷,香囊上扣著「寄與情郎陳君膝下」八字,依先折了,藏在袖中。府旁側首有個酒店,令小薑兒:「領八老同店內吃鐘酒,等我寫回帖與你。」小薑不敢怠慢,把四盒禮物收進去了。敬濟走到書院房內,悄悄寫了回柬,又包了五兩銀子,到酒店內問八老:「吃了酒不曾?」八老道:「多謝官人好酒,吃不得了,起身去罷。」敬濟將銀子並回柬付與八老,說:「到家多多拜上五姐,這五兩白金與他盤纏,過三兩日,我自去看他。」八老收了銀、柬,一直去了。敬濟回家,走入房中,葛翠屏便問:「是誰家送的禮物?」敬濟悉言:「店主人謝胖子,打聽我不快,送禮物來問安。」翠屏亦信其實。兩口兒計議,交丫鬟金錢兒拿盤子,拿了一隻燒鴨,一尾鮮血,半副蹄子,送到後邊與春梅吃,說是店主人家送的,也不查問。此事表過不題。

卻說八老到河下,天已晚了,入門將銀、柬都付與愛姐收了。拆開銀、柬,燈下觀看,上面寫道:
\begin{quote}
愛弟敬濟頓首字覆愛卿韓五姐妝次:向蒙會問,又承厚款,亦且雲情雨意,祚席鐘愛,無時少怠。所雲期望,正欲趨會,偶因賤軀不快,有失卿之盼望。又蒙遣人垂顧,兼惠可口佳餚,錦囊佳制,不勝感激!只在二三日間,容當面布。外具白金五兩,綾帕一方,少伸遠芹之敬,優乞心鑒,萬萬。敬濟再拜。
\end{quote}

愛姐看了,見帕上寫著四句詩曰:
\begin{quote}
吳綾帕兒織迴文,灑翰揮毫墨跡新。
寄與多情韓五姐,永諧鸞鳳百年情。
\end{quote}

看畢,愛姐把銀子付與王六兒。母子千歡萬喜,等候敬濟,不在話下。正是:
\begin{quote}
得意友來情不厭,知心人至話相投。
\end{quote}
有詩為證:
\begin{quote}
碧紗窗下啟箋封,一紙雲鴻香氣濃。
知你揮毫經玉手,相思都付不言中。
\end{quote}
