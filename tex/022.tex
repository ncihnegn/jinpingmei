
\chapter{蕙蓮兒偷期蒙愛 春梅姐正色閑邪}

詞曰:
\begin{quote}
今宵何夕?月痕初照。等閑間一見猶難,平白地兩邊湊巧。向燈前見他,向燈前見他,一似夢中來到。何曾心料,他怕人瞧。驚臉兒紅還白,熱心兒火樣燒。
\end{quote}

話說次日,有吳大妗子、楊姑娘、潘姥姥眾堂客,因來與孟玉樓做生日,月娘都留在後廳飲酒,其中惹出一件事兒。那來旺兒,因他媳婦癆病死了,月娘新又與他娶了一房媳婦,乃是賣棺材宋仁的女兒,也名喚金蓮。當先賣在蔡通判家房裡使喚,後因壞了事出來,嫁與廚役蔣聰為妻。這蔣聰常在西門慶家答應,來旺兒早晚到蔣聰家叫他去,看見這個老婆,兩個吃酒刮言,就把這個老婆刮上了。一日,不想這蔣聰因和一般廚役分財不均,酒醉廝打,動起刀杖來,把蔣聰戳死在地,那人便越牆逃走了。老婆央來旺兒對西門慶說了,替他拿帖兒縣裡和縣丞說,差人捉住正犯,問成死罪,抵了蔣聰命。後來,來旺兒哄月娘,只說是小人家媳婦兒,會做針指。月娘使了五兩銀子,兩套衣服,四匹青紅布,並簪環之類,娶與他為妻。月娘因他叫金蓮,不好稱呼,遂改名為蕙蓮。這個婦人小金蓮兩歲,今年二十四歲,生的白凈,身子兒不肥不瘦,模樣兒不短不長,比金蓮腳還小些兒。性明敏,善機變,會妝飾,就是嘲漢子的班頭,壞家風的領袖。若說他底的本事,他也曾:
\begin{quote}
斜倚門兒立,人來側目隨。
托腮並咬指,無故整衣裳。
坐立頻搖腿,無人曲唱低。
開窗推戶牖,停針不語時。
未言先欲笑,必定與人私。
\end{quote}

初來時,同眾媳婦上竈,還沒甚麼妝飾。後過了個月有餘,因看見玉樓、金蓮打扮,他便把鬏髻墊的高高的,頭髮梳的虛籠籠的,水髩描的長長的,在上邊遞茶遞水,被西門慶睃在眼裡。一日,設了條計策,教來旺兒押了五百兩銀子,往杭州替蔡太師製造慶賀生辰錦繡蟒衣,並家中穿的四季衣服,往回也有半年期程。從十一月半頭,搭在旱路車上起身去了。西門慶安心早晚要調戲他這老婆,不期到此正值孟玉樓生日,月娘和眾堂客在後廳吃酒。西門慶那日沒往那去,月娘分咐玉簫:「房中另放桌兒,打發酒菜你爹吃。」西門慶因打簾內看見蕙蓮身上穿著紅綢對襟襖、紫絹裙子,在席上斟酒,問玉簫道:「那個是新娶的來旺兒的媳婦子蕙蓮?怎的紅襖配著紫裙子,怪模怪樣?到明日對你娘說,另與他一條別的顏色裙子配著穿。」玉簫道:「這紫裙子,還是問我借的。」說著就罷了。

須臾,過了玉樓生日。一日,月娘往對門喬大戶家吃酒去了。約後晌時分,西門慶從外來家,已有酒了,走到儀門首,這蕙蓮正往外走,兩個撞個滿懷。西門慶便一手摟過脖子來,就親了個嘴,口中喃喃吶吶說道:「我的兒,你若依了我,頭面衣服,隨你揀著用。」那婦人一聲兒沒言語,推開西門慶手,一直往前走了。西門慶歸到上房,叫玉簫送了一匹藍緞子到他屋裡,如此這般對他說:「爹昨日見你穿著紅襖,配著紫裙子,怪模怪樣的不好看,才拿了這匹緞子,使我送與你,教你做裙子穿。」這蕙蓮開看,卻是一匹翠藍兼四季團花喜相逢緞子。說道:「我做出來,娘見了問怎了?」玉簫道:「爹到明日還對娘說,你放心。爹說來,你若依了這件事,隨你要甚麼,爹與你買。今日趕娘不在家,要和你會會兒,你心下如何?」那婦人聽了,微笑不言,因問:「爹多咱時分來?我好在屋裡伺候。」玉簫道:「爹說小廝們看著,不好進你屋裡來的。教你悄悄往山子底下洞兒里,那裡無人,堪可一會。」老婆道:「只怕五娘、六娘知道了,不好意思的。」玉簫道:「三娘和五娘都在六娘屋裡下棋,你去不妨事。」當下約會已定,玉簫走來回西門慶說話。兩個都往山子底下成事,玉簫在門首與他觀風。正是:
\begin{quote}
解帶色已戰,觸手心愈忙。
那識羅裙內,銷魂別有香。
\end{quote}

不想金蓮、玉樓都在李瓶兒房裡下棋,只見小鸞來請玉樓,說:「爹來家了。」三人就散了,玉樓回後邊去了。金蓮走到房中,勻了臉,亦往後邊來。走入儀門,只見小玉立在上房門首。金蓮問:「你爹在屋裡?」小玉搖手兒,往前指。金蓮就知其意,走到前邊山子角門首,只見玉簫攔著門。金蓮只猜玉簫和西門慶在此私狎,便頂進去。玉簫慌了,說道:「五娘休進去,爹在裡頭有勾當哩!」金蓮罵道:「怪狗肉,我又怕你爹了?」不由分說,進入花園裡來,各處尋了一遍。走到藏春塢山子洞兒里,只見他兩個人在裡面才了事。婦人聽見有人來,連忙繫上裙子往外走,看見金蓮,把臉通紅了。金蓮問道:「賊臭肉,你在這裡做甚麼?」蕙蓮道:「我來叫畫童兒。」說著,一溜煙走了。金蓮進來,看見西門慶在裡邊系褲子,罵道:「賊沒廉恥的貨,你和奴才淫婦大白日里在這裡,端的乾這勾當兒,剛纔我打與淫婦兩個耳刮子才好,不想他往外走了。原來你就是畫童兒,他來尋你!你與我實說,和這淫婦偷了幾遭?若不實說,等住回大姐姐來家,看我說不說。我若不把奴才淫婦臉打的脹豬,也不算。俺們閑的聲喚在這裡,你也來插上一把子。老娘眼裡卻放不過!」西門慶笑道:「怪小淫婦兒,悄悄兒罷,休要嚷的人知道。我實對你說,如此這般,連今日才第一遭。」金蓮道:「一遭二遭,我不信。你既要這奴才淫婦,兩個瞞神謊鬼弄刺子兒,我打聽出來,休怪了,我卻和你們答話!」那西門慶笑的出去了。

金蓮到後邊,聽見眾丫頭們說:「爹來家,使玉簫手巾裹著一匹藍緞子往前邊去,不知與誰。」金蓮就知是與蕙蓮的,對玉樓也不題起此事。這婦人每日在那邊,或替他造湯飯,或替他做針指鞋腳,或跟著李瓶兒下棋,常賊乖趨附金蓮。被西門慶撞在一處,無人,教他兩個苟合,圖漢子喜歡。蕙蓮自從和西門慶私通之後,背地與他衣服、首飾、香茶之類不算,只銀子成兩家帶在身邊,在門首買花翠胭脂,漸漸顯露,打扮的比往日不同。西門慶又對月娘說,他做的好湯水,不教他上大竈,只教他和玉簫兩個,在月娘房裡後邊小竈上,專頓茶水,整理菜蔬,打發月娘房裡吃飯,與月娘做針指,不必細說。看官聽說:凡家主,切不可與奴僕並家人之婦苟且私狎,久後必紊亂上下,竊弄姦欺,敗壞風俗,殆不可制。

一日,臘月初八日,西門慶早起,約下應伯爵,與大街坊尚推官家送殯。叫小廝馬也備下兩匹,等伯爵白不見到,一面李銘來了。西門慶就在大廳上圍爐坐的,教春梅、玉簫、蘭香、迎春一般兒四個,都打扮出來,看著李銘指撥、教演他彈唱。女婿陳敬濟,在旁陪著說話。正唱《三弄梅花》,還未了,只見伯爵來,應保夾著氈包進門。那春梅等四個就要往後走,被西門慶喝住,說道:「左右只是你應二爹,都來見見罷,躲怎的!」與伯爵兩個相見作揖,才待坐下,西門慶令四個過來:「與應二爹磕頭。」那春梅等朝上磕頭下去,慌的伯爵還喏不迭,誇道:「誰似哥有福,出落的恁四個好姐姐,水蔥兒的一般,一個賽一個。卻怎生好?你應二爹今日素手,促忙促急,沒曾帶的甚麼在身邊,改日送胭脂錢來罷。」春梅等四人,見了禮去了。陳敬濟向前作揖,一同坐下。西門慶道:「你如何今日這咱才來?」應伯爵道:「不好告訴你的。大小女病了一向,近日才好些。房下記掛著,今日接了他家來散心住兩日。亂著,旋叫應保叫了轎子,買了些東西在家,我才來了。」西門慶道:「教我只顧等著你。咱吃了粥,好去了。」隨即吩咐後邊看粥來吃。只見李銘,見伯爵打了半跪。伯爵道:「李日新,一向不見你。」李銘道:「小的有。連日小的在北邊徐公公那裡答應來。」說著,小廝放桌兒,拿粥來吃。西門慶陪應伯爵、陳敬濟吃了。就拿小銀鐘篩金華酒,每人吃了三杯。壺裡還剩下上半壺酒,吩咐畫童兒:「連桌兒抬去廂房內,與李銘吃。」就穿衣服起身,同伯爵並馬而行,與尚推官送殯去了。只落下李銘在西廂房,吃畢酒飯。

玉簫和蘭香眾人,打發西門慶出了門,在廂房內廝亂,頑成一塊。一回,都往對過東廂房西門大姐房裡摑混去了,止落下春梅一個,和李銘在這邊教演琵琶。李銘也有酒了。春梅袖口子寬,把手兜住了。李銘把他手拿起,略按重了些。被春梅怪叫起來,罵道:「好賊忘八!你怎的捻我的手,調戲我?賊少死的忘八,你還不知道我是誰哩!一日好酒好肉,越發養活的你這忘八聖靈兒出來了,平白捻我的手來了。賊忘八,你錯下這個鍬撅了。你問聲兒去,我手裡你來弄鬼!爹來家等我說了,把你這賊忘八,一條棍攆的離門離戶!沒你這忘八,學不成唱了?愁本司三院尋不出忘八來?撅臭了你這忘八了!」被他千忘八,萬忘八,罵的李銘拿著衣服,往外走不迭。正是:
\begin{quote}
兩手劈開生死路,翻身跳出是非門。
\end{quote}

當下春梅氣狠狠,直罵進後邊來。金蓮正和孟玉樓、李瓶兒並宋蕙蓮在房裡下棋,只聽見春梅從外罵將來。金蓮便問道:「賊小肉兒,你罵誰哩,誰惹你來?」春梅道:「情知是誰,叵耐李銘那忘八!爹臨去,好意吩咐小廝,留下一桌菜並粳米粥兒與他吃。也有玉簫他們,你推我,我打你,頑成一塊,對著忘八,呲牙露嘴的,狂的有些褶兒也怎的。頑了一回,都往大姐那邊去了。忘八見無人,儘力把我手上捻一下。吃的醉醉的,看著我嗤嗤呆笑。那忘八見我吆喝罵起來,他就夾著衣裳往外走了。剛纔打與賊忘八兩個耳刮子才好!賊忘八,你也看個人兒行事,我不是那不三不四的邪皮行貨,教你這個忘八在我手裡弄鬼。我把忘八臉打綠了!」金蓮道:「怪小肉兒,學不學沒要緊,把臉氣的黃黃的,等爹來家說了,把賊忘八攆了去就是了。那裡緊等著供唱撰錢哩,怎的教忘八調戲我這丫頭!我知道賊忘八業罐子滿了。」春梅道:「他就倒運,著量二娘的兄弟。那怕他!二娘莫不挾仇打我五棍兒?」宋蕙蓮道:「論起來,你是樂工,在人家教唱,也不該調戲良人家女子!照顧你一個錢,也是養身父母,休說一日三茶六飯兒扶侍著。」金蓮道:「扶侍著,臨了還要錢兒去了。按月兒,一個月與他五兩銀子。賊忘八,錯上了墳。你問聲家裡這些小廝們,那個敢望著他呲牙笑一笑兒,吊個嘴兒?遇喜歡罵兩句;若不歡喜,拉倒他主子跟前就是打。賊忘八,造化低,你惹他生薑,你還沒曾經著他辣手!」因向春梅道:「沒見你,你爹去了,你進來便罷了,平白只顧和他那房裡做甚麼?卻教那忘八調戲你!」春梅道:「都是玉簫和他們,只顧還笑成一塊,不肯進來。」玉樓道:「他三個如今還在那屋裡?」春梅道:「都往大姐房裡去了。」玉樓道:「等我瞧瞧去。」那玉樓起身去了。良久,李瓶兒亦回房,使繡春叫迎春去。至晚,西門慶來家,金蓮一五一十告訴西門慶。西門慶吩咐來興兒,今後休放進李銘來走動。自此斷了路兒,不敢上門。正是:
\begin{quote}
習教歌妓逞家豪,每日閑庭弄錦槽。
不是朱顏容易變,何由聲價競天高。
\end{quote}
