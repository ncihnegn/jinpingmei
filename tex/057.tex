
\chapter{開緣簿千金喜舍 戲雕欄一笑回嗔}

詩曰:
\begin{quote}
野寺根石壁,諸龕遍崔巍。
前佛不復辨,百身一莓苔。
惟有古殿存,世尊亦塵埃。
如聞龍象泣,足令信者哀。
公為領兵徒,咄嗟檀施開。
吾知多羅樹,卻倚蓮花台。
諸天必歡喜,鬼物無嫌猜。
\end{quote}

話說那山東東平府地方,向來有個永福禪寺,起建自梁武帝普通二年,開山是那萬回老祖。怎麼叫做萬回老祖?因那老祖做孩子的時節,才七八歲,有個哥兒從軍邊上,音信不通,不知生死。他老娘思想大的孩兒,時常在家啼哭。忽一日,孩子問母親,說道:「娘,這等清平世界,咱家也盡挨得過,為何時時掉下淚來?娘,你說與咱,咱也好分憂的。」老娘就說:「小孩子,你那裡知道。自從你老頭兒去世,你大哥兒到邊上去做了長官,四五年,信兒也沒一個。不知他生死存亡,教我老人家怎生弔的下!」說著,又哭起來。那孩子說:「早是這等,有何難哉!娘,如今哥在那裡?咱做弟郎的,早晚間走去抓尋哥兒,討個信來,回覆你老人家,卻不是好?」那婆婆一頭哭,一頭笑起來,說道:「怪獃子,你哥若是一百二百裡程途,便可去的,直在那遼東地面,去此一萬餘里,就是好漢子,也走四五個月才到哩,你孩兒家怎麼去的?」那孩子就說:「嗄,若是果在遼東,也終不在個天上,我去尋哥兒就回也。」只見他把趿鞋兒系好了,把直掇兒整一整,望著婆兒拜個揖,一溜煙去了。那婆婆叫之不應,追之不及,愈添愁悶。也有鄰舍街坊、婆兒婦女前來解勸,說道:「孩兒小,怎去的遠?早晚間自回也。」因此,婆婆收著兩眶眼淚,悶悶坐的。看看紅日西沉,那婆婆探頭探腦向外張望,只見遠遠黑魆魆影兒里,有一個小的兒來也。那婆婆就說:「靠天靠地,靠日月三光。若的俺小的兒子來了,也不枉了俺修齋吃素的念頭。」只見那萬回老祖忽地跪到跟前說:「娘,你還未睡哩?咱已到遼東抓尋哥兒,討的平安家信來也。」婆婆笑道:「孩兒,你不去的正好,免教我老人家掛心。只是不要弔慌哄著老娘。那有一萬里路程朝暮往還的?」孩兒道:「娘,你不信麼?」一直卸下衣包,取出平安家信,果然是他哥兒手筆。又取出一件汗衫,帶回漿洗,也是婆婆親手縫的,毫釐不差。因此哄動了街坊,叫做「萬回」。日後舍俗出家,就叫做「萬回長老」。果然道德高妙,神通廣大。曾在後趙皇帝石虎跟前,吞下兩升鐵針,又在梁武皇殿下,在頭頂上取出舍利三顆。因此敕建永福禪寺,做萬回老祖的香火院,正不知費了多少錢糧。正是:
\begin{quote}
神僧出世神通大,聖主尊隆聖澤深。
\end{quote}

不想歲月如梭,時移事改。那萬回老祖歸天圓寂,就有些得皮得肉的上人們,一個個多化去了。只有幾個憊賴和尚,養老婆,吃燒酒,甚事兒不弄出來!不消幾日兒,把袈裟也當了,鐘兒、磬兒都典了,殿上椽兒、磚兒、瓦兒換酒吃了。弄的那雨淋風刮,佛像兒倒的,荒荒涼涼,將一片鐘鼓道場,忽變作荒煙衰草。三四十年,那一個肯扶衰起廢!不想有個道長老,原是西印度國出身,因慕中國清華,打從流沙河、星宿海走了八九個年頭,才到中華區處。迤邐來到山東,就卓錫在這個破寺里,面壁九年,不言不語,真個是:
\begin{quote}
佛法原無文字障,工夫向好定中尋。
\end{quote}

忽一日發個念頭,說道:「呀,這寺院坍塌的不成模樣了,這些蠢狗才攮的禿驢,止會吃酒噇飯,把這古佛道場弄得赤白白地,豈不可惜!到今日,咱不做主,那個做主?咱不出頭,那個出頭?況山東有個西門大官人,居錦衣之職,他家私巨萬,富比王侯,前日餞送蔡御史,曾在咱這裡擺設酒席。他見寺宇傾頹,就有個鼎建重新的意思。若得他為主作倡,管情早晚間把咱好事成就也。咱須去走一遭。」當時喚起法子徒孫,打起鐘鼓,舉集大眾,上堂宣揚此意。那長老怎生打扮?但見:
\begin{quote}
身上禪衣猩血染,雙環掛耳是黃金。
手中錫杖光如鏡,百八明珠耀日明。
開覺明路現金繩,提起凡夫夢亦醒。
龐眉紺發銅鈴眼,道是西天老聖僧。
\end{quote}

長老宣揚已畢,就叫行者拿過文房四寶,寫了一篇疏文。好長老,真個是古佛菩薩現身。於是辭了大眾,著上禪鞋,戴上個斗笠子,一壁廂直奔到西門慶家裡來。

且說西門慶辭別了應伯爵,走到吳月娘房內,把應伯爵薦水秀才的事體說了一番,就說道:「咱前日東京去,多得眾親朋與咱把盞,如今少不的也要整酒回答他。今日到空閑,就把這事兒完了罷。」當下就叫了玳安,吩咐買辦嗄飯之類。又吩咐小廝,分頭去請各位。一面拉著月娘,走到李瓶兒房裡來看官哥。李瓶兒笑嘻嘻的接住了,就叫奶子抱出官哥兒來。只見眉目稀疏,就如粉塊妝成,笑欣欣,直攛到月娘懷裡來。月娘把手接著,抱起道:「我的兒,恁的乖覺,長大來,定是聰明伶俐的。」又向那孩子說:「兒,長大起來,恁地奉養老娘哩!」李瓶兒就說:「娘說那裡話。假饒兒子長成,討的一官半職,也先向上頭封贈起,那鳳冠霞帔,穩穩兒先到娘哩。」西門慶接口便說:「兒,你長大來還掙個文官。不要學你家老子做個西班出身,——雖有興頭,卻沒十分尊重。」正說著,不想潘金蓮在外邊聽見,不覺怒從心上起,就罵道:「沒廉恥、弄虛脾的臭娼根,偏你會養兒子!也不曾經過三個黃梅、四個夏至,又不曾長成十五六歲,出幼過關,上學堂讀書,還是個水泡,與閻羅王合養在這裡的,怎見的就做官,就封贈那老夫人?怪賊囚根子,沒廉恥的貨,怎的就見的要做文官,不要象你!」正在嘮嘮叨叨,喃喃吶吶,一頭罵,一頭著惱的時節,只見玳安走將進來,叫聲「五娘」,說道:「爹在那裡?」潘金蓮便罵:「怪尖嘴的賊囚根子,那個曉的你什麼爹在那裡!怎的到我這屋裡來?他自有五花官誥的太奶奶老封婆,八珍五鼎奉養他的在那裡,那裡問著我討!」那玳安就曉的不是路了,望六娘房裡就走。走到房門前,打個咳嗽,朝著西門慶道:「應二爹在廳上。」西門慶道:「應二爹,才送的他去,又做甚?」玳安道:「爹出去便知。」

西門慶只得撇了月娘、李瓶兒,走到外邊。見伯爵,正要問話,只見那募緣的道長老已到西門慶門首了。高聲叫:「阿彌陀佛!這是西門老爹門首麼?那個掌事的管家與吾傳報一聲,說道:扶桂子,保蘭孫,求福有福,求壽有壽。——東京募緣的長老求見。」原來,西門慶平日原是一個撒漫使錢的漢子,又是新得官哥,心下十分歡喜,也要幹些好事,保佑孩兒。小廝們通曉得,並不作難,一壁廂進報西門慶。西門慶就說:「且叫他進來看。」不一時,請那長老進到花廳裡面,打了個問訊,說道:「貧僧出身西印度國,行腳到東京汴梁,卓錫在永福禪寺,面壁九年,頗傳心印。止為那宇殿傾頹,琳宮倒塌,貧僧想起來,為佛弟子,自應為佛出力,因此上貧僧發了這個念頭。前日老檀越餞行各位老爹時,悲憐本寺廢壞,也有個良心美腹,要和本寺作主。那時,諸佛菩薩已作證盟。貧僧記的佛經上說得好:如有世間善男子、善女人以金錢喜舍莊嚴佛像者,主得桂於蘭孫,端嚴美貌,日後早登科甲,蔭子封妻之報。故此特叩高門,不拘五百一千,要求老檀那開疏發心,成就善果。」就把錦帕展開,取出那募緣疏簿,雙手遞上。不想那一席話兒,早已把西門慶的心兒打動了,不覺的歡天喜地接了疏簿,就叫小廝看茶。揭開疏簿,只見寫道:
\begin{quote}
伏以白馬駝經開象教,竺騰衍法啟宗門。大地眾僧,無不皈依佛祖;三千世界,盡皆蘭若莊嚴。看此瓦礫傾頹,成甚名山勝境?若不慈悲喜舍,何稱佛子仁人?今有永福禪寺,古佛道場,焚修福地。啟建自梁武皇帝,開山是萬回祖師。規制恢弘,彷彿那給孤園黃金鋪地;雕樓精製,依稀似祇洹舍白玉為階。高閣摩空,旃檀氣直接九霄雲表;層基亙地,大雄殿可容千眾禪僧。兩翼巍峨,儘是琳宮紺宇;廊房潔凈,果然精勝洞天。那時鐘鼓宣揚,盡道是寰中佛國;只這緇流濟楚,卻也像塵界人天。那知歲久年深,一瞬時移事換。莽和尚縱酒撒潑,毀壞清規;呆道人懶惰貪眠,不行打掃。漸成寂寞,斷絕門徒;以致凄涼,罕稀瞻仰。兼以鳥鼠穿蝕,那堪風雨漂搖。棟宇摧頹,一而二,二而三,支撐靡計;牆垣坍塌,日復日,年復年,振起無人。朱紅欞槅,拾來煨酒煨茶;合抱棟梁,拿去換鹽換米。風吹羅漢金消盡,雨打彌陀化作塵。吁嗟乎!金碧焜炫,一旦為灌莽荊榛。雖然有成有敗,終須否極泰來。幸而有道長老之虔誠,不忍見梵王宮之廢敗。發大弘願,遍叩檀那。伏願咸起慈悲,盡興惻隱。樑柱椽楹,不拘大小,喜舍到高題姓字;銀錢布幣,豈論豐贏,投櫃入疏簿標名。仰仗著佛祖威靈,福祿壽永永百年千載;倚靠他伽藍明鏡,父子孫個個厚祿高官。瓜瓞綿綿,森挺三槐五桂;門庭奕奕,輝煌金阜錢山。凡所營求,吉祥如意。疏文到日,各破慳心。謹疏。
\end{quote}

西門慶看畢,恭恭敬敬放在桌兒上面,對長老說:「實不相瞞,在下雖不成個人家,也有幾萬產業,忝居武職。不想偌大年紀,未曾生下兒子,有意做些善果。去年第六房賤內生下孩子,咱萬事已是足了。偶因餞送俺友,得到上方,因見廟字傾頹,實有個舍財助建的念頭。蒙老師下顧,那敢推辭!」拿著兔毫妙筆,正在躊躇之際,應伯爵就說:「哥,你既有這片好心為侄兒發願,何不一力獨成,也是小可的事體。」西門慶拿著筆笑道:「力薄,力薄。」伯爵又道:「極少也助一千。」西門慶又笑道:「力薄,力薄。」那長老就開口說道:「老檀越在上,不是貧僧多口,我們佛家的行徑,只要隨緣喜舍,終不強人所難,但憑老爹發心便是。此外親友,更求檀越吹噓吹噓。」西門慶說道:「還是老師體量。少也不成,就寫上五百兩。」擱了兔毫筆,那長老打個問訊謝了。西門慶又說:「我這裡內官太監、府縣倉巡,一個個都與我相好的,我明日就拿疏簿去要他們寫。寫的來,就不拘三百二百、一百五十,管情與老師成就這件好事。」當日留了長老素齋,相送出門。正是:
\begin{quote}
慈悲作善豪家事,保福消災父母心。
\end{quote}

西門慶送了長老,轉到廳上,與應伯爵坐地,道:「我正要差人請你,你來的正好。我前日往東京,多謝眾親友們與咱把盞,今日安排小酒與眾人回答,要二哥在此相陪,不想遇著這個長老,鬼混了一會兒。」伯爵便說道:「好個長老,想是果然有德行的。他說話中間,連咱也心動起來,做了施主。」西門慶說道:「你又幾時做施主來?疏簿又是幾時寫的?」應伯爵笑道:「哥,你不知道,佛經上第一重的是心施,第二法施,第三才是財施。難道我從旁攛掇的,不當個心施?」西門慶笑道:「二哥,只怕你有口無心哩。」兩人拍手大笑,應伯爵就說:「小弟在此等待客來,哥有正事,自與嫂子商議去。」

只見西門慶別了伯爵,轉到內院裡頭,只見那潘金蓮嘮嘮叨叨,沒揪沒採,不覺的睡魔纏擾,打了幾個噴涕,走到房中,倒在象牙床上睡去了。李瓶兒又為孩子啼哭,自與奶子、丫鬟在房中坐地,看官哥。只有吳月娘與孫雪娥兩個看著整辦嗄飯。西門慶走到面前坐的,就把道長老募緣與自己開疏的事,備細說了一番。又把應伯爵耍笑打覷的話也說了一番。歡天喜地,大家嘻笑了一會。那吳月娘畢竟是個正經的人,不慌不忙說下幾句話兒,到是西門慶頂門上針。正是:
\begin{quote}
妻賢每至雞鳴警,款語常聞藥石言。
\end{quote}

月娘說道:「哥,你天大的造化,生下孩兒。你又發起善念。廣結良緣,豈不是俺一家兒的福分!只是那善念頭怕他不多,那惡念頭怕他不盡。哥,你日後那沒來回沒正經養婆娘、沒搭煞貪財好色的事體少乾幾樁兒,卻不儹下些陰功,與那小孩子也好!」西門慶笑道:「你的醋話兒又來了。卻不道天地尚有陰陽,男女自然配合。今生偷情的、苟合的,都是前生分定,姻緣簿上註名,今生了還,難道是生剌剌胡搊亂扯歪廝纏做的?咱聞那佛祖西天,也止不過要黃金鋪地,陰司十殿,也要些楮鏹營求。咱只消盡這家私廣為善事,就使強姦了姮娥,和姦了織女,拐了許飛瓊,盜了西王母的女兒,也不減我潑天的富貴。」月娘笑道:「狗吃熱屎,原道是個香甜的;生血掉在牙兒內,怎生改得!」

正在笑間,只見王姑子同了薛姑子,提了一個盒兒,直闖進來,朝月娘打問訊,又向西門慶拜了拜,說:「老爹,你倒在家裡。」月娘一面讓坐。看官聽說,原來這薛姑子不是從幼出家的,少年間曾嫁丈夫,在廣成寺前賣蒸餅兒生理。不料生意淺薄,與寺里的和尚、行童調嘴弄舌,眉來眼去,刮上了四五六個。常有些饅頭齋供拿來進奉他,又有那應付錢與他買花,開地獄的布,送與他做裹腳。他丈夫那裡曉得!以後,丈夫得病死了,他因佛門情熟,就做了個姑子。專一在士夫人家往來,包攬經懺。又有那些不長進、要偷漢子的婦人,叫他牽引。聞得西門慶家裡豪富,侍妾多人,思想拐些用度,因此頻頻往來。有一隻歌兒道得好:
\begin{quote}
尼姑生來頭皮光,拖子和尚夜夜忙。三個光頭好象師父師兄並師弟,只是鐃鈸原何在里床?
\end{quote}

薛姑子坐下,就把小盒兒揭開,說道:「咱每沒有甚麼孝順,拿得施主人家幾個供佛的果子兒,權當獻新。」月娘道:「要來竟自來便了,何苦要你費心!」只見潘金蓮睡覺,聽得外邊有人說話,又認是前番光景,便走向前來聽看。見李瓶兒在房中弄孩子,因曉得王姑子在此,也要與他商議保佑官哥。因一同走到月娘房中。大家道個萬福,各各坐地。西門慶因見李瓶兒來,又把那道長老募緣與自家開疏舍財,替官哥求福的事情,又說一番。不想惱了潘金蓮,抽身竟走,喃喃噥噥,竟自去了。那薛姑子聽了,就站將起來,合掌叫聲:「佛阿!老爹你這等樣好心作福,怕不的壽年千歲,五男二女,七子團圓。只是我還有一件說與你老人家——這個因果費不甚多,更自獲福無量。咦,老檀越,你若干了這件功德,就是那老瞿曇雪山修道,迦葉尊散髮鋪地,二祖師投崖飼虎,給孤老滿地黃金,也比不得你功德哩!」西門慶笑道:「姑姑且坐下,細說甚麼功果,我便依你。」薛姑子就說:「我們佛祖留下一捲《陀羅經》,專一勸人生西方凈土。因為那肉眼凡夫不生尊信,故此佛祖演說此經,勸你專心念佛,竟往西方,永永不落輪迴。那佛祖說的好,如有人持誦此經,或將此經印刷抄寫,轉勸一人至千萬人持誦,獲福無量。況且此經裡面又有《護諸童子經》兒,凡有人家生育男女,必要從此發心,方得易長易養,災去福來。如今這副經板現在,只沒人印刷施行。老爹只消破些工料印上幾千捲,裝釘完成,普施十方。那個功德真是大的緊。」西門慶道:「這也不難,只不知這一捲經要多少紙札,多少裝釘,多少印刷,有個細數才好動彈。」薛姑子又道:「老爹,你那裡去細細算他,止消先付九兩銀子,叫經坊里印造幾千萬卷,裝釘完滿,以後一攪果算還他就是了。」

正說的熱鬧,只見陳敬濟要與西門慶說話,尋到捲棚底下,剛剛湊巧遇著了潘金蓮憑欄獨惱。猛抬頭兒見了敬濟,就是貓兒見了魚鮮飯一般,不覺把一天愁悶都改做春風和氣。兩個見沒有人來,就執手相偎,剝嘴咂舌頭。兩個肉麻頑了一回,又恐怕西門慶出來撞見,連算帳的事情也不提了。一雙眼又象老鼠兒防貓,左顧右盼,要做事又沒個方便,只得一溜煙出去了。

且說西門慶聽了薛姑子的話頭,不覺又動了一片善心,就叫玳安拿拜匣,取出一封銀子,準準三十兩,便交付薛姑子與王姑子:「即便同去經坊里,與我印下五千捲經,待完了,我就算帳找他。」正話間,只見書童忙忙來報道:「請的各位客人都到了。」少不的是吳大舅、花大舅、謝希大、常峙節這一班。西門慶忙整衣出外迎接升堂。就叫小廝擺下桌兒,請眾人一行兒分班列次,各敘長幼坐的。不一時,大魚大肉、時新果品,一齊兒捧將出來。只見酒逢知己,形跡都忘。猜枚的、打鼓的、催花的,三拳兩謊的,歌的歌,唱的唱,頑不盡少年場光景,說不了醉鄉裡日月。正是:
\begin{quote}
秋月春花隨處有,賞心樂事此時同。
\end{quote}
