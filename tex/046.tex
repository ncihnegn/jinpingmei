
\chapter{元夜遊行遇雪雨 妻妾戲笑卜龜兒}

詞曰:
\begin{quote}
小市東門欲雪天,眾中依約見神仙。蕊黃香細貼金蟬。
飲散黃昏人草草,醉容無語立門前。馬嘶塵哄一街煙。
\end{quote}

話說西門慶那日,打發吳月娘眾人往吳大妗子家吃酒去了。李智、黃四約坐到黃昏時分,就告辭起身。伯爵趕送出去,如此這般告訴:「我已替二公說了,準在明日還找五百兩銀子。」那李智、黃四向伯爵打了恭又打恭,去了。伯爵復到廂房中,和謝希大陪西門慶飲酒,只見李銘掀帘子進來。伯爵看見,便道:「李日新來了。」李銘扒在地下磕頭。西門慶問道:「吳惠怎的不來?」李銘道:「吳惠今日東平府官身也沒去,在家裡害眼。小的叫了王柱來了。」便叫王柱:「進來,與爹磕頭。」那王柱掀簾進入房裡,朝上磕了頭,與李銘站立在旁。伯爵道:「你家桂姐剛纔家去了,你不知道?」李銘道:「小的官身到家,洗了洗臉就來了,並不知道。」伯爵向西門慶說:「他兩個怕不的還沒吃飯哩,哥吩咐拿飯與他兩個吃。」書童在旁說:「二爹,叫他等一等,亦發和吹打的一答里吃罷,敢也拿飯去了。」伯爵令書童取過一個托盤來,桌上掉了兩碟下飯,一盤燒羊肉,遞與李銘:「等拿了飯來,你每拿兩碗在這明間吃罷。」說書童兒:「我那傻孩子,常言道:方以類聚,物以群分。你不知,他這行人故雖是當院出身,小優兒比樂工不同,一概看待也罷了,顯的說你我不幫襯了。」被西門慶向伯爵頭上打了一下,笑罵道:「怪不的你這狗才,行計中人只護行計中人,又知這當差的甘苦。」伯爵道:「傻孩兒,你知道甚麼!你空做子弟一場,連『惜玉憐香』四個字你還不曉的。粉頭、小優兒如同鮮花一般,你惜憐他,越發有精神。你但折剉他,敢就《八聲甘州》懨懨瘦損,難以存活。」西門慶笑道:「還是我的兒曉的道理。」

那李銘、王柱須臾吃了飯,應伯爵叫過來吩咐:「你兩個會唱『雪月風花共裁剪』不會?」李銘道:「此是黃鐘,小的每記的。」於是,王柱彈琵琶,李銘𢺈箏,頓開喉音唱了一套。唱完了,看看晚來,正是:
\begin{quote}
金烏漸漸落西山,玉兔看看上畫闌;
佳人款款來傳報,月透紗窗衾枕寒。
\end{quote}

西門慶命收了家火,使人請傅伙計、韓道國、雲主管、賁四、陳敬濟,大門首用一架圍屏安放兩張桌席,懸掛兩盞羊角燈,擺設酒筵,堆集許多春檠果盒,各樣餚饌。西門慶與伯爵、希大都一帶上面坐了,伙計、主管兩旁打橫。大門首兩邊,一邊十二盞金蓮燈。還有一座小煙火,西門慶吩咐等堂客來家時放。先是六個樂工,抬銅鑼銅鼓在大門首吹打。吹打了一回,又請吹細樂上來。李銘、王柱兩個小優兒箏、琵琶上來,彈唱燈詞。那街上來往圍看的人,莫敢仰視。西門慶帶忠靖冠,絲絨鶴氅,白綾襖子。玳安與平安兩個,一遞一桶放花兒。兩名排軍執攬桿攔擋閑人,不許向前擁擠。不一時,碧天雲靜,一輪皓月東升之時,街上遊人十分熱鬧,但見:
\begin{quote}
戶戶鳴鑼擊鼓,家家品竹彈絲。遊人隊隊踏歌聲,士女翩翩垂舞調。鰲山結彩,巍峨百尺矗晴雲;鳳禁褥香,縹緲千層籠綺隊。閑庭內外,溶溶寶月光輝;畫閣高低,燦燦花燈照耀。三市六街人鬧熱,鳳城佳節賞元宵。
\end{quote}

且說春梅、迎春、玉簫、蘭香、小玉眾人,見月娘不在,聽見大門首吹打銅鼓彈唱,又放煙火,都打扮著走來,在圍屏後扒著望外瞧。書童兒和畫童兒兩個,在圍屏後火盆上篩酒。原來玉簫和書童舊有私情,兩個常時戲狎。兩個因按在一處奪瓜子兒嗑,不防火盆上坐著一錫瓶酒,推倒了,那火烘烘望上騰起來,漰了一地灰起去。那玉簫還只顧嘻笑,被西門慶聽見,使下玳安兒來問:「是誰笑?怎的這等灰起?」那日春梅穿著新白綾襖子,大紅遍地金比甲,正坐在一張椅兒上,看見他兩個推倒了酒,就揚聲罵玉簫道:「好個怪浪的淫婦!見了漢子,就邪的不知怎麼樣兒的了,只當兩個把酒推倒了才罷了。都還嘻嘻哈哈,不知笑的是甚麼!把火也漰死了,平白落人恁一頭灰。」玉簫見他罵起來,唬的不敢言語,往後走了。慌的書童兒走上去,回說:「小的火盆上篩酒來,扒倒了錫瓶里酒了。」西門慶聽了,便不問其長短,就罷了。

先是那日,賁四娘子打聽月娘不在,平昔知道春梅、玉簫、迎春、蘭香四個是西門慶貼身答應得寵的姐兒,大節下安排了許多菜蔬果品,使了他女孩兒長兒來,要請他四個去他家裡坐坐。眾人領了來見李嬌兒。李嬌兒說:「我燈草拐杖——做不得主。你還請問你爹去。」問雪娥,雪娥亦發不敢承攬。看看挨到掌燈以後,賁四娘子又使了長兒來邀四人。蘭香推玉簫,玉簫推迎春,迎春推春梅,要會齊了轉央李嬌兒和西門慶說,放他去。那春梅坐著,紋絲兒也不動,反罵玉簫等:「都是那沒見食麵的行貨子,從沒見酒席,也聞些氣兒來!我就去不成,也不到央及他家去。一個個鬼攛攥的也似,不知忙些甚麼,教我半個眼兒看的上!」那迎春、玉簫、蘭香都穿上衣裳,打扮的齊齊整整齣來,又不敢去,這春梅又只顧坐著不動身。書童見賁四嫂又使了長兒來邀,說道:「我拚著爹罵兩句也罷,等我上去替姐每稟稟去。」一直走到西門慶身邊,附耳說道:「賁四嫂家大節間要請姐每坐坐,姐教我來稟問爹,去不去?」西門慶聽了,吩咐:「教你姐每收拾去,早些來,家裡沒人。」這書童連忙走下來,說道:「還虧我到上頭,一言就準了。教你姐快收拾去,早些來。」那春梅才慢慢往房裡勻施脂粉去了。

不一時,四個都一答兒里出門。書童扯圍屏掩過半邊來,遮著過去。到了賁四家,賁四娘子見了,如同天上落下來的一般,迎接進屋裡。頂槅上點著繡球紗燈,一張桌兒上整齊餚菜。趕著春梅叫大姑,迎春叫二姑,玉簫是三姑,蘭香是四姑,都見過禮。又請過韓回子娘子來相陪。春梅、迎春上坐,玉簫、蘭香對席,賁四嫂與韓回子娘子打橫,長兒往來燙酒拿菜。按下這裡不題。

西門慶因叫過樂工來吩咐:「你每吹一套『東風料悄』《好事近》與我聽。」正值後邊拿上玫瑰元宵來,眾人拿起來同吃,端的香甜美味,入口而化,甚應佳節。李銘、王柱席前拿樂器,接著彈唱此詞,端的聲韻悠揚,疾徐合節。這裡彈唱飲酒不題。

且說玳安與陳敬濟袖著許多花炮,又叫兩個排軍拿著兩個燈籠,竟往吳大妗子家來接月娘。眾人正在明間飲酒,見了陳敬濟來:「教二舅和姐夫房裡坐,你大舅今日不在家,衛里看著造冊哩。」一面放桌兒,拿春盛點心酒菜上來,陪敬濟。玳安走到上邊,對月娘說:「爹使小的來接娘每來了,請娘早些家去,恐晚夕人亂,和姐夫一答兒來了。」月娘因頭裡惱他,就一聲兒沒言語答他。吳大妗子便叫來定兒:「拿些兒甚麼與玳安兒吃。」來定兒道:「酒肉湯飯,都前頭擺下了。」吳月娘道:「忙怎的?那裡才來乍到就與他吃!教他前邊站著,我每就起身。」吳大妗子道:「三姑娘慌怎的?上門兒怪人家?大節下,姊妹間,眾位開懷大坐坐兒。左右家裡有他二娘和他姐在家裡,怕怎的?老早就要家去!是別人家又是一說。」因叫鬱大姐:「你唱個好曲兒,伏侍他眾位娘。」孟玉樓道:「他六娘好不惱他哩,說你不與他做生日。」鬱大姐連忙下席來,與李瓶兒磕了四個頭,說道:「自從與五娘做了生日,家去就不好起來。昨日妗奶奶這裡接我,教我才收拾䦛䦟了來。若好時,怎的不與你老人家磕頭?」金蓮道:「鬱大姐,你六娘不自在哩,你唱個好的與他聽,他就不惱你了。」那李瓶兒在旁只是笑,不做聲。鬱大姐道:「不打緊,拿琵琶過來,等我唱。」大妗子叫吳舜臣媳婦鄭三姐:「你把你二位姑娘和眾位娘的酒兒斟上。這一日還沒上過鐘酒兒。」那鬱大姐接琵琶在手,用心用意唱了一個《一江風》。

正唱著,月娘便道:「怎的這一回子恁涼凄凄的起來?」來安兒在旁說道:「外邊天寒下雪哩。」孟玉樓道:「姐姐,你身上穿的不單薄?我倒帶了個綿披襖子來了。咱這一回,夜深不冷麼?」月娘道:「既是下雪,叫個小廝家裡取皮襖來咱每穿。」那來安連忙走下來,對玳安說:「娘吩咐,叫人家去取娘們皮襖哩。」那玳安便叫琴童兒:「你取去罷,等我在這裡伺候。」那琴童也不問,一直家去了。少頃,月娘想起金蓮沒皮襖,因問來安兒:「誰取皮襖去了?」來安道:「琴童取去了。」月娘道:「也不問我,就去了。」玉樓道:「剛纔短了一句話,不該教他拿俺每的,他五娘沒皮襖,只取姐姐的來罷。」月娘道:「怎的沒有?還有當的人家一件皮襖,取來與六姐穿就是了。」因問:「玳安那奴才怎的不去,卻使這奴才去了?你叫他來!」一面把玳安叫到跟前,吃月娘儘力罵了幾句道:「好奴才!使你怎的不動?又坐壇遣將兒,使了那個奴才去了。也不問我聲兒,三不知就去了。怪不的你做大官兒,恐怕打動你展翅兒,就只遣他去!」玳安道:「娘錯怪了小的。頭裡娘吩咐若是叫小的去,小的敢不去?來安下來,只說叫一個家裡去。」月娘道:「那來安小奴才敢吩咐你?俺每恁大老婆,還不敢使你哩!如今慣的你這奴才們有些摺兒也怎的?一來主子煙薰的佛像——掛在牆上,有恁施主,有恁和尚。你說你恁行動兩頭戳舌,獻勤出尖兒,外合里應,好懶食饞,背地瞞官作弊,乾的那繭兒我不知道哩!頭裡你家主子沒使你送李桂兒家去,你怎的送他?人拿著氈包,你還匹手奪過去了。留丫頭不留丫頭不在你,使你進來說,你怎的不進來?你便送他,圖嘴吃去了,卻使別人進來。須知我若罵只罵那個人了。你還說你不久慣牢成!」玳安道:「這個也沒人,就是畫童兒過的舌。爹見他抱著氈包,教我:『你送送你桂姨去罷』,使了他進來的。娘說留丫頭不留丫頭不在於小的,小的管他怎的!」月娘大怒,罵道:「賊奴才,還要說嘴哩!我可不這裡閑著和你犯牙兒哩。你這奴才,脫脖倒㘭過颺了。我使著不動,耍嘴兒,我就不信到明日不對他說,把這欺心奴才打與你個爛羊頭也不算。」吳大妗子道:「玳安兒,還不快替你娘每取皮襖去。」又道:「姐姐,你吩咐他拿那裡皮襖與他五娘穿?」潘金蓮接過來說道:「姐姐,不要取去,我不穿皮襖,教他家裡捎了我的披襖子來罷。人家當的,好也歹也,黃狗皮也似的,穿在身上,教人笑話,也不長久,後還贖的去了。」月娘道:「這皮襖倒不是當的,是李智少十六兩銀子準折的。當的王招宣府里那件皮襖,與李嬌兒穿了。」因吩咐玳安:「皮襖在大櫥里,叫玉簫尋與你,就把大姐的皮襖也帶了來。」

玳安把嘴谷都,走出來,陳敬濟問道:「你到那去?」玳安道:「精是攮氣的營生,一遍生活兩遍做,這咱晚又往家裡跑一遭。」逕走到家。西門慶還在大門首吃酒,傅伙計、雲主管都去了,還有應伯爵、謝希大、韓道國、賁四眾人吃酒未去,便問玳安:「你娘們來了?」玳安道:「沒來,使小的取皮襖來了。」說畢,便往後走。先是琴童到家,上房裡尋玉簫要皮襖。小玉坐在炕上正沒好氣,說道:「四個淫婦今日都在賁四老婆家吃酒哩。我不知道皮襖放在那裡,往他家問他要去。」這琴童一直走到賁四家,且不叫,在窗外悄悄覷聽。只見賁四嫂說道:「大姑和三姑,怎的這半日酒也不上,菜兒也不揀一箸兒?嫌俺小家兒人家,整治的不好吃也怎的?」春梅道:「四嫂,俺每酒夠了。」賁四嫂道:「耶嚛!沒的說。怎的這等上門兒怪人家!」又叫韓回子老婆:「你是我的切鄰,就如副東一樣,三姑、四姑跟前酒,你也替我勸勸兒,怎的單板著,象客一般?」又叫長姐:「篩酒來,斟與三姑吃,你四姑鐘兒淺斟些兒罷。」蘭香道:「我自來吃不的。」賁四嫂道:「你姐兒們今日受餓,沒甚麼可口的菜兒管待,休要笑話。今日要叫了先生來,唱與姑娘們下酒,又恐怕爹那裡聽著。淺房淺屋,說不的俺小家兒人家的苦。」說著,琴童兒敲了敲門,眾人都不言語了。長兒問:「是誰?」琴童道:「是我,尋姐說話。」一面開了門,那琴童入來。玉簫便問:「娘來了?」那琴童看著待笑,半日不言語。玉簫道:「怪雌牙的,誰與你雌牙?問著不言語。」琴童道:「娘每還在妗子家吃酒哩,見天陰下雪,使我來家取皮襖來,都教包了去哩。」玉簫道:「皮襖在描金箱子里不是,叫小玉拿與你。」琴童道:「小玉說教我來問你要。」玉簫道:「你信那小淫婦兒,他不知道怎的!」春梅道:「你每有皮襖的,都打發與他。俺娘沒皮襖,只我不動身。」蘭香對琴童:「你三娘皮襖,問小鸞要。」迎春便向腰裡拿鑰匙與琴童兒:「教繡春開裡間門拿與你。」

琴童兒走到後邊,上房小玉和玉樓房中小鸞,都包了皮襖交與他。正拿著往外走,遇見玳安,問道:「你來家做甚麼?」玳安道:「你還說哩!為你來了,平白教大娘罵了我一頓好的。又使我來取五娘的皮襖來。」琴童道:「我如今取六娘的皮襖去也。」玳安道:「你取了,還在這裡等著我,一答兒里去。你先去了不打緊,又惹的大娘罵我。」說畢,玳安來到上房。小玉正在炕上籠著爐臺烤火,口中嗑瓜子兒,見了玳安,問道:「你也來了?」玳安道:「你又說哩,受了一肚子氣在這裡。娘說我遣將兒。因為五娘沒皮襖,又教我來,說大櫥里有李三準折的一領皮襖,教拿去哩。」小玉道:「玉簫拿了裡間門上鑰匙,都在賁四家吃酒哩,教他來拿。」玳安道:「琴童往六娘房裡去取皮襖,便來也,教他叫去,我且歇歇腿兒,烤烤火兒著。」那小玉便讓炕頭兒與他,並肩相挨著向火。小玉道:「壺裡有酒,篩盞子你吃?」玳安道:「可知好哩,看你下顧。」小玉下來,把壺坐在火上,抽開抽屜,拿了一碟子臘鵝肉,篩酒與他。無人處兩個就摟著咂舌親嘴。

正吃著酒,只見琴童兒進來。玳安讓他吃了一盞子,便使他:「叫玉簫姐來,拿皮襖與五娘穿。」那琴童抱氈包放下,走到賁四家叫玉簫。玉簫罵道:「賊囚根子,又來做甚麼?」又不來。遞與鑰匙,教小玉開門。那小玉開了裡間房門,取了一把鑰匙,通了半日,白通不開。琴童兒又往賁四家問去。那玉簫道:「不是那個鑰匙。娘櫥里鑰匙在床褥子座下哩。」小玉又罵道:「那淫婦丁子釘在人家不來,兩頭來回,只教使我。」及開了,櫥里又沒皮襖。琴童兒來回走的抱怨道:「就死也死三日三夜,又撞著恁瘟死鬼小奶奶兒們,把人魂也走出了。」向玳安道:「你說此回去,又惹的娘罵。不說屋裡,只怪俺們。」走去又對玉簫說:「裡間娘櫥里尋,沒有皮襖。」玉簫想了想,笑道:「我也忘記,在外間大櫥里。」到後邊,又被小玉罵道:「淫婦吃那野漢子搗昏了,皮襖在這裡,卻到處尋。」一面取出來,將皮襖包了,連大姐皮襖都交付與玳安、琴童。

兩個拿到吳大妗子家,月娘又罵道:「賊奴才,你說同了都不來罷了。」那玳安不敢言語,琴童道:「娘的皮襖都有了,等著姐又尋這件青鑲皮襖。」於是打開取出來。吳大妗子燈下觀看,說道:「好一件皮襖。五娘,你怎的說他不好,說是黃狗皮。那裡有恁黃狗皮,與我一件穿也罷了。」月娘道:「新新的皮襖兒,只是面前歇胸舊了些兒。到明日,從新換兩個遍地金歇胸,就好了。孟玉樓拿過來,與金蓮戲道:「我兒,你過來,你穿上這黃狗皮,娘與你試試看好不好。」金蓮道:「有本事到明日問漢子要一件穿,也不枉的。平白拾人家舊皮襖披在身上做甚麼!」玉樓戲道:「好個不認業的,人家有這一件皮襖,穿在身上念佛。」於是替他穿上。見寬寬大大,金蓮才不言語。

當下月娘與玉樓、瓶兒俱是貂鼠皮襖,都穿在身上,拜辭吳大妗子、二妗子起身。月娘與了鬱大姐一包二錢銀子。吳銀兒道:「我這裡就辭了妗子、列位娘,磕了頭罷。」當下吳大妗子與了一對銀花兒,月娘與李瓶兒每人袖中拿出一兩銀子與他,磕頭謝了。吳大妗子同二妗子、鄭三姐都還要送月娘眾人,因見天氣落雪,月娘阻回去了。琴童道:「頭裡下的還是雪,這回沾在身上都是水珠兒,只怕濕了娘們的衣服,問妗子這裡討把傘打了家去。」吳二舅連忙取了傘來,琴童兒打著,頭裡兩個排軍打燈籠,引著一簇男女,走幾條小巷,到大街上。陳敬濟沿路放了許多花炮,因叫:「銀姐,你家不遠了,俺每送你到家。」月娘便問:「他家在那裡?」敬濟道:「這條衚衕內一直進去,中間一座大門樓,就是他家。」吳銀兒道:「我這裡就辭了娘每家去。」月娘道:「地下濕,銀姐家去罷,頭裡已是見過禮了。我還著小廝送你到家。」因叫過玳安:「你送送銀姐家去。」敬濟道:「娘,我與玳安兩個去罷。」月娘道:「也罷,你與他兩個同送他送。」那敬濟得不的一聲,同玳安一路送去了。

吳月娘眾人便回家來。潘金蓮路上說:「大姐姐,你原說咱每送他家去,怎的又不去了?」月娘笑道:「你也只是個小孩兒,哄你說耍子兒,你就信了。麗春院是那裡,你我送去?」金蓮道:「像人家漢子在院里嫖了來,家裡老婆沒曾往那裡尋去?尋出沒曾打成一鍋粥?」月娘道:「你等他爹到明日往院里去,你尋他尋試試。倒沒的教人家漢子當粉頭拉了去,看你——」兩個口裡說著,看看走到東街上,將近喬大戶門首。只見喬大戶娘子和他外甥媳婦段大姐,在門首站立。遠遠見月娘一簇男女過來,就要拉請進去。月娘再三說道:「多謝親家盛情,天晚了,不進去罷。」那喬大戶娘子那裡肯放,說道:「好親家,怎的上門兒怪人家?」強把月娘眾人拉進去了。客位內掛著燈,擺設酒果,有兩個女兒彈唱飲酒,不題。

卻說西門慶,在門首與伯爵眾人飲酒將闌。伯爵與希大整吃了一日,頂顙吃不下去,見西門慶在椅子上打盹,趕眼錯把果碟兒都倒在袖子里,和韓道國就走了。只落下賁四,陪西門慶打發了樂工賞錢。吩咐小廝收家火,熄燈燭,歸後邊去了。只見平安走來,賁四家叫道:「你們還不起身,爹進去了。」玉簫聽見,和迎春、蘭香慌的辭也不辭,都一溜煙跑了。只落下春梅,拜謝了賁四嫂,才慢慢走回來。看見蘭香在後邊脫了鞋趕不上,因罵道:「你們都搶棺材奔命哩!把鞋都跑脫了,穿不上,象甚腔兒!」到後邊,打聽西門慶在李嬌兒房裡,都來磕頭。大師父見西門慶進入李嬌兒房中,都躲到上房,和小玉在一處。玉簫進來,道了萬福,那小玉就說玉簫:「娘那裡使小廝來要皮襖,你就不來管管兒,只教我拿。我又不知那根鑰匙開櫥門,及自開了又沒有,落後卻在外邊大櫥拒里尋出來。你放在裡頭,怎昏搶了不知道?姐姐每都吃勾來了罷,幾曾見長出塊兒來!」玉簫吃的臉紅紅的,道:「怪小淫婦兒,如何狗撾了臉似的?人家不請你,怎的和俺們使性兒!」小玉道:「我稀罕那淫婦請!」大師父在旁勸道:「姐姐每義讓一句兒罷,你爹在屋裡聽著。只怕你娘們來家,頓下些茶兒伺候。」正說著,只見琴童抱進氈包來。玉簫便問:「娘來了?」琴童道:「娘每來了,又被喬親家娘在門首讓進去吃酒哩,也將好起身。」兩個才不言語了。

不一時,月娘等從喬大戶娘子家出來。到家門首,賁四娘子走出來廝見。陳敬濟和賁四一面取出一架小煙火來,在門首又看放了一回煙火,方纔進來,與李嬌兒、大師父道了萬福。雪娥走來,向月娘磕了頭,與玉樓等三人見了禮。月娘因問:「他爹在那裡?」李嬌兒道:「剛纔在我那屋裡,我打發他睡了。」月娘一聲兒沒言語。只見春梅、迎春、玉簫、蘭香進來磕頭。李嬌兒便說:「今日前邊賁四嫂請了四個去,坐了回兒就來了。」月娘聽了,半日沒言語。罵道:「恁成精狗肉們,平白去做甚麼!誰教他去來?」李嬌兒道:「問過他爹才去來。」月娘道:「問他?好有張主的貨!你家初一十五開的廟門早了,放出些小鬼來了。」大師父道:「我的奶奶,恁四個上畫兒的姐姐,還說是小鬼。」月娘道:「上畫兒只畫的半邊兒,平白放出去做甚麼?與人家喂眼!」孟玉樓見月娘說來的不好,就先走了。落後金蓮見玉樓起身,和李瓶兒、大姐也走了。止落下大師父,和月娘同在一處睡了。那雪霰直下到四更方止。正是:
\begin{quote}
香消燭冷樓臺夜,挑菜燒燈掃雪天。
\end{quote}

一宿晚景題過。到次日,西門慶往衙門中去了。月娘約飯時前後,與孟玉樓、李瓶兒三個同送大師父家去。因在大門裡首站立,見一個鄉裡卜龜兒卦兒的老婆子,穿著水合襖、藍布裙子,勒黑包頭,背著褡褳,正從街上走來。月娘使小廝叫進來,在二門裡鋪下卦帖,安下靈龜,說道:「你卜卜俺每。」那老婆扒在地下磕了四個頭:「請問奶奶多大年紀?」月娘道:「你卜個屬龍的女命。」那老婆道:「若是大龍,四十二歲,小龍兒三十歲。」月娘道:「是三十歲了,八月十五日子時生。」那老婆把靈龜一擲,轉了一遭兒住了。揭起頭一張卦帖兒。上面畫著一個官人和一位娘子在上面坐,其餘都是侍從人,也有坐的,也有立的,守著一庫金銀財寶。老婆道:「這位當家的奶奶是戊辰生,戊辰己巳大林木。為人一生有仁義,性格寬洪,心慈好善,看經佈施,廣行方便。一生操持,把家做活,替人頂缸受氣,還不道是。喜怒有常,主下人不足。正是:
\begin{quote}
喜樂起來笑嘻嘻,惱將起來鬧哄哄。
\end{quote}
別人睡到日頭半天還未起,你老早在堂前轉了。梅香洗銚鐺,雖是一時風火性,轉眼卻無心。和人說也有,笑也有,只是這疾厄宮上著刑星,常沾些啾唧。虧你這心好,濟過來了,往後有七十歲活哩。」孟玉樓道:「你看這位奶奶命中有子沒有?」婆子道:「休怪婆子說,兒女宮上有些不實,往後只好招個出家的兒子送老罷了。隨你多少也存不的。」玉樓向李瓶兒笑道:「就是你家吳應元,見做道士家名哩。」月娘指著玉樓:「你也叫他卜卜。」玉樓道:「你卜個三十四歲的女命,十一月二十七日寅時生。」那婆子從新撇了卦帖,把靈龜一卜,轉到命宮上住了。揭起第二張卦帖來,上面畫著一個女人,配著三個男人:頭一個小帽商旅打扮;第二個穿紅官人;第三個是個秀才。也守著一庫金銀,左右侍從伏侍。婆子道:「這位奶奶是甲子年生。甲子乙醜海中金。命犯三刑六害,夫主克過方可。」玉樓道:「已克過了。」婆子道:「你為人溫柔和氣,好個性兒。你惱那個人也不知,喜歡那個人也不知,顯不出來。一生上人見喜下欽敬,為夫主寵愛。只一件,你饒與人為了美,多不得人心。命中一生替人頂缸受氣,小人駁雜,饒吃了還不道你是。你心地好了,雖有小人也拱不動你。」玉樓笑道:「剛纔為小廝討銀子和他亂了,這回說是頂缸受氣。」月娘道:「你看這位奶奶往後有子沒有?」婆子道:「濟得好,見個女兒罷了。子上不敢許,若說壽,倒盡有。」月娘道:「你卜卜這位奶奶。李大姐,你與他八字兒。」李瓶兒笑道:「我是屬羊的。」婆子道:「若屬小羊的,今年念七歲,辛未年生的。生幾月?」李瓶兒道:「正月十五日午時。」那婆子卜轉龜兒,到命宮上矻磴住了。揭起卦帖來,上面畫著一個娘子,三個官人:頭一個官人穿紅,第二個官人穿綠,第三個穿青。懷著個孩兒,守著一庫金銀財寶,旁邊立著個青臉獠牙紅髮的鬼。婆子道:「這位奶奶,庚午辛未路旁土。一生榮華富貴,吃也有,穿也有,所招的夫主都是貴人。為人心地有仁義,金銀財帛不計較,人吃了轉了他的,他喜歡;不吃他,不轉他,到惱。只是吃了比肩不和的虧,凡事恩將仇報。正是:
\begin{quote}
比肩刑害亂擾擾,轉眼無情就放刁;
寧逢虎摘三生路,休遇人前兩面刀。
\end{quote}
奶奶,你休怪我說:你盡好匹紅羅,只可惜尺頭短了些。氣惱上要忍耐些,就是子上也難為。」李瓶兒道:「今已是寄名做了道士。」婆子道:「既出了家,無妨了。又一件,你老人家今年計都星照命,主有血光之災,仔細七八月不見哭聲才好。」說畢,李瓶兒袖中掏出五分一塊銀子,月娘和玉樓每人與錢五十文。

剛打發卜龜卦婆子去了,只見潘金蓮和大姐從後邊出來,笑道:「我說後邊不見,原來你每都往前頭來了。」月娘道:「俺們剛纔送大師父出來,卜了這回龜兒卦。你早來一步,也教他與你卜卜兒。」金蓮搖頭兒道:「我是不卜他。常言:算的著命,算不著行。想前日道士說我短命哩,怎的哩?說的人心裡影影的。隨他明日街死街埋,路死路埋,倒在洋溝里就是棺材。」說畢,和月娘同歸後邊去了。正是:
\begin{quote}
萬事不由人算計,一生都是命安排。
\end{quote}
