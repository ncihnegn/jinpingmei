
\chapter{潘金蓮香腮偎玉 薛姑子佛口談經}

詩曰:
\begin{quote}
富貴如朝露,交遊似聚沙。
不如竹窗里,對捲自趺跏。
靜慮同聆偈,清神旋煮茶。
惟憂曉雞唱,塵里事如麻。
\end{quote}

話說西門慶摟抱潘金蓮,一覺睡到天明。婦人見他那話還直豎一條棍相似,便道:「達達,你饒了我罷,我來不得了。待我替你咂咂罷。」西門慶道:「怪小淫婦兒,你若咂的過了,是你造化。」這婦人真個蹲向他腰間,按著他一隻腿,用口替他吮弄那話。吮夠一個時分,精還不過,這西門慶用手按著粉項,往來只顧沒棱露腦搖撼,那話在口裡吞吐不絕。抽拽的婦人口邊白沫橫流,殘脂在莖。婦人一面問西門慶:「二十八日應二家請俺每,去不去?」西門慶道:「怎的不去!」婦人道:「我有樁事兒央你,依不依?」西門慶道:「怪小淫婦兒,你有甚事,說不是。」婦人道:「你把李大姐那皮襖拿出來與我穿了罷。明日吃了酒回來,他們都穿著皮襖,只奴沒件兒穿。」西門慶道:「有王招宣府當的皮襖,你穿就是了。」婦人道:「當的我不穿他,你與了李嬌兒去。把李嬌兒那皮襖卻與雪娥穿。你把李大姐那皮襖與了我,等我㩟上兩個大紅遍地金鶴袖,襯著白綾襖兒穿,也是與你做老婆一場,沒曾與了別人。」西門慶道:「賊小淫婦兒,單管愛小便宜兒。他那件皮襖值六十兩銀子哩,你穿在身上是會搖擺!」婦人道:「怪奴才,你與了張三、李四的老婆穿了?左右是你的老婆,替你裝門面,沒的有這些聲兒氣兒的。好不好我就不依了。」西門慶道:「你又求人又做硬兒。」婦人道:「怪硶貨,我是你房裡丫頭,在你跟前服軟?」一面說著,把那話放在粉臉上只顧偎晃,良久,又吞在口裡挑弄蛙口,一回又用舌尖抵其琴弦,攪其龜棱,然後將朱唇裹著,只顧動動的。西門慶靈犀灌頂,滿腔春意透腦,良久精來,呼:「小淫婦兒,好生裹緊著,我待過也!」言未絕,其精邈了婦人一口。婦人口口接著,都咽了。正是:
\begin{quote}
自有內事迎郎意,殷勤愛把紫簫吹。
\end{quote}

當日是安郎中擺酒,西門慶起來梳頭凈面出門。婦人還睡在被裡,便說道:「你趁閑尋尋兒出來罷。等住回,你又不得閑了。」這西門慶於是走到李瓶兒房中,奶子、丫頭又早起來頓下茶水供養。西門慶見如意兒薄施脂粉,長畫蛾眉,笑嘻嘻遞了茶,在旁邊說話兒。西門慶一面使迎春往後邊討床房裡鑰匙去,如意兒便問:「爹討來做甚麼?」西門慶道:「我要尋皮襖與你五娘穿。」如意道:「是娘的那貂鼠皮襖?」西門慶道:「就是。他要穿穿,拿與他罷。」迎春去了,就把老婆摟在懷裡,摸他奶頭,說道:「我兒,你雖然生了孩子,奶頭兒到還恁緊。」就兩個臉對臉兒親嘴咂舌頭做一處。如意兒道:「我見爹常在五娘身邊,沒見爹往別的房裡去。他老人家別的罷了,只是心多容不的人。前日爹不在,為個棒槌,好不和我大嚷了一場。多虧韓嫂兒和三娘來勸開了。落後爹來家,也沒敢和爹說。不知甚麼多嘴的人對他說,說爹要了我。他也告爹來不曾?」西門慶道:「他也告我來,你到明日替他陪個禮兒便了。他是恁行貨子,受不的人個甜棗兒就喜歡的。嘴頭子雖利害,到也沒什麼心。」如意兒道:「前日我和他嚷了,第二日爹到家,就和我說好活。說爹在他身邊偏多,『就是別的娘都讓我幾分,你凡事只有個不瞞我,我放著河水不洗船?』」西門慶道:「既是如此,大家取和些。」又許下老婆:「你每晚夕等我來這房裡睡。」如意道:「爹真個來?休哄俺每!」西門慶道:「誰哄你來!」正說著,只見迎春取鑰匙來。西門慶教開了床房門,又開櫥櫃,拿出那皮祆來抖了抖,還用包袱包了,教迎春拿到那邊房裡去。如意兒就悄悄向西門慶說:「我沒件好裙襖兒,爹趁著手兒再尋件兒與了我罷。有娘小衣裳兒,再與我一件兒。」西門慶連忙又尋出一套翠蓋緞子襖兒、黃綿綢裙子,又是一件藍潞綢綿褲兒,又是一雙妝花膝褲腿兒,與了他。老婆磕頭謝了。西門慶鎖上門,就使他送皮襖與金蓮房裡來。

金蓮才起來,在床上裹腳,只見春梅說:「如意兒送皮襖來了。」婦人便知其意,說道:「你教他進來。」問道:「爹使你來?」如意道:「是爹教我送來與娘穿。」金蓮道:「也與了你些什麼兒沒有?」如意道:「爹賞了我兩件綢絹衣裳年下穿。叫我來與娘磕頭。」於是向前磕了四個頭。婦人道:「姐姐每這般卻不好?你主子既愛你,常言:船多不礙港,車多不礙路,那好做惡人?你只不犯著我,我管你怎的?我這裡還多著個影兒哩!」如意兒道:「俺娘已是沒了,雖是後邊大娘承攬,娘在前邊還是主兒,早晚望娘抬舉。小媳婦敢欺心!那裡是葉落歸根之處?」婦人道:「你這衣服少不得還對你大娘說聲。」如意道:「小的前者也問大娘討來,大娘說:『等爹開時,拿兩件與你。』」婦人道:「既說知罷了。」這如意就出來,還到那邊房裡,西門慶已往前廳去了。如意便問迎春:「你頭裡取鑰匙去,大娘怎的說?」迎春說:「大娘問:『你爹要鑰匙做什麼?』我也沒說拿皮襖與五娘,只說我不知道。大娘沒言語。」

卻說西門慶走到廳上看設席,海鹽子弟張美、徐順、苟子孝都挑戲箱到了,李銘等四名小優兒又早來伺候,都磕頭見了。西門慶吩咐打發飯與眾人吃,吩咐李銘三個在前邊唱,左順後邊答應堂客。那日韓道國娘子王六兒沒來,打發申二姐買了兩盒禮物,坐轎子,他家進財兒跟著,也來與玉樓做生日。王經送到後邊,打發轎子出去了。不一時,門外韓大姨、孟大妗子都到了,又是傅伙計、甘伙計娘子、崔本媳婦兒段大姐並賁四娘子。西門慶正在廳上,看見夾道內玳安領著一個五短身子,穿綠緞襖兒、紅裙子,不搽胭粉,兩個密縫眼兒,一似鄭愛香模樣,便問是誰。玳安道:「是賁四嫂。」西門慶就沒言語。往後見了月娘。月娘擺茶,西門慶進來吃粥,遞與月娘鑰匙。月娘道:「你開門做什麼?」西門慶道:「潘六兒他說,明日往應二哥家吃酒沒皮襖,要李大姐那皮襖穿。」被月娘瞅了一眼,說道:「你自家把不住自家嘴頭了。他死了,嗔人分散他房裡丫頭,象你這等,就沒的話兒說了。他見放皮襖不穿,巴巴兒只要這皮襖穿。——早時他死了,他不死,你只好看一眼兒罷了。」幾句說的西門慶閉口無言。忽報劉學官來還銀子,西門慶出去陪坐,在廳上說話。只見玳安拿進帖兒說:「王招宣府送禮來了。」西門慶問:「是什麼禮?」玳安道:「是賀禮:一匹尺頭、一壇南酒、四樣下飯。」西門慶即叫王經拿眷生回帖兒謝了,賞了來人五錢銀子,打發去了。

只見李桂姐門首下轎,保兒挑四盒禮物。慌的玳安替他抱氈包,說道:「桂姨,打夾道內進去罷,廳上有劉學官坐著哩。」那桂姐即向夾道內進去,來安兒把盒子挑進月娘房裡。月娘道:「爹看見不曾?」玳安道:「爹陪著客,還不見哩。」月娘便說道:「且連盒放在明間內著。」一回客去了,西門慶進來吃飯,月娘道:「李桂姐送禮在這裡。」西門慶道:「我不知道。」月娘令小玉揭開盒兒,見一盒果餡壽糕、一盒玫瑰糖糕、兩隻燒鴨、一副豕蹄。只見桂姐從房內出來,滿頭珠翠,穿著大紅對衿襖兒,藍緞裙子,望著西門慶磕了四個頭。西門慶道:「罷了,又買這禮來做什麼?」月娘道:「剛纔桂姐對我說,怕你惱他。不乾他事,說起來都是他媽的不是:那日桂姐害頭疼來,只見這王三官領著一行人,往秦玉芝兒家去,打門首過,進來吃茶,就被人驚散了。桂姐也沒出來見他。」西門慶道:「那一遭兒沒出來見他,這一遭兒又沒出來見他,自家也說不過。論起來,我也難管你。這麗春院拿燒餅砌著門不成?到處銀錢兒都是一樣,我也不惱。」那桂姐跪在地下只顧不起來,說道:「爹惱的是。我若和他沾沾身子,就爛化了,一個毛孔兒里生一個天皰瘡。都是俺媽,空老了一片皮,乾的營生沒個主意。好的也招惹,歹的也招惹,平白叫爹惹惱。」月娘道:「你既來說開就是了,又惱怎的?」西門慶道:「你起來,我不惱你便了。」那桂姐故作嬌態,說道:「爹笑一笑兒我才起來。你不笑,我就跪一年也不起來。」潘金蓮在旁插口道:「桂姐你起來,只顧跪著他,求告他黃米頭兒,叫他張致!如今在這裡你便跪著他,明日到你家他卻跪著你,——你那時卻別要理他。」把西門慶、月娘都笑了,桂姐才起來了。只見玳安慌慌張張來報:「宋老爹、安老爹來了。」西門慶便拿衣服穿了,出去迎接。桂姐向月娘說道:「耶嚛嚛,從今後我也不要爹了,只與娘做女兒罷。」月娘道:「你的虛頭願心,說過道過罷了。前日兩遭往裡頭去,沒在那裡?」桂姐道:「天麼,天麼,可是殺人!爹何曾往我家裡?若是到我家裡,見爹一面,沾沾身子兒,就促死了!娘你錯打聽了,敢不是我那裡,是往鄭月兒家走了兩遭,請了他家小粉頭子了。我這篇是非,就是他氣不憤架的。不然,爹如何惱我?」金蓮道:「各人衣飯,他平白怎麼架你是非?」桂姐道:「五娘,你不知,俺們裡邊人,一個氣不憤一個,好不生分!」月娘接過來道:「你每裡邊與外邊差甚麼?也是一般,一個不憤一個。那一個有些時道兒,就要躧下去。」月娘擺茶與他吃,不在話下。

卻說西門慶迎接宋御史、安郎中,到廳上敘禮。每人一匹緞子、一部書,奉賀西門慶。見了桌席齊整,甚是稱謝不盡。一面分賓主坐下,吃了茶,宋御史道:「學生有一事奉瀆四泉:今有巡撫侯石泉老先生,新升太常卿,學生同兩司作東,三十日敢借尊府置杯酒奉餞,初二日就起行上京去了。未審四泉允否?」西門慶道:「老先生吩咐,敢不從命!但未知多少桌席?」宋御史道:「學生有分資在此。」即喚書吏取出布、按兩司連他共十二兩分資來,要一張大插桌、六張散桌,叫一起戲子。西門慶答應收了,就請去捲棚坐的。不一時,錢主事也到了。三員官會在一處下棋。宋御史見西門慶堂廡寬廣,院宇幽深,書畫文物極一時之盛。又見屏風前安著一座八仙捧壽的流金鼎,約數尺高,甚是做得奇巧。爐內焚著沉檀香,煙從龜鶴鹿口中吐出。只顧近前觀看,誇獎不已。問西門慶:「這副爐鼎造得好!」因向二官說:「我學生寫書與淮安劉年兄那裡,央他替我捎帶一副來,送蔡老先,還不見到。四泉不知是那裡得來的?」西門慶道:「也是淮上一個人送學生的。」說畢下棋。西門慶吩咐下邊,看了兩個桌盒細巧菜蔬果餡點心上來,一面叫生旦在上唱南曲。宋御史道:「客尚未到,主人先吃得面紅,說不通。」安郎中道:「天寒,飲一杯無礙。」宋御史又差人去邀,差人稟道:「邀了,在磚廠黃老爹那裡下棋,便來也。」一面下棋飲酒,安郎中喚戲子:「你們唱個《宜春令》奉酒。」於是生旦合聲唱一套「第一來為壓驚」。

唱未畢,忽吏進報:「蔡老爹和黃老爹來了。」宋御史忙令收了桌席,各整衣冠出來迎接。蔡九知府穿素服金帶,先令人投一「侍生蔡修」拜帖與西門慶。進廳上,安郎中道:「此是主人西門大人,見在本處作千兵,也是京中老先生門下。」那蔡知府又是作揖稱道:「久仰,久仰。」西門慶道:「容當奉拜。」敘禮畢,各寬衣服坐下。左右上了茶,各人扳話。良久,就上坐。蔡九知府居上,主位四坐。廚役割道湯飯,戲子呈遞手本,蔡九知府揀了《雙忠記》,演了兩折。酒過數巡,小優兒席前唱一套《新水令》「玉鞭驕馬出皇都」。蔡知府笑道:「松原直得多少,可謂『御史青驄馬』,三公乃『劉郎舊縈髯』。」安郎中道:「今日更不道『江州司馬青衫濕』。」言罷,眾人都笑了。西門慶又令春鴻唱了一套「金門獻罷平胡表」,把宋御史喜歡的要不的,因向西門慶道:「此子可愛。」西門慶道:「此是小價,原是揚州人。」宋御史攜著他手兒,教他遞酒,賞了他三錢銀子,磕頭謝了。正是:
\begin{quote}
窗外日光彈指過,席前花影坐間移。
一杯未盡笙歌送,階下申牌又報時。
\end{quote}

不覺日色沉西,蔡九知府見天色晚了,即令左右穿衣告辭。眾位款留不住,俱送出大門而去。隨即差了兩名吏典,把桌席羊酒尺頭抬送到新河口去訖。宋御史亦作辭西門慶,因說道:「今日且不謝,後日還要取擾。」各上轎而去。

西門慶送了回來,打發戲子,吩咐:「後日還是你們來,再唱一日。叫幾個會唱的來,宋老爹請巡撫侯爺哩。」戲子道:「小的知道了。」西門慶令攢上酒桌,使玳安:「去請溫師父來坐坐。」再叫來安兒:「去請應二爹去。」不一時,次第而至,各行禮坐下。三個小優兒在旁彈唱,把酒來斟。西門慶問伯爵:「你娘們明日都去,你叫唱的是雜耍的?」伯爵道:「哥到說得好,小人家那裡抬放?將就叫兩個唱女兒唱罷了。明日早些請眾位嫂子下降。」這裡前廳吃酒不題。

後邊,孟大姨與盂三妗子先起身去了。落後楊姑娘也要去,月娘道:「姑奶奶你再住一日兒不是,薛師父使他徒弟取了捲來,咱晚夕叫他宣捲咱們聽。」楊姑娘道:「老身實和姐姐說,要不是我也住,明日俺第二個侄兒定親事,使孩子來請我,我要瞧瞧去。」於是作辭而去。眾人吃到掌燈以後,三位伙計娘子也都作辭去了,止留下段大姐沒去,潘姥姥也往金蓮房內去了。只有大吟子、李桂姐、申二姐和三個姑子,鬱大姐和李嬌兒、孟玉樓、潘金蓮,在月娘房內坐的。忽聽前邊散了,小廝收下傢伙來。這金蓮忙抽身就往前走,到前邊悄悄立在角門首。只見西門慶扶著來安兒,打著燈,趔趄著腳兒就要往李瓶兒那邊走,看見金蓮在門首立著,拉了手進入房來。那來安兒便往上房交鐘箸。

月娘只說西門慶進來,把申二姐、李桂姐、鬱大姐都打發往李嬌兒房內去了。問來安道:「你爹來沒有?」來安道:「爹在五娘房裡,不耐煩了。」月娘聽了,心內就有些惱,因向玉樓道:「你看恁沒來頭的行貨子,我說他今日進來往你房裡去,如何三不知又摸到他屋裡去了?這兩日又浪風發起來,只在他前邊纏。」玉樓道:「姐姐,隨他纏去!這等說,恰似咱每爭他的一般。可是大師父說的笑話兒,左右這六房裡,由他串到。他爹心中所欲,你我管的他!」月娘道:「乾凈他有了話!剛纔聽見前頭散了,就慌的奔命往前走了。」因問小玉:「竈上沒人,與我把儀門拴上。後邊請三位師父來,咱每且聽他宣一回捲著。」又把李桂姐、申二姐、段大姐、鬱大姐都請了來。月娘向大妗子道:「我頭裡旋叫他使小沙彌請了《黃氏女捲》來宣,今日可可兒楊姑娘又去了。」吩咐玉簫頓下好茶。玉樓對李嬌兒說:「咱兩家輪替管茶,休要只顧累大姐姐。」於是各房裡吩咐預備茶去。

不一時,放下炕桌兒,三個姑子來到,盤膝坐在炕上。眾人俱各坐了,聽他宣捲。月娘洗手炷了香,這薛姑子展開《黃氏女捲》,高聲演說道:
\begin{quote}
蓋聞法初不滅,故歸空。道本無生,每因生而不用。由法身以垂八相,由八相以顯法身。朗朗惠燈,通開世戶;明明佛鏡,照破昏衢。百年景賴剎那間,四大幻身如泡影。每日塵勞碌碌,終朝業試忙忙。豈知一性圓明,徒逞六根貪欲。功名蓋世,無非大夢一場;富貴驚人,難免無常二字。風火散時無老少,溪山磨盡幾英雄!
\end{quote}

演說了一回,又宣念偈子,又唱幾個勸善的佛曲兒,方纔宣黃氏女怎的出身,怎的看經好善,又怎的死去轉世為男子,又怎的男女五人一時升天。

慢慢宣完,已有二更天氣。先是李嬌兒房內元宵兒拿了一道茶來,眾人吃了。落後孟玉樓房中蘭香,又拿了幾樣精製果菜、一大壺酒來,又是一大壺茶來,與大妗子、段大姐、桂姐眾人吃。月娘又教玉簫拿出四盒兒茶食餅糖之類,與三位師父點茶。李桂姐道:「三個師父宣了這一回捲,也該我唱個曲兒孝順。」月娘道:「桂姐,又起動你唱?」鬱大姐道:「等我先唱。」月娘道:「也罷,鬱大姐先唱。」申二姐道:「等姐姐唱了,我也唱個兒與娘們聽。」桂姐不肯,道:「還是我先唱。」因問月娘要聽什麼,月娘道:「你唱個『更深靜悄』罷。」當下桂姐送眾人酒,取過琵琶來,輕舒玉筍,款跨鮫綃,唱了一套。桂姐唱畢,鬱大姐才要接琵琶,早被申二姐要過去了,掛在胳膊上,先說道:「我唱個《十二月兒掛真兒》與大妗子和娘每聽罷。」於是唱道:「正月十五鬧元宵,滿把焚香天地燒……」那時大妗子害夜深困的慌,也沒等的申二姐唱完,吃了茶就先往月娘房內睡去了。須臾唱完,桂姐便歸李嬌兒房內,段大姐便往孟玉樓房內,三位師父便往孫雪娥房裡,鬱大姐、申二姐就與玉簫、小玉在那邊炕屋裡睡。月娘同大妗子在上房內睡,俱不在話下。看官聽說:古婦人懷孕,不側坐,不偃臥,不聽淫聲,不視邪色,常玩詩書金玉,故生子女端正聰慧,此胎教之法也。今月娘懷孕,不宜令僧尼宣捲,聽其死生輪迴之說。後來感得一尊古佛出世,投胎奪舍,幻化而去,不得承受家緣。蓋可惜哉!正是:
\begin{quote}
前程黑暗路途險,十二時中自著迷。
\end{quote}
