
\chapter{翟管家寄書致賻 黃真人發牒薦亡}

詞曰:
\begin{quote}
胸中千種愁,掛在斜陽樹。綠葉陰陰自得春,草滿鶯啼處。
不見凌波步,空想如簧語。門外重重疊疊山,遮不斷愁來路。
\end{quote}

話說西門慶陪吳大舅、應伯爵等飲酒中間,因問韓道國:「客夥中標船幾時起身?咱好收拾打包。」韓道國道:「昨日有人來會,也只在二十四日開船。」西門慶道:「過了二十念經,打包便了。」伯爵問道:「這遭起身,那兩位去?」西門慶道:「三個人都去。明年先打發崔大哥押一船杭州貨來,他與來保還往松江下五處,置買些布貨來賣。家中緞貨綢綿都還有哩。」伯爵道:「哥主張極妙。常言道:要的般般有,才是買賣。」說畢,已有起更時分,吳大舅起身說:「姐夫連日辛苦,俺每酒已夠了,告回,你可歇息歇息。」西門慶不肯,還留住,令小優兒奉酒唱曲,每人吃三鐘才放出門。西門慶賞小優四人六錢銀子,再三不敢接,說:「宋爺出票叫小的每來,官身如何敢受老爹重賞?」西門慶道:「雖然官差,此是我賞你,怕怎的!」四人方磕頭領去。西門慶便歸後邊歇去了。

次日早起往衙門中去,早有吳道官差了一個徒弟、兩名鋪排,來大廳上鋪設壇場,鋪設的齊齊整整。西門慶來家看見,打發徒弟鋪排齋食吃了回去。隨即令溫秀才寫帖兒,請喬大戶、吳大舅、吳二舅、花大舅、沈姨夫、孟二舅、應伯爵、謝希大、常峙節、吳舜臣許多親眷並堂客,明日念經。家中廚役落作,治辦齋供不題。

次日五更,道眾皆來,進入經壇內,明燭焚香,打動響樂,諷誦諸經,鋪排大門首掛起長幡,懸吊榜文,兩邊黃紙門對一聯,大書:
\begin{quote}
東極垂慈仙識乘晨而超登紫府;南丹赦罪凈魄受煉而逕上朱陵。
\end{quote}

大廳經壇,懸掛齋題二十字,大書:「青玄救苦、頒符告簡、五七轉經、水火煉度薦揚齋壇。」即日,黃真人穿大紅,坐牙轎,系金帶,左右圍隨,儀從暄喝,日高方到。吳道官率眾接至壇所,行禮畢,然後西門慶著素衣絰巾,拜見遞茶畢。洞案旁邊安設經筵法席,大紅銷金桌圍,妝花椅褥,二道童侍立左右。發文書之時,西門慶備金緞一匹;登壇之時,換了九陽雷巾,大紅金雲白百鶴法氅。先是表白宣畢齋意,齋官沐手上香。然後黃真人焚香凈壇,飛符召將,關發一應文書符命,啟奏三天,告盟十地。三獻禮畢,打動音樂,化財行香。西門慶與陳敬濟執手爐跟隨,排軍喝路,前後四把銷金傘、三對纓絡挑搭。行香回來,安請監齋畢,又動音樂,往李瓶兒靈前攝召引魂,朝參玉陛,旁設幾筵,聞經悟道。到了午朝,高功冠裳,步罡踏鬥,拜進朱表,遣差神將,飛下羅酆。原來黃真人年約三旬,儀表非常,妝束起來,午朝拜表,儼然就是個活神仙。但見:
\begin{quote}
星冠攢玉葉,鶴氅縷金霞。神清似長江皓月,貌古如太華喬松。踏罡朱履進丹霄,步虛琅函浮瑞氣。長髯廣頰,修行到無漏之天;皓齒明眸,佩籙掌五雷之令。三更步月鸞聲遠,萬里乘雲鶴背高。就是都仙太史臨凡世,廣惠真人降下方。
\end{quote}

拜了表文,吳道官當壇頒生天寶籙神虎玉札。行畢午香,捲棚內擺齋。黃真人前,大桌面定勝;吳道官等,稍加差小;其餘散眾,俱平頭桌席。黃真人、吳道官皆襯緞尺頭、四對披花、四匹絲綢,散眾各布一匹。桌面俱令人抬送廟中,散眾各有手下徒弟收入箱中,不必細說。

吃畢午齋,都往花園內遊玩散食去了。一面收下家火,從新擺上齋饌,請吳大舅等眾親朋伙計來吃。正吃之間,忽報:「東京翟爺那裡差人下書。」西門慶即出廳上,請來人進來。只見是府前承差乾辦,青衣窄褲,萬字頭巾,乾黃靴,全副弓箭,向前施禮。西門慶答禮相還。那人向身邊取出書來遞上,又是一封折賻儀銀十兩。問來人上姓,那人道:「小人姓王名玉,蒙翟爺差遣,送此書來。不知老爹這邊有喪事,安老爹書到才知。」西門慶問道:「你安老爹書幾時到的?」那人說:「十月才到京。因催皇木一年已滿,升都水司郎中。如今又奉敕修理河道,直到工完回京。」西門慶問了一遍,即令來保廂房中管待齋飯,吩咐明日來討回書。那人問:「韓老爹在那裡住?宅內捎信在此。小的見了,還要趕往東平府下書去。」西門慶即喚出韓道國來見那人,陪吃齋飯畢,同往家中去了。

西門慶拆看書中之意,於是乘著喜歡,將書拿到捲棚內教溫秀才看。說:「你照此修一封回書答他,就捎寄十方縐紗汗巾、十方綾汗巾、十副揀金挑牙、十個烏金酒盃作回奉之禮。他明日就來取回書。」溫秀才接過書來觀看,其書曰:
\begin{quote}
寓京都眷生翟謙頓首,書奉即擢大錦堂西門四泉親家大人門下:自京邸話別之後,未得從容相敘,心甚歉然。其領教之意,生已於家老爺前悉陳之矣。邇者,安鳳山書到,方知老親家有鼓盆之嘆,但恨不能一弔為悵,奈何,奈何!伏望以禮節哀可也。外具賻儀,少表微忱,希管納。又久仰貴任榮修德政,舉民有五絝之歌,境內有三留之譽,今歲考績,必有甄升。昨日神運都功,兩次工上,生已對老爺說了,安上親家名字。工完題奏,必有恩典,親家必有掌刑之喜。夏大人年終類本,必轉京堂指揮列銜矣。謹此預報,伏惟高照,不宣。
附雲:此書可自省覽,不可使聞之於渠。謹密,謹密!
又雲:楊老爺前月二十九日卒於獄。
冬上浣具
\end{quote}

溫秀才看畢,才待袖,早被應伯爵取過來,觀看了一遍,還付與溫秀才收了。說道:「老先生把回書千萬加意做好些。翟公府中人才極多,休要教他笑話。」溫秀才道:「貂不足,狗尾續。學生匪才,焉能在班門中弄大斧!不過乎塞責而已。」西門慶道:「溫老先他自有個主意,你這狗才曉的甚麼!」須臾,吃罷午齋,西門慶吩咐來興兒打發齋饌,送各親眷街鄰。又使玳安回院中李桂姐、吳銀兒、鄭愛月兒、韓釧兒、洪四兒、齊香兒六家香儀人情禮去。每家回答一匹大布、一兩銀子。

後晌,就叫李銘、吳惠、鄭奉三個小優兒來伺候。良久,道眾升壇發擂,上朝拜懺觀燈,解壇送聖。天色漸晚。比及設了醮,就有起更天氣。門外花大舅被西門慶留下不去了,喬大戶、沈姨夫、孟二舅告辭回家。止有吳大舅、二舅、應伯爵、謝希大、溫秀才、常峙節並眾伙計在此,晚夕觀看水火練度。就在大廳棚內搭高座,扎彩橋,安設水池火沼,放擺斛食。李瓶兒靈位另有幾筵幃幕,供獻齊整。旁邊一首魂幡、一首紅幡、一首黃幡,上書「制魔保舉,受煉南宮」。先是道眾音樂,兩邊列座,持節捧盂劍,四個道童侍立兩邊。黃真人頭戴黃金降魔冠,身披絳綃雲霞衣,登高座,口中念念有詞。宣偈雲:
\begin{quote}
太乙慈尊降駕來,夜壑幽關次第開。
童子雙雙前引導,死魂受煉步雲階。
\end{quote}

宣偈畢,又熏沐焚香,念曰:「伏以玄皇闡教,廣開度於冥途;正一垂科,俾煉形而升舉。恩沾幽爽,澤被飢噓。謹運真香,志誠上請東極大慈仁者太乙救苦天尊、十方救苦諸真人聖眾,仗此真香,來臨法會。切以人處塵凡,日縈俗務,不知有死,惟欲貪生。鮮能種於善根,多隨入於惡趣,昏迷弗省,恣欲貪嗔。將謂自己長存,豈信無常易到!一朝傾逝,萬事皆空。業障纏身,冥司受苦。今奉道伏為亡過室人李氏靈魂,一棄塵緣,久淪長夜。若非薦拔於愆辜,必致難逃於苦報。恭惟天尊秉好生之仁,救尋聲之苦。灑甘露而普滋群類,放瑞光而遍燭昏衢。命三官寬考較之條,詔十殿閣推研之筆。開囚釋禁,宥過解冤。各隨符使,盡出幽關。咸令登火池之沼,悉蕩滌黃華之形。凡得更生,俱歸道岸。茲焚靈寶煉形真符,謹當宣奏:
\begin{quote}
太微回黃旗,無英命靈幡,攝召長夜府,開度受生魂。」
\end{quote}

道眾先將魂幡安於水池內,焚結靈符,換紅幡;次於火沼內焚鬱儀符,換黃幡。高功念:「天一生水,地二生火,水火交煉,乃成真形。」煉度畢,請神主冠帔步金橋,朝參玉陛,皈依三寶,朝玉清,眾舉《五供養》。舉畢,高功曰:「既受三皈,當宣九戒。」九戒畢,道眾舉音樂,宣念符命並《十類孤魂》。煉度已畢,黃真人下高座,道眾音樂送至門外,化財焚燒箱庫。

回來,齋功圓滿,道眾都換了冠服,鋪排收捲道像。西門慶又早大廳上畫燭齊明,酒筵羅列。三個小優彈唱,眾親友都在堂前。西門慶先與黃真人把盞,左右捧著一匹天青雲鶴金緞、一匹色緞、十兩白銀,叩首下拜道:「亡室今日賴我師經功救拔,得遂超生,均感不淺,微禮聊表寸心。」黃真人道:「小道謬忝冠裳,濫膺玄教,有何德以達人天?皆賴大人一誠感格,而尊夫人已駕景朝元矣。此禮若受,實為赧顏。」西門慶道:「此禮甚薄,有褻真人,伏乞笑納!」黃真人方令小童收了。西門慶遞了真人酒,又與吳道官把盞,乃一匹金緞、五兩白銀,又是十兩經資。吳道官只受經資,餘者不肯受,說:「小道素蒙厚愛,自恁效勞誦經,追拔夫人往生仙界,以盡其心。受此經資尚為不可,又豈敢當此盛禮乎!」西門慶道:「師父差矣。真人掌壇,其一應文簡法事,皆乃師父費心。此禮當與師父酬勞,何為不可?」吳道官不得已,方領下,再三致謝。西門慶與道眾遞酒已畢,然後吳大舅、應伯爵等上來與西門慶散福遞酒。吳大舅把盞,伯爵執壺,謝希大捧菜,一齊跪下。伯爵道:「嫂子今日做此好事,幸請得真人在此,又是吳師父費心,嫂子自得好處。此雖賴真人追薦之力,實是哥的虔心,嫂子的造化。」於是滿斟一杯送與西門慶。西門慶道:「多蒙列位連日勞神,言謝不盡。」說畢,一飲而盡。伯爵又斟一盞,說:「哥,吃個雙杯,不要吃單杯。」謝希大慌忙遞一箸菜來吃了。西門慶回敬眾人畢,安席坐下。小優彈唱起來,廚役上割道。當夜在席前猜拳行令,品竹彈絲,直吃到二更時分,西門慶已帶半酣,眾人方作辭起身而去。西門慶進來賞小優兒三錢銀子,往後邊去了。正是:
\begin{quote}
人生有酒須當醉,一滴何曾到九泉。
\end{quote}
