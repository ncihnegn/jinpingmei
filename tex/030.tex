
\chapter{蔡太師擅恩錫爵 西門慶生子加官}

詞曰:
\begin{quote}
十千日日索花奴,白馬驕駝馮子都。今年新拜執金吾。
侵幙露桃初結子,妒花嬌鳥忽嗛雛。閨中姊妹半愁娛。
\end{quote}

話說西門慶與潘金蓮兩個洗畢澡,就睡在房中。春梅坐在穿廊下一張涼椅兒上納鞋,只見琴童兒在角門首探頭舒腦的觀看。春梅問道:「你有甚話說?」那琴童見秋菊頂著石頭跪在院內,只顧用手往來指。春梅罵道:「怪囚根子!有甚話,說就是了,指手畫腳怎的?」那琴童笑了半日,方纔說:「看墳的張安,在外邊等爹說話哩。」春梅道:「賊囚根子!張安就是了,何必大驚小怪,見鬼也似!悄悄兒的,爹和娘睡著了。驚醒他,你就是死。你且叫張安在外邊等等兒。」琴童兒走出來外邊,約等夠半日,又走來角門首踅探,問道:「爹起來了不曾?」春梅道:「怪囚!失張冒勢,唬我一跳,有要沒緊,兩頭遊魂哩!」琴童道:「張安等爹說了話,還要趕出門去,怕天晚了。」春梅道:「爹娘正睡的甜甜兒的,誰敢攪擾他,你教張安且等著去,十分晚了,教他明日去罷。」

正說著,不想西門慶在房裡聽見,便叫春梅進房,問誰說話。春梅道:「琴童說墳上張安兒在外邊,見爹說話哩。」西門慶道:「拿衣我穿,等我起去。」春梅一面打發西門慶穿衣裳,金蓮便問:「張安來說甚麼話?」西門慶道:「張安前日來說,咱家墳隔壁趙寡婦家莊子兒連地要賣,價銀三百兩。我只還他二百五十兩銀子,教張安和他講去。裡面一眼井,四個井圈打水。若買成這莊子,展開合為一處,裡面蓋三間捲棚,三間廳房,疊山子花園、井亭、射箭廳、打毬場,耍子去處,破使幾兩銀子收拾也罷。」婦人道:「也罷,咱買了罷。明日你娘每上墳,到那裡好遊玩耍子。」說畢,西門慶往前邊和張安說話去了。

金蓮起來,向鏡臺前重勻粉臉,再整雲鬟。出來院內要打秋菊。那春梅旋去外邊叫了琴童兒來弔板子。金蓮問道:「叫你拿酒,你怎的拿冷酒與爹吃?原來你家沒大了,說著,你還釘嘴鐵舌兒的!」喝聲:「叫琴童兒與我老實打與這奴才二十板子!」那琴童才打到十板子上,多虧了李瓶兒笑嘻嘻走過來勸住了,饒了他十板。金蓮教與李瓶兒磕了頭,放他起來,廚下去了。李瓶兒道:「老潘領了個十五歲的丫頭,後邊二姐姐買了房裡使喚,要七兩五錢銀子。請你過去瞧瞧。」金蓮遂與李瓶兒一同後邊去了。李嬌兒果問西門慶用七兩銀子買了,改名夏花兒,房中使喚,不在話下。

單表來保同吳主管押送生辰擔,正值炎蒸天氣,路上十分難行,免不得飢餐渴飲。有日到了東京萬壽門外,尋客店安下。到次日,齎台馱箱禮物,逕到天漢橋蔡太師府門前伺候。來保教吳主管押著禮物,他穿上青衣,逕向守門官吏唱了個喏。那守門官吏問道:「你是那裡來的?」來保道:「我是山東清河縣西門員外家人,來與老爺進獻生辰禮物。」官吏罵道:「賊少死野囚軍!你那裡便興你東門員外、西門員外?俺老爺當今一人之下,萬人之上,不論三台八位,不論公子王孫,誰敢在老爺府前這等稱呼?趁早靠後!」內中有認的來保的,便安撫來保說道:「此是新參的守門官吏,才不多幾日,他不認的你,休怪。你要稟見老爺,等我請出翟大叔來。」這來保便向袖中取出一包銀子,重一兩,遞與那人。那人道:「我到不消。你再添一分,與那兩個官吏,休和他一般見識。」來保連忙拿出三包銀子來,每人一兩,都打發了。那官吏才有些笑容兒,說道:「你既是清河縣來的,且略等候,等我領你先見翟管家。老爺才從上清寶霄宮進了香回來,書房內睡。」良久,請將翟管家出來,穿著涼鞋凈襪,青絲絹道袍。來保見了,忙磕下頭去。翟管家答禮相還,說道:「前者累你。你來與老爺進生辰擔禮來了?」來保先遞上一封揭帖,腳下人捧著一對南京尺頭,三十兩白金,說道:「家主西門慶,多上覆翟爹,無物表情,這些薄禮,與翟爹賞人。前者鹽客王四之事,多蒙翟爹費心。」翟謙道:「此禮我不當受。罷,罷,我且收下。」來保又遞上太師壽禮帖兒,看了,還付與來保,吩咐把禮抬進來,到二門裡首伺候。原來二門西首有三間倒座,來往雜人都在那裡待茶。須臾,一個小童拿了兩盞茶來,與來保、吳主管吃了。

少頃,太師出廳。翟謙先稟知太師,然後令來保、吳主管進見,跪於階下。翟謙先把壽禮揭帖呈遞與太師觀看,來保、吳主管各抬獻禮物。但見:
\begin{quote}
黃烘烘金壺玉盞,白晃晃減靸仙人。錦繡蟒衣,五彩奪目;南京紵緞,金碧交輝。湯羊美酒,盡貼封皮;異果時新,高堆盤盒。
\end{quote}
如何不喜,便道:「這禮物決不好受的,你還將回去。」慌的來保等在下叩頭,說道:「小的主人西門慶,沒甚孝意,些小微物,進獻老爺賞人。」太師道:「既是如此,令左右收了。」旁邊祗應人等,把禮物盡行收下去。太師又道:「前日那滄州客人王四等之事,我已差人下書,與你巡撫侯爺說了。可見了分上不曾?」來保道:「蒙老爺天恩,書到,眾鹽客就都放出來了。」太師又向來保說道:「累次承你主人費心,無物可伸,如何是好?你主人身上可有甚官役?」來保道:「小人的主人一介鄉民,有何官役?」太師道:「既無官役,昨日朝廷欽賜了我幾張空名告身札付,我安你主人在你那山東提刑所,做個理刑副千戶,頂補千戶賀金的員缺,好不好?」來保慌的叩頭謝道:「蒙老爺莫大之恩,小的家主舉家粉首碎身,莫能報答!」於是喚堂候官抬書案過來,即時簽押了一道空名告身札付,把西門慶名字填註上面,列銜金吾衛衣左所副千戶、山東等處提刑所理刑。又向來保道:「你二人替我進獻生辰禮物,多有辛苦。」因問:「後邊跪的是你甚麼人?」來保才待說是伙計,那吳主管向前道:「小的是西門慶舅子,名喚吳典恩。」太師道:「你既是西門慶舅子,我觀你倒好個儀錶。」喚堂候官取過一張札付:「我安你在本處清河縣做個驛丞,倒也去的。」那吳典恩慌的磕頭如搗蒜。又取過一張札付來,把來保名字填寫山東鄆王府,做了一名校尉。俱磕頭謝了,領了札付。吩咐明日早晨,吏、兵二部掛號,討勘合,限日上任應役。又吩咐翟謙西廂房管待酒飯,討十兩銀子與他二人做路費,不在話下。

看官聽說:那時徽宗,天下失政,姦臣當道,讒佞盈朝,高、楊、童、蔡四個姦黨,在朝中賣官鬻獄,賄賂公行,懸秤陞官,指方補價。夤緣鑽刺者,驟升美任;賢能廉直者,經歲不除。以致風俗頹敗,贓官污吏遍滿天下,役煩賦興,民窮盜起,天下騷然。不因姦臣居台輔,合是中原血染人。

當下翟謙把來保、吳主管邀到廂房管待,大盤大碗飽餐了一頓。翟謙向來保說:「我有一件事,央及你爹替我處處,未知你爹肯應承否?」來保道:「翟爹說那裡話!蒙你老人家這等老爺前扶持看顧,不揀甚事,但肯吩咐,無不奉命。」翟謙道:「不瞞你說,我答應老爺,每日止賤荊一人。我年將四十,常有疾病,身邊通無所出。央及你爹,你那貴處有好人才女子,不拘十五六上下,替我尋一個送來。該多少財禮,我一一奉過去。」說畢,隨將一封人事並回書付與來保,又送二人五兩盤纏。來保再三不肯受,說道:「剛纔老爺上已賞過了。翟爹還收回去。」翟謙道:「那是老爺的,此是我的,不必推辭。」當下吃畢酒飯,翟謙道:「如今我這裡替你差個辦事官,同你到下處,明早好往吏、兵二部掛號,就領了勘合,好起身。省的你明日又費往返了。我吩咐了去,部里不敢遲滯你文書。」一面喚了個辦事官,名喚李中友:「你與二位明日同到部里掛了號,討勘合來回我話。」那員官與來保、吳典恩作辭,出的府門,來到天漢橋街上白酒店內會話。來保管待酒飯,又與了李中友三兩銀子,約定明日絕早先到吏部,然後到兵部,都掛號討了勘合。聞得是太師老爺府里,誰敢遲滯,顛倒奉行。金吾衛太尉朱勔,即時使印,簽了票帖,行下頭司,把來保填註在本處山東鄆王府當差。又拿了個拜帖,回翟管家。不消兩日,把事情幹得完備。有日雇頭口起身,星夜回清河縣來報喜。正是:
\begin{quote}
富貴必因姦巧得,功名全仗鄧通成。
\end{quote}

且說一日三伏天氣,西門慶在家中聚景堂上大卷棚內,賞玩荷花,避暑飲酒。吳月娘與西門慶俱上坐,諸妾與大姐都兩邊列坐,春梅、迎春、玉簫、蘭香,一般兒四個家樂在旁彈唱。怎見的當日酒席?但見:
\begin{quote}
盆栽綠草,瓶插紅花。水晶簾捲蝦須,雲母屏開孔雀。盤堆麟脯,佳人笑捧紫霞觴;盆浸冰桃,美女高擎碧玉斝。食烹異品,果獻時新。弦管謳歌,奏一派聲清韻美;綺羅珠翠,擺兩行舞女歌兒。當筵象板撒紅牙,遍體舞裙鋪錦繡。消遣壺中閑日月,遨遊身外醉乾坤。
\end{quote}

妻妾正飲酒中間,坐間不見了李瓶兒。月娘向繡春說道:「你娘往屋裡做甚麼哩?」繡春道:「我娘害肚裡疼,歪著哩。」月娘道:「還不快對他說去,休要歪著,來這裡聽一回唱罷。」西門慶便問月娘:「怎的?」月娘道:「李大姐忽然害肚裡疼,房裡躺著哩。我使小丫頭請他去了。」因向玉樓道:「李大姐七八臨月,只怕攪撒了。」潘金蓮道:「大姐姐,他那裡是這個月?約他是八月里孩子,還早哩!」西門慶道:「既是早哩,使丫頭請你六娘來聽唱。」不一時,只見李瓶兒來到。月娘道:「只怕你掉了風冷氣,你吃上鐘熱酒,管情就好了。」不一時,各人面前斟滿了酒。西門慶吩咐春梅:「你每唱個『人皆畏夏日』我聽。」那春梅等四個方纔箏排雁柱,阮跨鮫綃,啟朱唇,露皓齒,唱「人皆畏夏日」。那李瓶兒在酒席上,只是把眉頭忔㤘著,也沒等的唱完,就回房中去了。月娘聽了詞曲,耽著心,使小玉房中瞧去。回來報說:「六娘害肚裡疼,在炕上打滾哩。」慌了月娘道:「我說是時候,這六姐還強說早哩。還不喚小廝快請老娘去!」西門慶即令平安兒:「風跑!快請蔡老娘去!」於是連酒也吃不成,都來李瓶兒房中問他。

月娘問道:「李大姐,你心裡覺的怎的?」李瓶兒回道:「大娘,我只心口連小肚子,往下鱉墜著疼。」月娘道:「你起來,休要睡著,只怕滾壞了胎。老娘請去了,便來也。」少頃,漸漸李瓶兒疼的緊了。月娘又問:「使了誰請老娘去了?這咱還不見來?」玳安道:「爹使來安去了。」月娘罵道:「這囚根子,你還不快迎迎去!平白沒算計,使那小奴才去,有緊沒慢的。」西門慶叫玳安快騎了騾子趕去。月娘道:「一個風火事,還象尋常慢條斯禮兒的。」那潘金蓮見李瓶兒待養孩子,心中未免有幾分氣。在房裡看了一回,把孟玉樓拉出來,兩個站在西梢間檐柱兒底下那裡歇涼,一處說話。說道:「耶嚛嚛!緊著熱剌剌的擠了一屋子的人,也不是養孩子,都看著下象膽哩。」良久,只見蔡老娘進門,望眾人道:「那位是主家奶奶?」李嬌兒指著月娘道:「這位大娘哩。」那蔡老娘倒身磕頭。月娘道:「姥姥,生受你。怎的這咱才來?請看這位娘子,敢待生養也?」蔡老娘向床前摸了摸李瓶兒身上,說道:「是時候了。」問:「大娘預備下綳接、草紙不曾?」月娘道:「有。」便叫小玉:「往我房中快取去!」

且說玉樓見老娘進門,便向金蓮說:「蔡老娘來了,咱不往屋裡看看去?」那金蓮一面不是一面,說道:「你要看,你去。我是不看他。他是有孩子的姐姐,又有時運,人怎的不看他?頭裡我自不是,說了句話兒『只怕是八月里的』,叫大姐姐白搶白相。我想起來好沒來由,倒惱了我這半日。」玉樓道:「我也只說他是六月里孩子。」金蓮道:「這回連你也韶刀了!我和你恁算:他從去年八月來,又不是黃花女兒,當年懷,入門養。一個婚後老婆,漢子不知見過了多少,也一兩個月才生胎,就認做是咱家孩子?我說差了?若是八月里孩兒,還有咱家些影兒;若是六月的,踩小板凳兒糊險神道——還差著一帽頭子哩!失迷了家鄉,那裡尋犢兒去?」正說著,只見小玉抱著草紙、綳接並小褥子兒來。孟玉樓道:「此是大姐姐自預備下他早晚用的,今日且借來應急兒。」金蓮道:「一個是大老婆,一個是小老婆,明日兩個對養,十分養不出來,零碎出來也罷。俺每是買了個母雞不下蛋,莫不吃了我不成!」又道:「仰著合著,沒的狗咬尿胞虛歡喜?」玉樓道:「五姐是甚麼話!」以後見他說話不防頭腦,只低著頭弄裙帶子,並不作聲應答他。少頃,只見孫雪娥聽見李瓶兒養孩子,從後邊慌慌張張走來觀看,不防黑影里被台基險些不曾絆了一交。金蓮看見,教玉樓:「你看獻勤的小婦奴才!你慢慢走,慌怎的?搶命哩!黑影子絆倒了,磕了牙也是錢!養下孩子來,明日賞你這小婦奴才一個紗帽戴!」良久,只聽房裡「呱」的一聲養下來了。蔡老娘道:「對當家的老爹說,討喜錢,分娩了一位哥兒。」吳月娘報與西門慶。西門慶慌忙洗手,天地祖先位下滿爐降香,告許一百二十分清醮,要祈母子平安,臨盆有慶,坐草無虞。這潘金蓮聽見生下孩子來了,合家歡喜,亂成一塊,越發怒氣,逕自去到房裡,自閉門戶,向床上哭去了。時宣和四年戊申六月念三日也。正是:
\begin{quote}
不如意事常八九,可與人言無二三。
\end{quote}

蔡老娘收拾孩子,咬去臍帶,埋畢衣胞,熬了些定心湯,打發李瓶兒吃了,安頓孩兒停當。月娘讓老娘後邊管待酒飯。臨去,西門慶與了他五兩一錠銀子,許洗三朝來,還與他一匹緞子。這蔡老娘千恩萬謝出門。

當日,西門慶進房去,見一個滿抱的孩子,生的甚是白凈,心中十分歡喜。合家無不歡悅。晚夕,就在李瓶兒房中歇了,不住來看孩兒。次日,巴天不明起來,拿十副方盒,使小廝各親戚鄰友處,分投送喜面。應伯爵、謝希大聽見西門慶生了子,送喜面來,慌的兩步做一步走來賀喜。西門慶留他捲棚內吃面。剛打發去了,正要使小廝叫媒人來尋養娘,忽有薛嫂兒領了個奶子來。原是小人家媳婦兒,年三十歲,新近丟了孩兒,不上一個月。男子漢當軍,過不的,恐出征去無人養贍,只要六兩銀子賣他。月娘見他生的乾凈,對西門慶說,兌了六兩銀子留下,取名如意兒,教他早晚看奶哥兒。又把老馮叫來暗房中使喚,每月與他五錢銀子,管顧他衣服。

正熱鬧一日,忽有平安報:「來保、吳主管在東京回還,見在門首下頭口。」不一時,二人進來,見了西門慶報喜。西門慶問:「喜從何來?」二人悉把到東京見蔡太師進禮一節,從頭至尾說道:「老爺見了禮物甚喜,說道:『我累次受你主人之禮,無可補報。』朝廷欽賞了他幾張空名誥身札付,就與了爹一張,把爹名姓填註在金吾衛副千戶之職,就委差在本處提刑所理刑,頂補賀老爺員缺。把小的做了鐵鈴衛校尉,填註鄆王府當差。吳主管升做本縣驛丞。」於是把一樣三張印信札付,並吏、兵二部勘合,並誥身都取出來,放在桌上與西門慶觀看。西門慶看見上面銜著許多印信,朝廷欽依事例,果然他是副千戶之職,不覺歡從額角眉尖出,喜向腮邊笑臉生。便把朝廷明降,拿到後邊與吳月娘眾人觀看,說:「太師老爺抬舉我,升我做金吾衛副千戶,居五品大夫之職。你頂受五花官誥,做了夫人。又把吳主管攜帶做了驛丞,來保做了鄆王府校尉。吳神仙相我不少紗帽戴,有平地登雲之喜,今日果然。不上半月,兩椿喜事都應驗了。」又對月娘說:「李大姐養的這孩子甚是腳硬,到三日洗了三,就起名叫做官哥兒罷。」來保進來,與月娘眾人磕頭,說了回話。吩咐明日早把文書下到提刑所衙門裡,與夏提刑知會了。吳主管明日早下文書到本縣,作辭西門慶回家去了。

到次日,洗三畢,眾親鄰朋友一概都知西門慶第六個娘子新添了娃兒,未過三日,就有如此美事,官祿臨門,平地做了千戶之職。誰人不來趨附?送禮慶賀,人來人去,一日不斷頭。常言:時來誰不來?時不來誰來!正是:
\begin{quote}
時來頑鐵有光輝,運退真金無顏色。
\end{quote}
