
\chapter{陳敬濟被陷嚴州府 吳月娘大鬧授官廳}

詩曰:
\begin{quote}
猛虎馮其威,往往遭急縛。
雷吼徒暴哮,枝撐已在腳。
忽看皮寢處,無復晴閃爍。
人有甚於斯,盡以勸元惡。
\end{quote}

話說李衙內打了玉簪兒一頓,即時叫陶媽媽來領出,賣了八兩銀子,另買了個十八歲使女,名喚滿堂兒上竈,不在話下。

卻表陳敬濟,自從西門大姐來家,交還了許多床帳妝奩,箱籠傢伙,三日一場嚷,五日一場鬧,問他娘張氏要本錢做買賣。他母舅張團練,來問他母親借了五十兩銀子,復謀管事。被他吃醉了,往張舅門上罵嚷。他張舅受氣不過,另問別處借了銀子,乾成管事,還把銀子交還交來。他母親張氏,著了一場重氣,染病在身,日逐臥床不起,終日服藥,請醫調治。吃他逆毆不過,只得兌出三百兩銀子與他,叫陳定在家門首,打開兩間房子開布鋪,做買賣。敬濟便逐日結交朋友陸三郎、楊大郎狐朋狗黨,在鋪中彈琵琶,抹骨牌,打雙陸,吃半夜酒,看看把本錢弄下去了。陳定對張氏說他每日飲酒花費。張氏聽信陳定言語,便不肯托他。敬濟反說陳定染布去,克落了錢,把陳定兩口兒攆出來外邊居住,卻搭了楊大郎做伙計。這楊大郎名喚楊光彥,綽號為鐵指甲,專一糶風賣雨,架謊鑿空。他許人話,如捉影捕風,騙人財,似探囊取物。這敬濟問娘又要出二百兩銀子來添上,共湊了五百兩銀子,信著他往臨清販布去。

這楊大郎到家收拾行李,跟著敬濟從家中起身,前往臨清馬頭上尋缺貨去。到了臨清,這臨清閘上是個熱鬧繁華大馬頭去處,商賈往來之所,車輛輻湊之地,有三十二條花柳巷,七十二座管弦樓。這敬濟終是年小後生,被這楊大郎領著游娼樓,登酒店,貨物到販得不多。因走在一娼樓,見了一個粉頭,名喚馮金寶,生的風流俏麗,色藝雙全。問青春多少,鴇子說:「姐兒是老身親生之女,止是他一人掙錢養活。今年青春才交二九一十八歲。」敬濟一見,心目蕩然,與了鴇子五兩銀子房金,一連和他歇了幾夜。楊大郎見他愛這粉頭,留連不舍,在旁花言說念,就要娶他家去。鴇子開口要銀一百二十兩,講到一百兩上,兌了銀子,娶了來家。一路上用轎抬著,楊大郎和敬濟都騎馬,押著貨物車走,一路揚鞭走馬,那樣歡喜。正是:
\begin{quote}
多情燕子樓,馬道空迴首。
載得武陵春,陪作鸞凰友。
\end{quote}

張氏見敬濟貨到販得不多,把本錢到娶了一個唱的來家,又著了口重氣,嗚呼哀哉,斷氣身亡。這敬濟不免買棺裝殮,念經做七,停放了一七光景,發送出門,祖塋合葬。他母舅張團練看他娘面上,亦不和他一般見識。這敬濟墳上覆墓回來,把他娘正房三間,中間供養靈位,那兩間收拾與馮金寶住,大姐到住著耳房。又替馮金寶買了丫頭重喜兒伏侍。門前楊大郎開著鋪子,家裡大酒大肉買與唱的吃。每日只和唱的睡,把大姐丟著不去揪採。

一日,打聽孟玉樓嫁了李知縣兒子李衙內,帶過許多東西去。三年任滿,李知縣升在浙江嚴州府做了通判,領憑起身,打水路赴任去了。這陳敬濟因想起昔日在花園中拾了孟玉樓那根簪子,就要把這根簪子做個證兒,趕上嚴州去。只說玉樓先與他有了姦,與了他這根簪子,不合又帶了許多東西,嫁了李衙內,都是昔日楊戩寄放金銀箱籠,應沒官之物。「那李通判一個文官,多大湯水!聽見這個利害口聲,不怕不叫他兒子雙手把老婆奉與我。我那時娶將來家,與馮金寶做一對兒,落得好受用。」正是:
\begin{quote}
計就月中擒月兔,謀成日里捉金烏。
\end{quote}
敬濟不來到好,此一來,正是:
\begin{quote}
失曉人家逢五道,溟泠餓鬼撞鐘馗。
\end{quote}
有詩為證:
\begin{quote}
趕到嚴州訪玉人,人心難忖似石沉。
侯門一旦深似海,從此蕭郎落陷坑。
\end{quote}

一日,陳敬濟打點他娘箱中,尋出一千兩金銀,留下一百兩與馮金寶家中盤纏,把陳定復叫進來看家,並門前鋪子發賣零碎布匹。他與楊大郎又帶了家人陳安,押著九百兩銀子,從八月中秋起身,前往湖州販了半船絲綿綢絹,來到清江浦馬頭上,灣泊住了船隻,投在個店主人陳二店內。交陳二殺雞取酒,與楊大郎共飲。飲酒中間,和楊大郎說:「伙計,你暫且看守船上貨物,在二郎店內略住數日。等我和陳安拿些人事禮物,往浙江嚴州府,看看家姐嫁在府中。多不上五日,少只三日就來。」楊大郎道:「哥去只顧去。兄弟情願店中等候。哥到日,一同起身。」

這陳敬濟千不合萬不合和陳安身邊帶了些銀兩、人事禮物,有日取路徑到嚴州府。進入城內,投在寺中安下。打聽李通判到任一個月,家小船隻才到三日。這陳敬濟不敢怠慢,買了四盤禮物,四匹紵絲尺頭,陳安押著。他便揀選衣帽齊整,眉目光鮮,徑到府衙前,與門吏作揖道:「煩報一聲,說我是通判老爹衙內新娶娘子的親,孟二舅來探望。」這門吏聽了,不敢怠慢,隨即稟報進去。衙內正在書房中看書,聽見是婦人兄弟,令左右先把禮物抬進來,一面忙整衣冠,道:「有請。」把陳敬濟請入府衙廳上敘禮,分賓主坐下,說道:「前日做親之時,怎的不會二舅?」敬濟道:「在下因在川廣販貨,一年方回。不知家姐嫁與府上,有失親近。今日敬備薄禮,來看看家姐。」李衙內道:「一向不知,失禮,恕罪,恕罪。」須臾,茶湯已罷,衙內令左右:「把禮貼並禮物取進去,對你娘說,二舅來了。」孟玉樓正在房中坐的,只聽小門子進來,報說:「孟二舅來了。」玉樓道:「再有那個舅舅,莫不是我二哥孟銳來家了,千山萬水來看我?」只見伴當拿進禮物和貼兒來,上面寫著:「眷生孟銳」,就知是他兄弟,一面道:「有請。」令蘭香收拾後堂乾凈。

玉樓裝點打扮,俟候出見。只見衙內讓直來,玉樓在簾內觀看,可霎作怪,不是他兄弟,卻是陳姐夫。「他來做甚麼?等我出去,見他怎的說話?常言,親不親,故鄉人;美不美,鄉中水。雖然不是我兄弟,也是我女婿人家。」一面整妝出來拜見。那敬濟說道:「一向不知姐姐嫁在這裡,沒曾看得……」才說得這句,不想門子來請衙內,外邊有客來了。這衙內分付玉樓款待二舅,就出去待客去了。玉樓見敬濟磕下頭去,連忙還禮,說道:「姐夫免禮,那陣風兒刮你到此?」敘畢禮數,上坐,叫蘭香看茶出來。吃了茶,彼此敘了些家常話兒,玉樓因問:「大姐好麼?」敬濟就把從前西門慶家中出來,並討箱籠的一節話告訴玉樓。玉樓又把清明節上墳,在永福寺遇見春梅,在金蓮墳上燒紙的話告訴他。又說:「我那時在家中,也常勸你大娘,疼女兒就疼女婿,親姐夫,不曾養活了外人。他聽信小人言語,把姐夫打發出來。落後姐夫討箱子,我就不知道。」敬濟道:「不瞞你老人家說,我與六姐相交,誰人不知?生生吃他聽奴才言語,把他打發出去,才吃武鬆殺了。他若在家,那武鬆有七個頭八個膽,敢往你家來殺他?我這仇恨,結的有海來深。六姐死在陰司里,也不饒他。」玉樓道:「姐夫也罷,丟開手的事,自古冤讎只可解,不可結。」

說話中間,丫鬟放下桌兒,擺下酒來,杯盤餚品,堆滿春台。玉樓斟上一杯酒,雙手遞與敬濟說:「姐夫遠路風塵,無可破費,且請一杯兒水酒。」這敬濟用手接了,唱了喏,也斟一杯回奉婦人,敘禮坐下,因見婦人「姐夫長,姐夫短」叫他,口中不言,心內暗道:「這淫婦怎的不認犯,只叫我姐夫?等我慢慢的探他。」當下酒過三巡,餚添五道,無人在跟前,先丟幾句邪言說入去,道:「我兄弟思想姐姐,如渴思漿,如熱思涼,想當初在丈人家,怎的在一處下棋抹牌,同坐雙雙,似背蓋一般。誰承望今日各自分散,你東我西。」玉樓笑道:「姐夫好說。自古清者清而渾者渾,久而自見。」這敬濟笑嘻嘻向袖中取出一包雙人兒的香茶,遞與婦人,說:「姐姐,你若有情,可憐見兄弟,吃我這個香茶兒。」說著,就連忙跪下。那婦人登時一點紅從耳畔起,把臉飛紅了,一手把香茶包兒掠在地下,說道:「好不識人敬重!奴好意遞酒與你吃,到戲弄我起來。」就撇了酒席往房裡去了。敬濟見他不理,一面拾起香茶來,就發話道:「我好意來看你,你到變了卦兒。你敢說你嫁了通判兒子好漢子,不採我了。你當初在西門慶家做第三個小老婆,沒曾和我兩個有首尾?」因向袖中取出舊時那根金頭銀簪子,拿在手內說:「這個是誰人的?你既不和我有姦,這根簪兒怎落在我手裡?上面還刻著玉樓名字。你和大老婆串同了,把我家寄放的八箱子金銀細軟、玉帶寶石東西,都是當朝楊戩寄放應沒官之物,都帶來嫁了漢子。我教你不要慌,到八字八\textcombine{釒夏}兒上和你答話!」

玉樓見他發話,拿的簪子委是他頭上戴的金頭蓮瓣簪兒:「昔日在花園中不見,怎的落在這短命手裡?」恐怕嚷的家下人知道,須臾變作笑吟吟臉兒,走將出來,一把手拉敬濟,說道:「好阻夫,奴鬥你耍子,如何就惱起來。」因觀看左右無人,悄悄說:「你既有心,奴亦有意。」兩個不由分說,摟著就親嘴。這陳敬濟把舌頭似蛇吐信子一般,就舒到他口裡交他咂,說道:「你叫我聲親親的丈夫,才算你有我之心。」婦人道:「且禁聲,只怕有人聽見。」敬濟悄悄向他說:「我如今治了半船貨,在清江浦等候。你若肯下顧時,如此這般,到晚夕假扮門子,私走出來,跟我上船家去,成其夫婦,有何不可?他一個文職官,怕是非,莫不敢來抓尋你不成?」婦人道:「既然如此,也罷。」約會下:「你今晚在府牆後等著,奴有一包金銀細軟,打牆上系過去,與你接了,然後奴才扮做門子,打門裡出來,跟你上船去罷。」看官聽說,正是佳人有意,那怕粉牆高萬丈;紅粉無情,總然共坐隔千山。當時孟玉樓若嫁得個痴蠢之人,不如敬濟,敬濟便下得這個鍬钁著;如今嫁這李衙內,有前程,又且人物風流,青春年少,恩情美滿,他又勾你做甚?休說平日又無連手。這個郎君也是合當倒運,就吐實話,泄機與他,倒吃婆娘哄賺了。正是:
\begin{quote}
花枝葉下猶藏刺,人心難保不懷毒。
\end{quote}

當下二人會下話,這敬濟吃了幾杯酒,告辭回去。李衙內連忙送出府門,陳安跟隨而去。衙內便問婦人:「你兄弟住那裡下處?我明日回拜他去,送些嗄程與他。」婦人便說:「那裡是我兄弟,他是西門慶家女婿,如此這般,來勾搭要拐我出去。奴已約下他,今晚三更在後牆相等。咱不如將計就計,把他當賊拿下,除其後患如何?」衙內道:「叵耐這廝無端,自古無毒不丈夫,不是我去尋他,他自來送死。」一面走出外邊,叫過左右伴當,心腹快手,如此這般預備去了。這陳敬濟不知機變,至半夜三更,果然帶領家人陳安,來府衙後牆下,以咳嗽為號,只聽牆內玉樓聲音,打牆上掠過一條索子去,那邊系過一大包銀子。原來是庫內拿的二百兩贓罰銀子。這敬濟才待教陳安拿著走,忽聽一陣梆子響,黑影里閃出四五條漢,叫聲:「有賊了!」登時把敬濟連陳安都綁了,稟知李通判,分付:「都且押送牢里去,明日問理。」

原來嚴州府正堂知府姓徐,名喚徐崶,系陝西臨洮府人氏,庚戌進士,極是個清廉剛正之人。次早升堂,左右排兩行官吏,這李通判上去,畫了公座,庫子呈稟賊情事,帶陳敬濟上去,說:「昨夜至一更時分,有先不知名今知名賊人二名:陳敬濟、陳安,鍬開庫門鎖鑰,偷出贓銀二百兩,越牆而過,致被捉獲,來見老爺。」徐知府喝令:「帶上來!」把陳敬濟並陳安揪採驅擁至當廳跪下。知府見敬濟年少清俊,便問:「這廝是那裡人氏?因何來我這府衙公廨,夜晚做賊,偷盜官庫贓銀,有何理說?」那陳敬濟只顧磕頭聲冤。徐知府道:「你做賊如何聲冤?」李通判在旁欠身便道:「老先生不必問他,眼見得贓證明白,何不回刑起來。」徐知府即令左右:「拿下去打二十板。」李通判道:「人是苦蟲,不打不成。不然,這賊便要展轉。」當下兩邊皂隸,把敬濟、陳安拖番,大板打將下來。這陳敬濟口內只罵:「誰知淫婦孟三兒陷我至此,冤哉!苦哉!」這徐知府終是黃堂出身官人,聽見這一聲,必有緣故,才打到十板上,喝令:「住了,且收下監去,明日再問。」李通判道:「老先生不該發落他,常言『人心似鐵,官法如爐』,從容他一夜不打緊,就翻異口詞。」徐知府道:「無妨,吾自有主意。」當下獄卒把敬濟、陳安押送監中去訖。

這徐知府心中有些疑忌,即喚左右心腹近前,如此這般,下監中探聽敬濟所犯來歷,即便回報。這幹事人假扮作犯人,和敬濟晚間在一㭱上睡,問其所以:「我看哥哥青春年少,不是做賊的,今日落在此,打屈官司。」敬濟便說:「一言難盡,小人本是清河縣西門慶女婿,這李通判兒子新娶的婦人孟氏,是俺丈人的小,舊與我有姦的。今帶過我家老爺楊戩寄放十箱金銀寶玩之物來他家,我來此間問他索討,反被他如此這般欺負,把我當賊拿了。苦打成招,不得見其天日,是好苦也!」這人聽了,走來退廳告報徐知府。知府道:「如何?我說這人聲冤叫孟氏,必有緣故。」

到次日升堂,官吏兩旁侍立。這徐知府把陳敬濟、陳安提上來,摘了口詞,取了張無事的供狀,喝令釋放。李通判在旁不知,還再三說:「老先生,這廝賊情既的,不可放他。」反被徐知府對佐貳官儘力數說了李通判一頓,說:「我居本府正官,與朝廷幹事,不該與你家官報私仇,誣陷平人作賊。你家兒子娶了他丈人西門慶妾孟氏,帶了許多東西,應沒官贓物,金銀箱籠來。他是西門慶女婿,徑來索討前物,你如何假捏賊情,拿他入罪,教我替你家出力?做官養兒養女,也要長大,若是如此,公道何堪?」當廳把李通判數說的滿面羞慚,垂首喪氣而不敢言。陳敬濟與陳安便釋放出去了。良久,徐知府退堂。

這李通判回到本宅,心中十分焦燥。便對夫人大嚷大叫道:「養的好不肖子,今日吃徐知府當堂對眾同僚官吏,儘力數落了我一頓,可不氣殺我也!」夫人慌了,便道:「甚麼事?」李通判即把兒子叫到跟前,喝令左右:「拿大板子來,氣殺我也!」說道:「你拿得好賊,他是西門慶女婿。因這婦人帶了許多妝奩、金銀箱籠來,他口口聲聲稱是當朝逆犯楊戩寄放應沒官之物,來問你要。說你假盜出庫中官銀,當賊情拿他。我通一字不知,反被正堂徐知府對眾數說了我這一頓。此是我頭一日官未做,你照顧我的。我要你這不肖子何用?」即令左右雨點般大板子打將下來。可憐打得這李衙內皮開肉綻,鮮血迸流。夫人見打得不像模樣,在旁哭泣勸解。孟玉樓立在後廳角門首,掩淚潛聽。當下打了三十大板,李通判分付左右:「押著衙內,即時與我把婦人打發出門,令他任意改嫁,免惹是非,全我名節。」那李衙內心中怎生捨得離異,只顧在父母跟前啼哭哀告:「寧把兒子打死爹爹跟前,並舍不的婦人。」李通判把衙內用鐵索墩鎖在後堂,不放出去,只要囚禁死他。夫人哭道:「相公,你做官一場,年紀五十餘歲,也只落得這點骨血。不爭為這婦人,你囚死他,往後你年老休官,倚靠何人?」李通判道:「不然,他在這裡,須帶累我受人氣。」夫人道:「你不容他在此,打發他兩口兒回原籍真定府家去便了。」通判依聽夫人之言,放了衙內,限三日就起身,打點車輛,同婦人歸棗強縣裡攻書去了。

卻表陳敬濟與陳安出離嚴州府,到寺中取了行李,徑往清江浦陳二店中來尋楊大郎。陳二說:「他三日前,說你有信來說不得來,他收拾了貨船,起身往家中去了。」這敬濟未信,還向河下去尋船隻,撲了個空。說道:「這天殺的,如何不等我來就起身去了!」況新打監中出來,身邊盤纏已無,和陳安不免搭在人船上,把衣衫解當,討吃歸家,忙忙似喪家之犬,急急如漏網之魚,隨行找尋楊大郎,並無蹤跡。那時正值秋暮天氣,樹木凋零,金風搖落,甚是凄涼。有詩八句,單道這秋天行人最苦:
\begin{quote}
棲棲芰荷枯,葉葉梧桐墜。
蛩鳴腐草中,雁落平沙地。
細雨濕青林,霜重寒天氣。
不見路行人,怎曉秋滋味。
\end{quote}

有日敬濟到家。陳定正在門首,看見敬濟來家,衣衫襤褸,面貌黧黑,唬了一跳。接到家中,問貨船到於何處。敬濟氣得半日不言,把嚴州府遭官司一節說了:「多虧正堂徐知府放了我,不然性命難保。今被楊大郎這天殺的,把我貨物不知拐的往那裡去了。」先使陳定往他家探聽,他家說還不曾來家。敬濟又親去問了一遭,並沒下落,心中著慌,走入房中。那馮金寶又和西門大姐首南面北,自從敬濟出門,兩個合氣,直到如今。大姐便說:「馮金寶拿著銀子錢,轉與他鴇子去了。他家保兒成日來,瞞藏背掖,打酒買肉,在屋裡吃。家中要的沒有,睡到晌午,諸事兒不買,只熬俺們。」馮金寶又說:「大姐成日模草不拈,豎草不動,偷米換燒餅吃。又把煮的腌肉偷在房裡,和丫頭元宵兒同吃。」這陳敬濟就信了,反罵大姐:「賊不是才料淫婦,你害饞癆讒痞了,偷米出去換燒餅吃,又和丫頭打夥兒偷肉吃。」把元宵兒打了一頓,把大姐踢了幾腳。這大姐急了,趕著馮金寶兒撞頭,罵道:「好養漢的淫婦!你偷盜的東西與鴇子不值了,到學舌與漢子,說我偷米偷肉,犯夜的倒拿住巡更的了,教漢子踢我。我和你這淫婦兌換了罷,要這命做甚麼!」這敬濟道:「好淫婦,你換兌他,你還不值他幾個腳指頭兒哩。」也是合當有事,於是一把手採過大姐頭髮來,用拳撞腳踢、拐子打,打得大姐鼻口流血,半日蘇醒過來。這敬濟便歸唱的房裡睡去了。由著大姐在下邊房裡嗚嗚咽咽,只顧哭泣。元宵兒便在外間睡著了。可憐大姐到半夜,用一條索子懸梁自縊身死,亡年二十四歲。

到次日早辰,元宵起來,推裡間不開。上房敬濟和馮金寶還在被窩裡,使他丫頭重喜兒來叫大姐,要取木盆洗坐腳,只顧推不開。敬濟還罵:「賊淫婦,如何還睡?這咱晚不起來!我這一跺開門進去,把淫婦鬢毛都拔凈了。」重喜兒打窗眼內望里張看,說道:「他起來了,且在房裡打鞦韆耍子兒哩。」又說:「他提偶戲耍子兒哩。」只見元宵瞧了半日,叫道:「爹,不好了,俺娘弔在床頂上弔死了。」這小郎才慌了,和唱的齊起來,跺開房門,向前解卸下來,灌救了半日,那得口氣兒來。不知多咱時分,嗚呼哀哉死了。正是:
\begin{quote}
不知真性歸何處,疑在行雲秋水中。
\end{quote}

陳定聽見大姐死了,恐怕連累,先走去報知月娘。月娘聽見大姐弔死了,敬濟娶唱的在家,正是冰厚三尺,不是一日之寒,率領家人小廝、丫鬟媳婦七八口,往他家來。見了大姐屍首弔的直挺挺的,哭喊起來,將敬濟拿住,揪採亂打,渾身錐了眼兒也不計數。唱的馮金寶躲在床底下,採出來,也打了個臭死。把門窗戶壁都打得七零八落,房中床帳妝奩都還搬的去了。歸家請將吳大舅、二舅來商議。大舅說:「姐姐,你趁此時咱家人死了不到官,到明日他過不得日子,還來纏要箱籠。人無遠慮,必有近憂。不如到官處斷開了,庶杜絕後患。」月娘道:「哥見得是。」一面寫了狀子。

次日,月娘親自出官,來到本縣授官廳下,遞上狀去。原來新任知縣姓霍,名大立,湖廣黃岡縣人氏,舉人出身,為人鯁直。聽見系人命重事,即升廳受狀。見狀上寫著:
\begin{quote}
告狀人吳氏,年三十四歲,系已故千戶西門慶妻。狀告為惡婿欺凌孤孀,聽信娼婦,熬打逼死女命,乞憐究治,以存殘喘事。比有女婿陳敬濟,遭官事投來氏家,潛住數年。平日吃酒行兇,不守本分,打出弔入。氏懼法逐離出門。豈期敬濟懷恨,在家將氏女西門氏,時常熬打,一向含忍。不料伊又娶臨清娼婦馮金寶來家,奪氏女正房居住,聽信唆調,將女百般痛辱熬打,又採去頭髮,渾身踢傷,受忍不過,比及將死,於本年八月廿三日三更時分,方纔將女上吊縊死。切思敬濟,恃逞凶頑,欺氏孤寡,聲言還要持刀殺害等語,情理難容。乞賜行拘到案,嚴究女死根由,盡法如律。庶凶頑知警,良善得以安生,而死者不為含冤矣。為此具狀上告本縣青天老爺施行。
\end{quote}

這霍知縣在公座上看了狀子,又見吳月娘身穿縞素,腰系孝裙,系五品職官之妻,生的容貌端莊,儀容閑雅。欠身起來,說道:「那吳氏起來,據我看,你也是個命官娘子,這狀上情理,我都知了。你請回去,今後只令一家人在此伺候就是了。我就出牌去拿他。」那吳月娘連忙拜謝了知縣,出來坐轎子回家,委付來昭廳下伺候。須臾批了呈狀,委兩個公人,一面白牌,行拘敬濟、娼婦馮金寶,並兩鄰保甲,正身赴官聽審。

這敬濟正在家裡亂喪事,聽見月娘告下狀來,縣中差公人發牌來拿他,唬的魂飛天外,魄喪九霄。那馮金寶已被打得渾身疼痛,睡在床上。聽見人拿他,唬的魂也不知有無。陳敬濟沒高低使錢,打發公人吃了酒飯,一條繩子連唱的都拴到縣裡。左鄰範綱,右鄰孫紀,保甲王寬。霍知縣聽見拿了人來,即時升廳。來昭跪在上首,陳敬濟、馮金寶一行人跪在階下。知縣看了狀子,便叫敬濟上去說:「你這廝可惡!因何聽信娼婦,打死西門氏,方令上吊,有何理說?」敬濟磕頭告道:「望乞青天老爺察情,小的怎敢打死他。因為搭夥計在外,被人坑陷了資本,著了氣來家,問他要飯吃。他不曾做下飯,委被小的踢了兩腳。他到半夜自縊身死了。」知縣喝道:「你既娶下娼婦,如何又問他要飯吃?尤說不通。吳氏狀上說你打死他女兒,方纔上吊,你還不招認!」敬濟說:「吳氏與小的有仇,故此誣陷小的,望老爺察情。」知縣大怒,說:「他女兒見死了,還推賴那個?」喝令左右拿下去,打二十大板。提馮金寶上來,拶了一拶,敲一百敲。令公人帶下收監。次日,委典史臧不息帶領吏書、保甲、鄰人等,前至敬濟家,抬出屍首,當場檢驗。身上俱有青傷,脖項間亦有繩痕,生前委因敬濟踢打傷重,受忍不過,自縊身死。取供具結,回報縣中。知縣大怒,又打了敬濟十板。金寶褪衣,也是十板。問陳敬濟夫毆妻至死者絞罪,馮金寶遞決一百,發回本司院當差。

這陳敬濟慌了,監中寫出貼子,對陳定說,把布鋪中本錢,連大姐頭面,共湊了一百兩銀子,暗暗送與知縣。知縣一夜把招捲改了,止問了個逼令身死,系雜犯,準徒五年,運灰贖罪。吳月娘再三跪門哀告。知縣把月娘叫上去,說道:「娘子,你女兒項上已有繩痕,如何問他毆殺條律?人情莫非忒偏向麼?你怕他後邊纏擾你,我這裡替你取了他杜絕文書,令他再不許上你門就是了。」一面把陳敬濟提到跟前,分付道:「我今日饒你一死,務要改過自新,不許再去吳氏家纏擾。再犯到我案下,決然不饒。即便把西門氏買棺裝殮,發送葬埋來回話,我這裡好申文書往上司去。」這敬濟得了個饒,交納了贖罪銀子,歸到家中,抬屍入棺,停放一七,念經送葬,埋城外。前後坐了半個月監,使了許多銀兩,唱的馮金寶也去了,家中所有都乾凈了,房兒也典了,剛刮剌出個命兒來,再也不敢聲言丈母了。正是:
\begin{quote}
禍福無門人自招,須知樂極有悲來。
\end{quote}
有詩為證:
\begin{quote}
風波平地起蕭牆,義重恩深不可忘。
水溢藍橋應有會,三星權且作參商。
\end{quote}
