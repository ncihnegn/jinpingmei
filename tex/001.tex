
\chapter{西門慶熱結十弟兄 武二郎冷遇親哥嫂}

詩曰:
\begin{quote}
豪華去後行人絕,簫箏不響歌喉咽。
雄劍無威光彩沉,寶琴零落金星滅。
玉階寂寞墜秋露,月照當時歌舞處。
當時歌舞人不回,化為今日西陵灰。
\end{quote}

又詩曰:
\begin{quote}
二八佳人體似酥,腰間仗劍斬愚夫。
雖然不見人頭落,暗裡教君骨髓枯。
\end{quote}

這一首詩,是昔年大唐國時,一個修真煉性的英雄,入聖超凡的豪傑,到後來位居紫府,名列仙班,率領上八洞群仙,救拔四部洲沉苦一位仙長,姓呂名岩,道號純陽子祖師所作。單道世上人,營營逐逐,急急巴巴,跳不出七情六慾關頭,打不破酒色財氣圈子。到頭來同歸於盡,著甚要緊!雖是如此說,只這酒色財氣四件中,惟有財色二者更為利害。怎見得他的利害?假如一個人到了那窮苦的田地,受盡無限凄涼,耐盡無端懊惱,晚來摸一摸米瓮,苦無隔宿之炊,早起看一看廚前,愧無半星煙火,妻子饑寒,一身凍餒,就是那粥飯尚且艱難,那討餘錢沽酒!更有一種可恨處,親朋白眼,面目寒酸,便是凌雲志氣,分外消磨,怎能夠與人爭氣!正是:
\begin{quote}
一朝馬死黃金盡,親者如同陌路人。
\end{quote}

到得那有錢時節,揮金買笑,一擲巨萬。思飲酒真個瓊漿玉液,不數那琥珀杯流;要鬥氣錢可通神,果然是頤指氣使。趨炎的壓脊挨肩,附勢的吮癰舐痔,真所謂得勢疊肩而來,失勢掉臂而去。古今炎冷惡態,莫有甚於此者。這兩等人,豈不是受那財的利害處!如今再說那色的利害。請看如今世界,你說那坐懷不亂的柳下惠,閉門不納的魯男子,與那秉燭達旦的關雲長,古今能有幾人?至如三妻四妾,買笑追歡的,又當別論。還有那一種好色的人,見了個婦女略有幾分顏色,便百計千方偷寒送暖,一到了著手時節,只圖那一瞬歡娛,也全不顧親戚的名分,也不想朋友的交情。起初時不知用了多少濫錢,費了幾遭酒食。正是:
\begin{quote}
三杯花作合,兩盞色媒人。
\end{quote}

到後來情濃事露,甚而鬥狠殺傷,性命不保,妻孥難顧,事業成灰。就如那石季倫潑天豪富,為綠珠命喪囹圄;楚霸王氣概拔山,因虞姬頭懸垓下。真所謂:「生我之門死我戶,看得破時忍不過」。這樣人豈不是受那色的利害處!

說便如此說,這財色二字,從來只沒有看得破的。若有那看得破的,便見得堆金積玉,是棺材內帶不去的瓦礫泥沙;貫朽粟紅,是皮囊內裝不盡的臭淤糞土。高堂廣廈,玉宇瓊樓,是墳山上起不得的享堂;錦衣繡襖,狐服貂裘,是骷髏上裹不了的敗絮。即如那妖姬艷女,獻媚工妍,看得破的,卻如交鋒陣上將軍叱吒獻威風;硃唇皓齒,掩袖回眸,懂得來時,便是閻羅殿前鬼判夜叉增惡態。羅襪一彎,金蓮三寸,是砌墳時破土的鍬鋤;枕上綢繆,被中恩愛,是五殿下油鍋中生活。只有那金剛經上兩句說得好,他說道:「如夢幻泡影,如電復如露。」見得人生在世,一件也少不得,到了那結束時,一件也用不著。隨著你舉鼎蕩舟的神力,到頭來少不得骨軟筋麻;由著你銅山金谷的奢華,正好時卻又要冰消雪散。假饒你閉月羞花的容貌,一到了垂眉落眼,人皆掩鼻而過之;比如你陸賈隋何的機鋒,若遇著齒冷唇寒,吾未如之何也已。到不如削去六根清凈,披上一領袈裟,參透了空色世界,打磨穿生滅機關,直超無上乘,不落是非窠,倒得個清閒自在,不向火坑中翻筋斗也。正是:
\begin{quote}
三寸氣在千般用,一日無常萬事休。
\end{quote}

說話的為何說此一段酒色財氣的緣故?只為當時有一個人家,先前恁地富貴,到後來煞甚凄涼,權謀術智,一毫也用不著,親友兄弟,一個也靠不著,享不過幾年的榮華,倒做了許多的話靶。內中又有幾個鬥寵爭強,迎姦賣俏的,起先好不妖嬈嫵媚,到後來也免不得屍橫燈影,血染空房。正是:
\begin{quote}
善有善報,惡有惡報;天網恢恢,疏而不漏。
\end{quote}

話說大宋徽宗皇帝政和年間,山東省東平府清河縣中,有一個風流子弟,生得狀貌魁梧,性情瀟灑,饒有幾貫家資,年紀二十六七。這人複姓西門,單諱一個慶字。他父親西門達,原走川廣販藥材,就在這清河縣前開著一個大大的生藥鋪。現住著門面五間到底七進的房子。家中呼奴使婢,騾馬成群,雖算不得十分富貴,卻也是清河縣中一個殷實的人家。只為這西門達員外夫婦去世的早,單生這個兒子卻又百般愛惜,聽其所為,所以這人不甚讀書,終日閒遊浪蕩。一自父母亡後,專一在外眠花宿柳,惹草招風,學得些好拳棒,又會賭博,雙陸象棋,抹牌道字,無不通曉。結識的朋友,也都是些幫閒抹嘴,不守本分的人。第一個最相契的,姓應名伯爵,表字光侯,原是開綢緞鋪應員外的第二個兒子,落了本錢,跌落下來,專在本司三院幫嫖貼食,因此人都起他一個渾名叫做應花子。又會一腿好氣毬,雙陸棋子,件件皆通。第二個姓謝名希大,字子純,乃清河衛千戶官兒應襲子孫,自幼父母雙亡,遊手好閒,把前程丟了,亦是幫閒勤兒,會一手好琵琶。自這兩個與西門慶甚合得來。其餘還有幾個,都是些破落戶,沒名器的。一個叫做祝實念,表字貢誠。一個叫做孫天化,表字伯修,綽號孫寡嘴。一個叫做吳典恩,乃是本縣陰陽生,因事革退,專一在縣前與官吏保債,以此與西門慶往來。還有一個雲參將的兄弟叫做雲理守,字非去。一個叫做常峙節,表字堅初。一個叫做卜志道。一個叫做白賚光,表字光湯。說這白賚光,眾人中也有道他名字取的不好聽的,他卻自己解說道:「不然我也改了,只為當初取名的時節,原是一個門館先生,說我姓白,當初有一個什麼故事,是白魚躍入武王舟。又說有兩句書是『周有大賚,於湯有光」,取這個意思,所以表字就叫做光湯。我因他有這段故事,也便不改了。」說這一干共十數人,見西門慶手裡有錢,又撒漫肯使,所以都亂撮哄著他耍錢飲酒,嫖賭齊行。正是:
\begin{quote}
把盞銜杯意氣深,兄兄弟弟抑何親。
一朝平地風波起,此際相交才見心。
\end{quote}

說話的,這等一個人家,生出這等一個不肖的兒子,又搭了這等一班無益有損的朋友,隨你怎的豪富也要窮了,還有甚長進的日子!卻有一個緣故,只為這西門慶生來秉性剛強,作事機深詭譎,又放官吏債,就是那朝中高楊童蔡四大姦臣,他也有門路與他浸潤。所以專在縣裡管些公事,與人把攪說事過錢,因此滿縣人都懼怕他。因他排行第一,人都叫他是西門大官人。這西門大官人先頭渾家陳氏早逝,身邊只生得一個女兒,叫做西門大姐,就許與東京八十萬禁軍楊提督的親家陳洪的兒子陳敬濟為室,尚未過門。只為亡了渾家,無人管理家務,新近又娶了本縣清河左衛吳千戶之女填房為繼室。這吳氏年紀二十五六,是八月十五生的,小名叫做月姐,後來嫁到西門慶家,都順口叫他月娘。卻說這月娘秉性賢能,夫主面上百依百隨。房中也有三四個丫鬟婦女,都是西門慶收用過的。又嘗與勾欄內李嬌兒打熱,也娶在家裡做了第二房娘子。南街又占著窠子卓二姐,名卓丟兒,包了些時,也娶來家做了第三房。只為卓二姐身子瘦怯,時常三病四痛,他卻又去飄風戲月,調弄人家婦女。正是:
\begin{quote}
東家歌笑醉紅顏,又向西鄰開玳宴。
幾日碧桃花下臥,牡丹開處總堪憐。
\end{quote}

話說西門慶一日在家閒坐,對吳月娘說道:「如今是九月廿五日了,出月初三日,卻是我兄弟們的會期。到那日也少不的要整兩席齊整的酒席,叫兩個唱的姐兒,自恁在咱家與兄弟們好生玩耍一日。你與我料理料理。」吳月娘便道:「你也便別要說起這幹人,那一個是那有良心和行貨!無過每日來勾使的遊魂撞屍。我看你自搭了這起人,幾時曾有個家哩!現今卓二姐自恁不好,我勸你把那酒也少要吃了。」西門慶道:「你別的話倒也中聽。今日這些說話,我卻有些不耐煩聽他。依你說,這些兄弟們沒有好人,使著他,沒有一個不依順的,做事又十分停當,就是那謝子純這個人,也不失為個伶俐能事的好人。咱如今是這等計較罷,只管恁會來會去,終不著個切實。咱不如到了會期,都結拜了兄弟罷,明日也有個靠傍些。」吳月娘接過來道:「結拜兄弟也好。只怕後日還是別個靠你的多哩。若要你去靠人,提傀儡兒上戲場——還少一口氣兒哩。」西門慶笑道:「自恁長把人靠得著,卻不更好了。咱只等應二哥來,與他說這話罷。」

正說著話,只見一個小廝兒,生得眉清目秀,伶俐乖覺,原是西門慶貼身伏侍的,喚名玳安兒,走到面前來說:「應二叔和謝大叔在外見爹說話哩。」西門慶道:「我正說他,他卻兩個就來了。」一面走到廳上來,只見應伯爵頭上戴一頂新盔的玄羅帽兒,身上穿一件半新不舊的天青夾縐紗褶子,腳下絲鞋淨襪,坐在上首。下首坐的,便是姓謝的謝希大。見西門慶出來,一齊立起身來,邊忙作揖道:「哥在家,連日少看。」西門慶讓他坐下,一面喚茶來吃,說道:「你們好人兒,這幾日我心裡不耐煩,不出來走跳,你們通不來傍個影兒。」伯爵向希大道:「何如?我說哥哥要說哩。」因對西門慶道:「哥,你怪的是。連咱自也不知道成日忙些什麼!自咱們這兩隻腳,還趕不上一張嘴哩。」西門慶因問道:「你這兩日在那裡來?」伯爵道:「昨日在院中李家瞧了個孩子兒,就是哥這邊二嫂子的侄女兒桂卿的妹子,叫做桂姐兒。幾時兒不見他,就出落的好不標緻了。到明日成人的時候,還不知怎的樣好哩!昨日他媽再三向我說:『二爹,千萬尋個好子弟梳籠他。』敢怕明日還是哥的貨兒哩。」西門慶道:「有這等事!等咱空閒了去瞧瞧。」謝希大接過來道:「哥不信,委的生得十分顏色。」西門慶道:「昨日便在他家,前幾日卻在那裡去來?」伯爵道:「便是前日卜志道兄弟死了,咱在他家幫著亂了幾日,發送他出門。他嫂子再三向我說,叫我拜上哥,承哥這裡送了香楮奠禮去,因他沒有寬轉地方兒,晚夕又沒甚好酒席,不好請哥坐的,甚是過不意去。」西門慶道:「便是我聞得他不好得沒多日子,就這等死了。我前日承他送我一把真金川扇兒,我正要拿甚答謝答謝,不想他又作了故人!」

謝希大便嘆了一口氣道:「咱會中兄弟十人,卻又少他一個了。」因向伯爵說:「出月初三日,又是會期,咱每少不得又要煩大官人這裡破費,兄弟們頑耍一日哩。」西門慶便道:「正是,我剛纔正對房下說來,咱兄弟們似這等會來會去,無過只是吃酒頑耍,不著一個切實,倒不如尋一個寺院裡,寫上一個疏頭,結拜做了兄弟,到後日彼此扶持,有個傍靠。到那日,咱少不得要破些銀子,買辦三牲,眾兄弟也便隨多少各出些分資。不是我科派你們,這結拜的事,各人出些,也見些情分。」伯爵連忙道:「哥說的是。婆兒燒香當不的老子念佛,各自要盡自的心。只是俺眾人們,老鼠尾巴生瘡兒——有膿也不多。」西門慶笑道:「怪狗才,誰要你多來!你說這話。」謝希大道:「結拜須得十個方好。如今卜志道兄弟沒了,卻教誰補?」西門慶沉吟了一回,說道:「咱這間壁花二哥,原是花太監侄兒,手裡肯使一股濫錢,常在院中走動。他家後邊院子與咱家只隔著一層壁兒,與我甚說得來,咱不如叫小廝邀他邀去。」應伯爵拍著手道:「敢就是在院中包著吳銀兒的花子虛麼?」西門慶道:「正是他!」伯爵笑道:「哥,快叫那個大官兒邀他去。與他往來了,咱到日後,敢又有一個酒碗兒。」西門慶笑道:「傻花子,你敢害饞癆痞哩,說著的是吃。」大家笑了一回。西門慶旋叫過玳安兒來說:「你到間壁花家去,對你花二爹說,如此這般:『俺爹到了出月初三日,要結拜十兄弟,敢叫我請二爹上會哩。』看他怎的說,你就來回我話。你二爹若不在家,就對他二娘說罷。」玳安兒應諾去了。伯爵便道:「到那日還在哥這裡是,還在寺院裡好?」希大道:「咱這裡無過只兩個寺院,僧家便是永福寺,道家便是玉皇廟。這兩個去處,隨分那裡去罷。」西門慶道:「這結拜的事,不是僧家管的,那寺里和尚,我又不熟,倒不如玉皇廟吳道官與我相熟,他那裡又寬展又幽靜。」伯爵接過來道:「哥說的是,敢是永福寺和尚倒和謝家嫂子相好,故要薦與他去的。」希大笑罵道:「老花子,一件正事,說說就放出屁來了。」

正說笑間,只見玳安兒轉來了,因對西門慶說道:「他二爹不在家,俺對他二娘說來。二娘聽了,好不歡喜,說道:『既是你西門爹攜帶你二爹做兄弟,那有個不來的。等來家我與他說,至期以定攛掇他來,多拜上爹。』又與了小的兩件茶食來了。」西門慶對應、謝二人道:「自這花二哥,倒好個伶俐標緻娘子兒。」說畢,又拿一盞茶吃了,二人一齊起身道:「哥,別了罷,咱好去通知眾兄弟,糾他分資來。哥這裡先去與吳道官說聲。」西門慶道:「我知道了,我也不留你罷。」於是一齊送出大門來。應伯爵走了幾步,迴轉來道:「那日可要叫唱的?」西門慶道:「這也罷了,弟兄們說說笑笑,到有趣些。」說畢,伯爵舉手,和希大一路去了。

話休饒舌,捻指過了四五日,卻是十月初一日。西門慶早起,剛在月娘房裡坐的,只見一個才留頭的小廝兒,手裡拿著個描金退光拜匣,走將進來,向西門慶磕了一個頭兒,立起來站在旁邊說道:「俺是花家,俺爹多拜上西門爹。那日西門爹這邊叫大官兒請俺爹去,俺爹有事出門了,不曾當面領教的。聞得爹這邊是初三日上會,俺爹特使小的先送這些分資來,說爹這邊胡亂先用著,等明日爹這裡用過多少派開,該俺爹多少,再補過來便了。」西門慶拿起封袋一看,簽上寫著「分資一兩」,便道:「多了,不消補的。到後日叫爹莫往那去,起早就要同眾爹上廟去。」那小廝兒應道:「小的知道。」剛待轉身,被吳月娘喚住,叫大丫頭玉簫在食籮里揀了兩件蒸酥果餡兒與他。因說道:「這是與你當茶的。你到家拜上你家娘,你說西門大娘說,遲幾日還要請娘過去坐半日兒哩。」那小廝接了,又磕了一個頭兒,應著去了。

西門慶才打發花家小廝出門,只見應伯爵家應寶夾著個拜匣,玳安兒引他進來見了,磕了頭,說道:「俺爹糾了眾爹們分資,叫小的送來,爹請收了。」西門慶取出來看,共總八封,也不拆看,都交與月娘,道:「你收了,到明日上廟,好湊著買東西。」說畢,打發應寶去了。立起身到那邊看卓二姐。剛走到坐下,只見玉簫走來,說道:「娘請爹說話哩。」西門慶道:「怎的起先不說來?」隨即又到上房,看見月娘攤著些紙包在面前,指著笑道:「你看這些分子,止有應二的是一錢二分八成銀子,其餘也有三分的,也有五分的,都是些紅的黃的,倒象金子一般。咱家也曾沒見這銀子來,收他的也污個名,不如掠還他罷。」西門慶道:「你也耐煩,丟著罷,咱多的也包補,在乎這些!」說著一直往前去了。

到了次日初二日,西門慶稱出四兩銀子,叫家人來興兒買了一口豬、一口羊、五六壇金華酒和香燭紙札、雞鴨案酒之物,又封了五錢銀子,旋叫了大家人來保和玳安兒、來興三個:「送到玉皇廟去,對你吳師父說:『俺爹明日結拜兄弟,要勞師父做紙疏辭,晚夕就在師父這裡散福。煩師父與俺爹預備預備,俺爹明早便來。』」只見玳安兒去了一會,來回說:「已送去了,吳師父說知道了。」

須臾,過了初二,次日初三早,西門慶起來梳洗畢,叫玳安兒:「你去請花二爹,到咱這裡吃早飯,一同好上廟去。一發到應二叔家,叫他催催眾人。」玳安應諾去,剛請花子虛到來,只見應伯爵和一班兄弟也來了,卻正是前頭所說的這幾個人。為頭的便是應伯爵,謝希大、孫天化、祝念實、吳典恩、雲理守、常峙節、白賚光,連西門慶、花子虛共成十個。進門來一齊籮圈作了一個揖。伯爵道:「咱時候好去了。」西門慶道:「也等吃了早飯著。」便叫:「拿茶來。」一面叫:「看菜兒。」須臾,吃畢早飯,西門慶換了一身衣服,打選衣帽光鮮,一齊徑往玉皇廟來。

不到數里之遙,早望見那座廟門,造得甚是雄峻。但見:
\begin{quote}
殿宇嵯峨,宮牆高聳。正面前起著一座牆門八字,一帶都粉赭色紅泥;進裡邊列著三條甬道川紋,四方都砌水痕白石。正殿上金碧輝煌,兩廊下檐阿峻峭。三清聖祖莊嚴寶相列中央,太上老君背倚青牛居後殿。
\end{quote}

進入第二重殿後,轉過一重側門,卻是吳道官的道院。進的門來,兩下都是些瑤草琪花,蒼松翠竹。西門慶抬頭一看,只見兩邊門楹上貼著一副對聯道:
\begin{quote}
洞府無窮歲月,壺天別有乾坤。
\end{quote}

上面三間敞廳,卻是吳道官朝夕做作功課的所在。當日鋪設甚是齊整,上面掛的是昊天金闕玉皇上帝,兩邊列著的紫府星官,側首掛著便是馬、趙、溫、關四大元帥。當下吳道官卻又在經堂外躬身迎接。西門慶一起人進入裡邊,獻茶已罷,眾人都起身,四圍觀看。白賚光攜著常峙節手兒,從左邊看將過來,一到馬元帥面前,見這元帥威風凜凜,相貌堂堂,面上畫著三隻眼睛,便叫常峙節道:「哥,這卻是怎的說?如今世界,開隻眼閉隻眼兒便好,還經得多出隻眼睛看人破綻哩!」應伯爵聽見,走過來道:「呆兄弟,他多隻眼兒看你倒不好麼?」眾人笑了。常峙節便指著下首溫元帥道:「二哥,這個通身藍的,卻也古怪,敢怕是盧杞的祖宗。」伯爵笑著猛叫道:「吳先生你過來,我與你說個笑話兒。」那吳道官真個走過來聽他。伯爵道:「一個道家死去,見了閻王,閻王問道:『你是什麼人?』道者說:『是道士。』閻王叫判官查他,果系道士,且無罪孽。這等放他還魂。只見道士轉來,路上遇著一個染房中的博士,原認得的,那博士問道:『師父,怎生得轉來?』道者說:『我是道士,所以放我轉來。』那博士記了,見閻王時也說是道士。那閻王叫查他身上,只見伸出兩隻手來是藍的,問其何故。那博士打著宣科的聲音道:『曾與溫元帥搔胞。』」說的眾人大笑。一面又轉過右首來,見下首供著個紅臉的卻是關帝。上首又是一個黑面的是趙元壇元帥,身邊畫著一個大老虎。白賚光指著道:「哥,你看這老虎,難道是吃素的,隨著人不妨事麼?」伯爵笑道:「你不知,這老虎是他一個親隨的伴當兒哩。」謝希大聽得走過來,伸出舌頭道:「這等一個伴當隨著,我一刻也成不的。我不怕他要吃我麼?」伯爵笑著向西門慶道:「這等虧他怎地過來!」西門慶道:「卻怎的說?」伯爵道:「子純一個要吃他的伴當隨不的,似我們這等七八個要吃你的隨你,卻不嚇死了你罷了。」說著,一齊正大笑時,吳道官走過來,說道:「官人們講這老虎,只俺這清河縣,這兩日好不受這老虎的虧!往來的人也不知吃了多少,就是獵戶,也害死了十來人。」西門慶問道:「是怎的來?」吳道官道:「官人們還不知道。不然我也不曉的,只因日前一個小徒,到滄州橫海郡柴大官人那裡去化些錢糧,整整住了五七日,才得過來。俺這清河縣近著滄州路上,有一條景陽岡,岡上新近出了一個弔睛白額老虎,時常出來吃人。客商過往,好生難走,必須要成群結夥而過。如今縣裡現出著五十兩賞錢,要拿他,白拿不得。可憐這些獵戶,不知吃了多少限棒哩!」白賚光跳起來道:「咱今日結拜了,明日就去拿他,也得些銀子使。」西門慶道:「你性命不值錢麼?」白賚光笑道:「有了銀子,要性命怎的!」眾人齊笑起來。應伯爵道:「我再說個笑話你們聽:一個人被虎銜了,他兒子要救他,拿刀去殺那虎。這人在虎口裡叫道:『兒子,你省可而的砍,怕砍壞了虎皮。』」說著眾人哈哈大笑。

只見吳道官打點牲禮停當,來說道:「官人們燒紙罷。」一面取出疏紙來,說:「疏已寫了,只是那位居長?那位居次?排列了,好等小道書寫尊諱。」眾人一齊道:「這自然是西門大官人居長。」西門慶道:「這還是敘齒,應二哥大如我,是應二哥居長。」伯爵伸著舌頭道:「爺,可不折殺小人罷了!如今年時,只好敘些財勢,那裡好敘齒!若敘齒,這還有大如我的哩。且是我做大哥,有兩件不妥:第一不如大官人有威有德,眾兄弟都服你;第二我原叫做應二哥,如今居長,卻又要叫應大哥,倘或有兩個人來,一個叫『應二哥』,一個叫『應大哥』,我還是應『應二哥』,應『應大哥』呢?」西門慶笑道:「你這搊斷腸子的,單有這些閑說的!」謝希大道:「哥,休推了。」西門慶再三謙讓,被花子虛、應伯爵等一干人逼勒不過,只得做了大哥。第二便是應伯爵,第三謝希大,第四讓花子虛有錢做了四哥。其餘挨次排列。吳道官寫完疏紙,於是點起香燭,眾人依次排列。吳道官伸開疏紙朗聲讀道:
\begin{quote}
維大宋國山東東平府清河縣信士西門慶、應伯爵、謝希大、花子虛、孫天化、祝念實、雲理守、吳典恩、常峙節、白賚光等,是日沐手焚香請旨。伏為桃園義重,眾心仰慕而敢效其風;管鮑情深,各姓追維而欲同其志。況四海皆可兄弟,豈異姓不如骨肉?是以涓今政和年月日,營備豬羊牲禮,鸞馭金資,瑞叩齋壇,虔誠請禱,拜投昊天金闕玉皇上帝,五方值日功曹,本縣城隍社令,過往一切神祇,仗此真香,普同鑒察。伏念慶等生雖異日,死冀同時,期盟言之永固;安樂與共,顛沛相扶,思締結以常新。必富貴常念貧窮,乃始終有所依倚。情共日往以月來,誼若天高而地厚。伏願自盟以後,相好無尤,更祈人人增有永之年,戶戶慶無疆之福。凡在時中,全叨覆庇,謹疏。
政和年月日文疏
\end{quote}

吳道官讀畢,眾人拜神已罷,依次又在神前交拜了八拜。然後送神,焚化錢紙,收下福禮去。不一時,吳道官又早叫人把豬羊卸開,雞魚果品之類整理停當,俱是大碗大盤擺下兩桌,西門慶居於首席,其餘依次而坐,吳道官側席相陪。須臾,酒過數巡,眾人猜枚行令,耍笑鬨堂,不必細說。正是:
\begin{quote}
才見扶桑日出,又看曦馭銜山。醉後倩人扶去,樹梢新月彎彎。
\end{quote}

飲酒熱鬧間,只見玳安兒來附西門慶耳邊說道:「娘叫小的接爹來了,說三娘今日發昏哩,請爹早些家去。」西門慶隨即立起來說道:「不是我搖席破座,委的我第三個小妾十分病重,咱先去休。」只見花子虛道:「咱與哥同路,咱兩個一搭兒去罷。」伯爵道:「你兩個財主的都去了,丟下俺們怎的!花二哥你再坐回去。」西門慶道:「他家無人,俺兩個一搭里去的是,省和他嫂子疑心。」玳安兒道:「小的來時,二娘也叫天福兒備馬來了。」只見一個小廝走近前,向子虛道:「馬在這裡,娘請爹家去哩。」於是二人一齊起身,向吳道官致謝打攪,與伯爵等舉手道:「你們自在耍耍,我們去也。」說著出門上馬去了。單留下這幾個嚼倒泰山不謝土的,在廟流連痛飲不題。

卻表西門慶到家,與花子虛別了進來,問吳月娘:「卓二姐怎的發昏來?」月娘道:「我說一個病人在家,恐怕你搭了這起人又纏到那裡去了,故此叫玳安兒恁地說。只是一日日覺得重來,你也要在家看他的是。」西門慶聽了,往那邊去看,連日在家守著不題。

卻說光陰過隙,又早是十月初十外了。一日,西門慶正使小廝請太醫診視卓二姐病癥,剛走到廳上,只見應伯爵笑嘻嘻走將進來。西門慶與他作了揖,讓他坐了。伯爵道:「哥,嫂子病體如何?」西門慶道:「多分有些不起解,不知怎的好。」因問:「你們前日多咱時分才散?」伯爵道:「承吳道官再三苦留,散時也有二更多天氣。咱醉的要不的,倒是哥早早來家的便益些。」西門慶因問道:「你吃了飯不曾?」伯爵不好說不曾吃,因說道:「哥,你試猜。」西門慶道:「你敢是吃了?」伯爵掩口道:「這等猜不著。」西門慶笑道:「怪狗才,不吃便說不曾吃,有這等張致的!」一面叫小廝:「看飯來,咱與二叔吃。」伯爵笑道:「不然咱也吃了來了,咱聽得一件稀罕的事兒,來與哥說,要同哥去瞧瞧。」西門慶道:「甚麼稀罕的?」伯爵道:「就是前日吳道官所說的景陽岡上那隻大蟲,昨日被一個人一頓拳頭打死了。」西門慶道:「你又來胡說了,咱不信。」伯爵道:「哥,說也不信,你聽著,等我細說。」於是手舞足蹈說道:「這個人有名有姓,姓武名松,排行第二。」先前怎的避難在柴大官人莊上,後來怎的害起病來,病好了又怎的要去尋他哥哥,過這景陽岡來,怎的遇了這虎,怎的怎的被他一頓拳腳打死了。一五一十說來,就象是親見的一般,又象這隻猛虎是他打的一般。說畢,西門慶搖著頭兒道:「既恁的,咱與你吃了飯同去看來。」伯爵道:「哥,不吃罷,怕誤過了。咱們倒不如大街上酒樓上去坐罷。」只見來興兒來放桌兒,西門慶道:「對你娘說,叫別要看飯了,拿衣服來我穿。」

須臾,換了衣服,與伯爵手拉著手兒同步出來。路上撞著謝希大,笑道:「哥們,敢是來看打虎的麼?」西門慶道:「正是。」謝希大道:「大街上好挨擠不開哩。」於是一同到臨街一個大酒樓上坐下。不一時,只聽得鑼鳴鼓響,眾人都一齊瞧看。只見一對對纓槍的獵戶,擺將過來,後面便是那打死的老虎,好象錦布袋一般,四個人還抬不動。末後一匹大白馬上,坐著一個壯士,就是那打虎的這個人。西門慶看了,咬著指頭道:「你說這等一個人,若沒有千百斤水牛般氣力,怎能夠動他一動兒。」這裡三個兒飲酒評品,按下不題。

單表迎來的這個壯士怎生模樣?但見:
\begin{quote}
雄軀凜凜,七尺以上身材;闊面稜稜,二十四五年紀。雙目直豎,遠望處猶如兩點明星;兩手握來,近覷時好似一雙鐵碓。腳尖飛起,深山虎豹失精魂;拳手落時,窮谷熊羆皆喪魄。頭戴著一頂萬字頭巾,上簪兩朵銀花;身穿著一領血腥衲襖,披著一方紅錦。
\end{quote}

這人不是別人,就是應伯爵說所陽谷縣的武二郎。只為要來尋他哥子,不意中打死了這個猛虎,被知縣迎請將來。眾人看著他迎入縣裡。卻說這時正值知縣升堂,武鬆下馬進去,扛著大蟲在廳前。知縣看了武鬆這般模樣,心中自忖道:「不恁地,怎打得這個猛虎!」便喚武鬆上廳。參見畢,將打虎首尾訴說一遍。兩邊官吏都嚇呆了。知縣在廳上賜了三杯酒,將庫中眾土戶出納的賞錢五十兩,賜與武鬆。武鬆稟道:「小人托賴相公福蔭,偶然僥幸打死了這個大蟲,非小人之能,如何敢受這些賞賜!眾獵戶因這畜生,受了相公許多責罰,何不就把賞給散與眾人,也顯得相公恩典。」知縣道:「既是如此,任從壯士處分。」武鬆就把這五十兩賞錢,在廳上散與眾獵戶傅去了。知縣見他仁德忠厚,又是一條好漢,有心要抬舉他,便道:「你雖是陽谷縣人氏,與我這清河縣只在咫尺。我今日就參你在我縣裡做個巡捕的都頭,專在河東水西擒拿賊盜,你意下如何?」武鬆跪謝道:「若蒙恩相抬舉,小人終身受賜。」知縣隨即喚押司立了文案,當日便參武鬆做了巡捕都頭。眾里長大戶都來與武鬆作賀慶喜,連連吃了數日酒。正要回陽谷縣去抓尋哥哥,不料又在清河縣做了都頭,卻也歡喜。那時傳得東平一府兩縣,皆知武鬆之名。正是:
\begin{quote}
壯士英雄藝略芳,挺身直上景陽岡。
醉來打死山中虎,自此聲名播四方。
\end{quote}

卻說武鬆一日在街上閑行,只聽背後一個人叫道:「兄弟,知縣相公抬舉你做了巡捕都頭,怎不看顧我!」武鬆回頭見了這人,不覺的——
\begin{quote}
欣從額角眉邊出,喜逐歡容笑口開。
\end{quote}

這人不是別人,卻是武鬆日常間要去尋他的嫡親哥哥武大。卻說武大自從兄弟分別之後,因時遭饑饉,搬移在清河縣紫石街賃房居住。人見他為人懦弱,模樣猥蕤,起了他個渾名叫做三寸丁谷樹皮,俗語言其身上粗糙,頭臉窄狹故也。只因他這般軟弱朴實,多欺侮也。這也不在話下。且說武大無甚生意,終日挑擔子出去街上賣炊餅度日,不幸把渾家故了,丟下個女孩兒,年方十二歲,名喚迎兒,爺兒兩個過活。那消半年光景,又消折了資本,移在大街坊張大戶家臨街房居住。張宅家下人見他本分,常看顧他,照顧他依舊賣些炊餅。閑時在鋪中坐地,武大無不奉承。因此張宅家下人個個都歡喜,在大戶面前一力與他說方便。因此大戶連房錢也不問武大要。

卻說這張大戶有萬貫家財,百間房屋,年約六旬之上,身邊寸男尺女皆無。媽媽餘氏,主家嚴厲,房中並無清秀使女。只因大戶時常拍胸嘆氣道:「我許大年紀,又無兒女,雖有幾貫家財,終何大用。」媽媽道:「既然如此說,我叫媒人替你買兩個使女,早晚習學彈唱,服侍你便了。」大戶聽了大喜,謝了媽媽。過了幾時,媽媽果然叫媒人來,與大戶買了兩個使女,一個叫做潘金蓮,一個喚做白玉蓮。玉蓮年方二八,樂戶人家出身,生得白凈小巧。這潘金蓮卻是南門外潘裁的女兒,排行六姐。因他自幼生得有些姿色,纏得一雙好小腳兒,所以就叫金蓮。他父親死了,做娘的度日不過,從九歲賣在王招宣府里,習學彈唱,閑常又教他讀書寫字。他本性機變伶俐,不過十二三,就會描眉畫眼,傅粉施朱,品竹彈絲,女工針指,知書識字,梳一個纏髻兒,著一件扣身衫子,做張做致,喬模喬樣。到十五歲的時節,王招宣死了,潘媽媽爭將出來,三十兩銀子轉賣於張大戶家,與玉蓮同時進門。大戶教他習學彈唱,金蓮原自會的,甚是省力。金蓮學琵琶,玉蓮學箏,這兩個同房歇臥。主家婆餘氏初時甚是抬舉二人,與他金銀首飾裝束身子。後日不料白玉蓮死了,止落下金蓮一人,長成一十八歲,出落的臉襯桃花,眉彎新月。張大戶每要收他,只礙主家婆厲害,不得到手。一日主家婆鄰家赴席不在,大戶暗把金蓮喚至房中,遂收用了。正是:
\begin{quote}
莫訝天台相見晚,劉郎還是老劉郎。
\end{quote}

大戶自從收用金蓮之後,不覺身上添了四五件病癥。端的那五件?第一腰便添疼,第二眼便添淚,第三耳便添聾,第四鼻便添涕,第五尿便添滴。自有了這幾件病後,主家婆頗知其事,與大戶嚷罵了數日,將金蓮百般苦打。大戶知道不容,卻賭氣倒賠了房奩,要尋嫁得一個相應的人家。大戶家下人都說武大忠厚,見無妻小,又住著宅內房兒,堪可與他。這大戶早晚還要看覷此女,因此不要武大一文錢,白白地嫁與他為妻。這武大自從娶了金蓮,大戶甚是看顧他。若武大沒本錢做炊餅,大戶私與他銀兩。武大若挑擔兒出去,大戶候無人,便踅入房中與金蓮廝會。武大雖一時撞見,原是他的行貨,不敢聲言。朝來暮往,也有多時。忽一日大戶得患陰寒病癥,嗚呼死了。主家婆察知其事,怒令家僮將金蓮、武大即時趕出。武大故此遂尋了紫石街西王皇親房子,賃內外兩間居住,依舊賣炊餅。

原來這金蓮自嫁武大,見他一味老實,人物猥瑣,甚是憎嫌,常與他合氣。報怨大戶:「普天世界斷生了男子,何故將我嫁與這樣個貨!每日牽著不走,打著倒退的,只是一味吃酒,著緊處卻是錐鈀也不動。奴端的那世里悔氣,卻嫁了他!是好苦也!」常無人處,唱個《山坡羊》為證:
\begin{quote}
想當初,姻緣錯配,奴把你當男兒漢看覷。不是奴自己誇獎,他烏鴉怎配鸞鳳對!奴真金子埋在土裡,他是塊高號銅,怎與俺金色比!他本是塊頑石,有甚福抱著我羊脂玉體!好似糞土上長出靈芝。奈何,隨他怎樣,到底奴心不美。聽知:奴是塊金磚,怎比泥土基!
\end{quote}

看官聽說:但凡世上婦女,若自己有幾分顏色,所稟伶俐,配個好男子便罷了,若是武大這般,雖好殺也未免有幾分憎嫌。自古佳人才子相配著的少,買金偏撞不著賣金的。

武大每日自挑擔兒出去賣炊餅,到晚方歸。那婦人每日打發武大出門,只在帘子下嗑瓜子兒,一徑把那一對小金蓮故露出來,勾引浮浪子弟,日逐在門前彈胡博詞,撒謎語,叫唱:「一塊好羊肉,如何落在狗嘴裡?」油似滑的言語,無般不說出來。因此武大在紫石街又住不牢,要往別處搬移,與老婆商議。婦人道:「賊餛飩不曉事的,你賃人家房住,淺房淺屋,可知有小人羅唣!不如添幾兩銀子,看相應的,典上他兩間住,卻也氣概些,免受人欺侮。」武大道:「我那裡有錢典房?」婦人道:「呸!濁才料,你是個男子漢,倒擺佈不開,常交老娘受氣。沒有銀子,把我的釵梳湊辦了去,有何難處!過後有了再治不遲。」武大聽老婆這般說,當下湊了十數兩銀子,典得縣門前樓上下兩層四間房屋居住。第二層是樓,兩個小小院落,甚是乾凈。

武大自從搬到縣西街上來,照舊賣炊餅過活,不想這日撞見自己嫡親兄弟。當日兄弟相見,心中大喜。一面邀請到家中,讓至樓上坐,房裡喚出金蓮來,與武鬆相見。因說道:「前日景陽岡上打死大蟲的,便是你的小叔。今新充了都頭,是我一母同胞兄弟。」那婦人叉手向前,便道:「叔叔萬福。」武鬆施禮,倒身下拜。婦人扶住武鬆道:「叔叔請起,折殺奴家。」武鬆道:「嫂嫂受禮。」兩個相讓了一回,都平磕了頭起來。少頃,小女迎兒拿茶,二人吃了。武鬆見婦人十分妖嬈,只把頭來低著。不多時,武大安排酒飯,款待武鬆。

說話中間,武大下樓買酒菜去了,丟下婦人,獨自在樓上陪武鬆坐地。看了武鬆身材凜凜,相貌堂堂,又想他打死了那大蟲,畢竟有千百斤氣力。口中不說,心下思量道:「一母所生的兄弟,怎生我家那身不滿尺的丁樹,三分似人七分似鬼,奴那世里遭瘟撞著他來!如今看起武鬆這般人壯健,何不叫他搬來我家住?想這段姻緣卻在這裡了。」於是一面堆下笑來,問道:「叔叔你如今在那裡居住?每日飯食誰人整理?」武鬆道:「武二新充了都頭,逐日答應上司,別處住不方便,胡亂在縣前尋了個下處,每日撥兩個土兵伏侍做飯。」婦人道:「叔叔何不搬來家裡住?省的在縣前土兵服侍做飯腌臢。一家裡住,早晚要些湯水吃時,也方便些。就是奴家親自安排與叔叔吃,也乾凈。」武鬆道:「深謝嫂嫂。」婦人又道:「莫不別處有嬸嬸?可請來廝會。」武鬆道:「武二並不曾婚娶。」婦人道:「叔叔青春多少?」武鬆道:「虛度二十八歲。」婦人道:「原來叔叔倒長奴三歲。叔叔今番從那裡來?」武鬆道:「在滄州住了一年有餘,只想哥哥在舊房居住,不道移在這裡。」婦人道:「一言難盡。自從嫁得你哥哥,吃他忒善了,被人欺負,才到這裡來。若是叔叔這般雄壯,誰敢道個不字!」武鬆道:「家兄從來本分,不似武鬆撒潑。」婦人笑道:「怎的顛倒說!常言:人無剛強,安身不長。奴家平生性快,看不上那三打不回頭,四打和身轉的」武鬆道:「家兄不惹禍,免得嫂嫂憂心。」二人在樓上一遞一句的說。有詩為證:
\begin{quote}
叔嫂萍蹤得偶逢,嬌嬈偏逞秀儀容。
私心便欲成歡會,暗把邪言釣武松。
\end{quote}

話說金蓮陪著武松正在樓上說話未了,只見武大買了些肉菜果餅歸家。放在廚,走上樓來,叫道:「大嫂,你且下來則個。」那婦人應道:「你看那不曉事的!叔叔在此無人陪侍,卻交我撇了下去。」武鬆道:「嫂嫂請方便。」婦人道:「何不去間壁請王乾娘來安排?只是這般不見便。」武大便自去央了間壁王婆來。安排端正,都拿上樓來,擺在桌子上,無非是些魚肉果菜點心之類。隨即燙酒上來。武大叫婦人坐了主位,武鬆對席,武大打橫。三人坐下,把酒來斟,武大篩酒在各人面前。那婦人拿起酒來道:「叔叔休怪,沒甚管待,請杯兒水酒。」武鬆道:「感謝嫂嫂,休這般說。」武大隻顧上下篩酒,那婦人笑容可掬,滿口兒叫:「叔叔,怎的肉果兒也不揀一箸兒?」揀好的遞將過來。武鬆是個直性的漢子,只把做親嫂嫂相待。誰知這婦人是個使女出身,慣會小意兒。亦不想這婦人一片引人心。那婦人陪武鬆吃了幾杯酒,一雙眼只看著武鬆的身上。武鬆吃他看不過,只得倒低了頭。吃了一歇,酒闌了,便起身。武大道:「二哥沒事,再吃幾杯兒去。」武鬆道:「生受,我再來望哥哥嫂嫂罷。」都送下樓來。出的門外,婦人便道:「叔叔是必上心搬來家裡住,若是不搬來,俺兩口兒也吃別人笑話。親兄弟難比別人,與我們爭口氣,也是好處。」武松道:「既是嫂嫂厚意,今晚有行李便取來。」婦人道:「奴這裡等候哩!」正是:
\begin{quote}
滿前野意無人識,幾點碧桃春自開。
\end{quote}
